
% Default to the notebook output style

    


% Inherit from the specified cell style.




    
\documentclass[11pt]{article}

    
    
    \usepackage[T1]{fontenc}
    % Nicer default font (+ math font) than Computer Modern for most use cases
    \usepackage{mathpazo}

    % Basic figure setup, for now with no caption control since it's done
    % automatically by Pandoc (which extracts ![](path) syntax from Markdown).
    \usepackage{graphicx}
    % We will generate all images so they have a width \maxwidth. This means
    % that they will get their normal width if they fit onto the page, but
    % are scaled down if they would overflow the margins.
    \makeatletter
    \def\maxwidth{\ifdim\Gin@nat@width>\linewidth\linewidth
    \else\Gin@nat@width\fi}
    \makeatother
    \let\Oldincludegraphics\includegraphics
    % Set max figure width to be 80% of text width, for now hardcoded.
    \renewcommand{\includegraphics}[1]{\Oldincludegraphics[width=.8\maxwidth]{#1}}
    % Ensure that by default, figures have no caption (until we provide a
    % proper Figure object with a Caption API and a way to capture that
    % in the conversion process - todo).
    \usepackage{caption}
    \DeclareCaptionLabelFormat{nolabel}{}
    \captionsetup{labelformat=nolabel}

    \usepackage{adjustbox} % Used to constrain images to a maximum size 
    \usepackage{xcolor} % Allow colors to be defined
    \usepackage{enumerate} % Needed for markdown enumerations to work
    \usepackage{geometry} % Used to adjust the document margins
    \usepackage{amsmath} % Equations
    \usepackage{amssymb} % Equations
    \usepackage{textcomp} % defines textquotesingle
    % Hack from http://tex.stackexchange.com/a/47451/13684:
    \AtBeginDocument{%
        \def\PYZsq{\textquotesingle}% Upright quotes in Pygmentized code
    }
    \usepackage{upquote} % Upright quotes for verbatim code
    \usepackage{eurosym} % defines \euro
    \usepackage[mathletters]{ucs} % Extended unicode (utf-8) support
    \usepackage[utf8x]{inputenc} % Allow utf-8 characters in the tex document
    \usepackage{fancyvrb} % verbatim replacement that allows latex
    \usepackage{grffile} % extends the file name processing of package graphics 
                         % to support a larger range 
    % The hyperref package gives us a pdf with properly built
    % internal navigation ('pdf bookmarks' for the table of contents,
    % internal cross-reference links, web links for URLs, etc.)
    \usepackage{hyperref}
    \usepackage{longtable} % longtable support required by pandoc >1.10
    \usepackage{booktabs}  % table support for pandoc > 1.12.2
    \usepackage[inline]{enumitem} % IRkernel/repr support (it uses the enumerate* environment)
    \usepackage[normalem]{ulem} % ulem is needed to support strikethroughs (\sout)
                                % normalem makes italics be italics, not underlines
    \usepackage{mathrsfs}
    

    
    
    % Colors for the hyperref package
    \definecolor{urlcolor}{rgb}{0,.145,.698}
    \definecolor{linkcolor}{rgb}{.71,0.21,0.01}
    \definecolor{citecolor}{rgb}{.12,.54,.11}

    % ANSI colors
    \definecolor{ansi-black}{HTML}{3E424D}
    \definecolor{ansi-black-intense}{HTML}{282C36}
    \definecolor{ansi-red}{HTML}{E75C58}
    \definecolor{ansi-red-intense}{HTML}{B22B31}
    \definecolor{ansi-green}{HTML}{00A250}
    \definecolor{ansi-green-intense}{HTML}{007427}
    \definecolor{ansi-yellow}{HTML}{DDB62B}
    \definecolor{ansi-yellow-intense}{HTML}{B27D12}
    \definecolor{ansi-blue}{HTML}{208FFB}
    \definecolor{ansi-blue-intense}{HTML}{0065CA}
    \definecolor{ansi-magenta}{HTML}{D160C4}
    \definecolor{ansi-magenta-intense}{HTML}{A03196}
    \definecolor{ansi-cyan}{HTML}{60C6C8}
    \definecolor{ansi-cyan-intense}{HTML}{258F8F}
    \definecolor{ansi-white}{HTML}{C5C1B4}
    \definecolor{ansi-white-intense}{HTML}{A1A6B2}
    \definecolor{ansi-default-inverse-fg}{HTML}{FFFFFF}
    \definecolor{ansi-default-inverse-bg}{HTML}{000000}

    % commands and environments needed by pandoc snippets
    % extracted from the output of `pandoc -s`
    \providecommand{\tightlist}{%
      \setlength{\itemsep}{0pt}\setlength{\parskip}{0pt}}
    \DefineVerbatimEnvironment{Highlighting}{Verbatim}{commandchars=\\\{\}}
    % Add ',fontsize=\small' for more characters per line
    \newenvironment{Shaded}{}{}
    \newcommand{\KeywordTok}[1]{\textcolor[rgb]{0.00,0.44,0.13}{\textbf{{#1}}}}
    \newcommand{\DataTypeTok}[1]{\textcolor[rgb]{0.56,0.13,0.00}{{#1}}}
    \newcommand{\DecValTok}[1]{\textcolor[rgb]{0.25,0.63,0.44}{{#1}}}
    \newcommand{\BaseNTok}[1]{\textcolor[rgb]{0.25,0.63,0.44}{{#1}}}
    \newcommand{\FloatTok}[1]{\textcolor[rgb]{0.25,0.63,0.44}{{#1}}}
    \newcommand{\CharTok}[1]{\textcolor[rgb]{0.25,0.44,0.63}{{#1}}}
    \newcommand{\StringTok}[1]{\textcolor[rgb]{0.25,0.44,0.63}{{#1}}}
    \newcommand{\CommentTok}[1]{\textcolor[rgb]{0.38,0.63,0.69}{\textit{{#1}}}}
    \newcommand{\OtherTok}[1]{\textcolor[rgb]{0.00,0.44,0.13}{{#1}}}
    \newcommand{\AlertTok}[1]{\textcolor[rgb]{1.00,0.00,0.00}{\textbf{{#1}}}}
    \newcommand{\FunctionTok}[1]{\textcolor[rgb]{0.02,0.16,0.49}{{#1}}}
    \newcommand{\RegionMarkerTok}[1]{{#1}}
    \newcommand{\ErrorTok}[1]{\textcolor[rgb]{1.00,0.00,0.00}{\textbf{{#1}}}}
    \newcommand{\NormalTok}[1]{{#1}}
    
    % Additional commands for more recent versions of Pandoc
    \newcommand{\ConstantTok}[1]{\textcolor[rgb]{0.53,0.00,0.00}{{#1}}}
    \newcommand{\SpecialCharTok}[1]{\textcolor[rgb]{0.25,0.44,0.63}{{#1}}}
    \newcommand{\VerbatimStringTok}[1]{\textcolor[rgb]{0.25,0.44,0.63}{{#1}}}
    \newcommand{\SpecialStringTok}[1]{\textcolor[rgb]{0.73,0.40,0.53}{{#1}}}
    \newcommand{\ImportTok}[1]{{#1}}
    \newcommand{\DocumentationTok}[1]{\textcolor[rgb]{0.73,0.13,0.13}{\textit{{#1}}}}
    \newcommand{\AnnotationTok}[1]{\textcolor[rgb]{0.38,0.63,0.69}{\textbf{\textit{{#1}}}}}
    \newcommand{\CommentVarTok}[1]{\textcolor[rgb]{0.38,0.63,0.69}{\textbf{\textit{{#1}}}}}
    \newcommand{\VariableTok}[1]{\textcolor[rgb]{0.10,0.09,0.49}{{#1}}}
    \newcommand{\ControlFlowTok}[1]{\textcolor[rgb]{0.00,0.44,0.13}{\textbf{{#1}}}}
    \newcommand{\OperatorTok}[1]{\textcolor[rgb]{0.40,0.40,0.40}{{#1}}}
    \newcommand{\BuiltInTok}[1]{{#1}}
    \newcommand{\ExtensionTok}[1]{{#1}}
    \newcommand{\PreprocessorTok}[1]{\textcolor[rgb]{0.74,0.48,0.00}{{#1}}}
    \newcommand{\AttributeTok}[1]{\textcolor[rgb]{0.49,0.56,0.16}{{#1}}}
    \newcommand{\InformationTok}[1]{\textcolor[rgb]{0.38,0.63,0.69}{\textbf{\textit{{#1}}}}}
    \newcommand{\WarningTok}[1]{\textcolor[rgb]{0.38,0.63,0.69}{\textbf{\textit{{#1}}}}}
    
    
    % Define a nice break command that doesn't care if a line doesn't already
    % exist.
    \def\br{\hspace*{\fill} \\* }
    % Math Jax compatibility definitions
    \def\gt{>}
    \def\lt{<}
    \let\Oldtex\TeX
    \let\Oldlatex\LaTeX
    \renewcommand{\TeX}{\textrm{\Oldtex}}
    \renewcommand{\LaTeX}{\textrm{\Oldlatex}}
    % Document parameters
    % Document title
    \title{Working with Data}
    
    
    
    
    

    % Pygments definitions
    
\makeatletter
\def\PY@reset{\let\PY@it=\relax \let\PY@bf=\relax%
    \let\PY@ul=\relax \let\PY@tc=\relax%
    \let\PY@bc=\relax \let\PY@ff=\relax}
\def\PY@tok#1{\csname PY@tok@#1\endcsname}
\def\PY@toks#1+{\ifx\relax#1\empty\else%
    \PY@tok{#1}\expandafter\PY@toks\fi}
\def\PY@do#1{\PY@bc{\PY@tc{\PY@ul{%
    \PY@it{\PY@bf{\PY@ff{#1}}}}}}}
\def\PY#1#2{\PY@reset\PY@toks#1+\relax+\PY@do{#2}}

\expandafter\def\csname PY@tok@w\endcsname{\def\PY@tc##1{\textcolor[rgb]{0.73,0.73,0.73}{##1}}}
\expandafter\def\csname PY@tok@c\endcsname{\let\PY@it=\textit\def\PY@tc##1{\textcolor[rgb]{0.25,0.50,0.50}{##1}}}
\expandafter\def\csname PY@tok@cp\endcsname{\def\PY@tc##1{\textcolor[rgb]{0.74,0.48,0.00}{##1}}}
\expandafter\def\csname PY@tok@k\endcsname{\let\PY@bf=\textbf\def\PY@tc##1{\textcolor[rgb]{0.00,0.50,0.00}{##1}}}
\expandafter\def\csname PY@tok@kp\endcsname{\def\PY@tc##1{\textcolor[rgb]{0.00,0.50,0.00}{##1}}}
\expandafter\def\csname PY@tok@kt\endcsname{\def\PY@tc##1{\textcolor[rgb]{0.69,0.00,0.25}{##1}}}
\expandafter\def\csname PY@tok@o\endcsname{\def\PY@tc##1{\textcolor[rgb]{0.40,0.40,0.40}{##1}}}
\expandafter\def\csname PY@tok@ow\endcsname{\let\PY@bf=\textbf\def\PY@tc##1{\textcolor[rgb]{0.67,0.13,1.00}{##1}}}
\expandafter\def\csname PY@tok@nb\endcsname{\def\PY@tc##1{\textcolor[rgb]{0.00,0.50,0.00}{##1}}}
\expandafter\def\csname PY@tok@nf\endcsname{\def\PY@tc##1{\textcolor[rgb]{0.00,0.00,1.00}{##1}}}
\expandafter\def\csname PY@tok@nc\endcsname{\let\PY@bf=\textbf\def\PY@tc##1{\textcolor[rgb]{0.00,0.00,1.00}{##1}}}
\expandafter\def\csname PY@tok@nn\endcsname{\let\PY@bf=\textbf\def\PY@tc##1{\textcolor[rgb]{0.00,0.00,1.00}{##1}}}
\expandafter\def\csname PY@tok@ne\endcsname{\let\PY@bf=\textbf\def\PY@tc##1{\textcolor[rgb]{0.82,0.25,0.23}{##1}}}
\expandafter\def\csname PY@tok@nv\endcsname{\def\PY@tc##1{\textcolor[rgb]{0.10,0.09,0.49}{##1}}}
\expandafter\def\csname PY@tok@no\endcsname{\def\PY@tc##1{\textcolor[rgb]{0.53,0.00,0.00}{##1}}}
\expandafter\def\csname PY@tok@nl\endcsname{\def\PY@tc##1{\textcolor[rgb]{0.63,0.63,0.00}{##1}}}
\expandafter\def\csname PY@tok@ni\endcsname{\let\PY@bf=\textbf\def\PY@tc##1{\textcolor[rgb]{0.60,0.60,0.60}{##1}}}
\expandafter\def\csname PY@tok@na\endcsname{\def\PY@tc##1{\textcolor[rgb]{0.49,0.56,0.16}{##1}}}
\expandafter\def\csname PY@tok@nt\endcsname{\let\PY@bf=\textbf\def\PY@tc##1{\textcolor[rgb]{0.00,0.50,0.00}{##1}}}
\expandafter\def\csname PY@tok@nd\endcsname{\def\PY@tc##1{\textcolor[rgb]{0.67,0.13,1.00}{##1}}}
\expandafter\def\csname PY@tok@s\endcsname{\def\PY@tc##1{\textcolor[rgb]{0.73,0.13,0.13}{##1}}}
\expandafter\def\csname PY@tok@sd\endcsname{\let\PY@it=\textit\def\PY@tc##1{\textcolor[rgb]{0.73,0.13,0.13}{##1}}}
\expandafter\def\csname PY@tok@si\endcsname{\let\PY@bf=\textbf\def\PY@tc##1{\textcolor[rgb]{0.73,0.40,0.53}{##1}}}
\expandafter\def\csname PY@tok@se\endcsname{\let\PY@bf=\textbf\def\PY@tc##1{\textcolor[rgb]{0.73,0.40,0.13}{##1}}}
\expandafter\def\csname PY@tok@sr\endcsname{\def\PY@tc##1{\textcolor[rgb]{0.73,0.40,0.53}{##1}}}
\expandafter\def\csname PY@tok@ss\endcsname{\def\PY@tc##1{\textcolor[rgb]{0.10,0.09,0.49}{##1}}}
\expandafter\def\csname PY@tok@sx\endcsname{\def\PY@tc##1{\textcolor[rgb]{0.00,0.50,0.00}{##1}}}
\expandafter\def\csname PY@tok@m\endcsname{\def\PY@tc##1{\textcolor[rgb]{0.40,0.40,0.40}{##1}}}
\expandafter\def\csname PY@tok@gh\endcsname{\let\PY@bf=\textbf\def\PY@tc##1{\textcolor[rgb]{0.00,0.00,0.50}{##1}}}
\expandafter\def\csname PY@tok@gu\endcsname{\let\PY@bf=\textbf\def\PY@tc##1{\textcolor[rgb]{0.50,0.00,0.50}{##1}}}
\expandafter\def\csname PY@tok@gd\endcsname{\def\PY@tc##1{\textcolor[rgb]{0.63,0.00,0.00}{##1}}}
\expandafter\def\csname PY@tok@gi\endcsname{\def\PY@tc##1{\textcolor[rgb]{0.00,0.63,0.00}{##1}}}
\expandafter\def\csname PY@tok@gr\endcsname{\def\PY@tc##1{\textcolor[rgb]{1.00,0.00,0.00}{##1}}}
\expandafter\def\csname PY@tok@ge\endcsname{\let\PY@it=\textit}
\expandafter\def\csname PY@tok@gs\endcsname{\let\PY@bf=\textbf}
\expandafter\def\csname PY@tok@gp\endcsname{\let\PY@bf=\textbf\def\PY@tc##1{\textcolor[rgb]{0.00,0.00,0.50}{##1}}}
\expandafter\def\csname PY@tok@go\endcsname{\def\PY@tc##1{\textcolor[rgb]{0.53,0.53,0.53}{##1}}}
\expandafter\def\csname PY@tok@gt\endcsname{\def\PY@tc##1{\textcolor[rgb]{0.00,0.27,0.87}{##1}}}
\expandafter\def\csname PY@tok@err\endcsname{\def\PY@bc##1{\setlength{\fboxsep}{0pt}\fcolorbox[rgb]{1.00,0.00,0.00}{1,1,1}{\strut ##1}}}
\expandafter\def\csname PY@tok@kc\endcsname{\let\PY@bf=\textbf\def\PY@tc##1{\textcolor[rgb]{0.00,0.50,0.00}{##1}}}
\expandafter\def\csname PY@tok@kd\endcsname{\let\PY@bf=\textbf\def\PY@tc##1{\textcolor[rgb]{0.00,0.50,0.00}{##1}}}
\expandafter\def\csname PY@tok@kn\endcsname{\let\PY@bf=\textbf\def\PY@tc##1{\textcolor[rgb]{0.00,0.50,0.00}{##1}}}
\expandafter\def\csname PY@tok@kr\endcsname{\let\PY@bf=\textbf\def\PY@tc##1{\textcolor[rgb]{0.00,0.50,0.00}{##1}}}
\expandafter\def\csname PY@tok@bp\endcsname{\def\PY@tc##1{\textcolor[rgb]{0.00,0.50,0.00}{##1}}}
\expandafter\def\csname PY@tok@fm\endcsname{\def\PY@tc##1{\textcolor[rgb]{0.00,0.00,1.00}{##1}}}
\expandafter\def\csname PY@tok@vc\endcsname{\def\PY@tc##1{\textcolor[rgb]{0.10,0.09,0.49}{##1}}}
\expandafter\def\csname PY@tok@vg\endcsname{\def\PY@tc##1{\textcolor[rgb]{0.10,0.09,0.49}{##1}}}
\expandafter\def\csname PY@tok@vi\endcsname{\def\PY@tc##1{\textcolor[rgb]{0.10,0.09,0.49}{##1}}}
\expandafter\def\csname PY@tok@vm\endcsname{\def\PY@tc##1{\textcolor[rgb]{0.10,0.09,0.49}{##1}}}
\expandafter\def\csname PY@tok@sa\endcsname{\def\PY@tc##1{\textcolor[rgb]{0.73,0.13,0.13}{##1}}}
\expandafter\def\csname PY@tok@sb\endcsname{\def\PY@tc##1{\textcolor[rgb]{0.73,0.13,0.13}{##1}}}
\expandafter\def\csname PY@tok@sc\endcsname{\def\PY@tc##1{\textcolor[rgb]{0.73,0.13,0.13}{##1}}}
\expandafter\def\csname PY@tok@dl\endcsname{\def\PY@tc##1{\textcolor[rgb]{0.73,0.13,0.13}{##1}}}
\expandafter\def\csname PY@tok@s2\endcsname{\def\PY@tc##1{\textcolor[rgb]{0.73,0.13,0.13}{##1}}}
\expandafter\def\csname PY@tok@sh\endcsname{\def\PY@tc##1{\textcolor[rgb]{0.73,0.13,0.13}{##1}}}
\expandafter\def\csname PY@tok@s1\endcsname{\def\PY@tc##1{\textcolor[rgb]{0.73,0.13,0.13}{##1}}}
\expandafter\def\csname PY@tok@mb\endcsname{\def\PY@tc##1{\textcolor[rgb]{0.40,0.40,0.40}{##1}}}
\expandafter\def\csname PY@tok@mf\endcsname{\def\PY@tc##1{\textcolor[rgb]{0.40,0.40,0.40}{##1}}}
\expandafter\def\csname PY@tok@mh\endcsname{\def\PY@tc##1{\textcolor[rgb]{0.40,0.40,0.40}{##1}}}
\expandafter\def\csname PY@tok@mi\endcsname{\def\PY@tc##1{\textcolor[rgb]{0.40,0.40,0.40}{##1}}}
\expandafter\def\csname PY@tok@il\endcsname{\def\PY@tc##1{\textcolor[rgb]{0.40,0.40,0.40}{##1}}}
\expandafter\def\csname PY@tok@mo\endcsname{\def\PY@tc##1{\textcolor[rgb]{0.40,0.40,0.40}{##1}}}
\expandafter\def\csname PY@tok@ch\endcsname{\let\PY@it=\textit\def\PY@tc##1{\textcolor[rgb]{0.25,0.50,0.50}{##1}}}
\expandafter\def\csname PY@tok@cm\endcsname{\let\PY@it=\textit\def\PY@tc##1{\textcolor[rgb]{0.25,0.50,0.50}{##1}}}
\expandafter\def\csname PY@tok@cpf\endcsname{\let\PY@it=\textit\def\PY@tc##1{\textcolor[rgb]{0.25,0.50,0.50}{##1}}}
\expandafter\def\csname PY@tok@c1\endcsname{\let\PY@it=\textit\def\PY@tc##1{\textcolor[rgb]{0.25,0.50,0.50}{##1}}}
\expandafter\def\csname PY@tok@cs\endcsname{\let\PY@it=\textit\def\PY@tc##1{\textcolor[rgb]{0.25,0.50,0.50}{##1}}}

\def\PYZbs{\char`\\}
\def\PYZus{\char`\_}
\def\PYZob{\char`\{}
\def\PYZcb{\char`\}}
\def\PYZca{\char`\^}
\def\PYZam{\char`\&}
\def\PYZlt{\char`\<}
\def\PYZgt{\char`\>}
\def\PYZsh{\char`\#}
\def\PYZpc{\char`\%}
\def\PYZdl{\char`\$}
\def\PYZhy{\char`\-}
\def\PYZsq{\char`\'}
\def\PYZdq{\char`\"}
\def\PYZti{\char`\~}
% for compatibility with earlier versions
\def\PYZat{@}
\def\PYZlb{[}
\def\PYZrb{]}
\makeatother


    % Exact colors from NB
    \definecolor{incolor}{rgb}{0.0, 0.0, 0.5}
    \definecolor{outcolor}{rgb}{0.545, 0.0, 0.0}



    
    % Prevent overflowing lines due to hard-to-break entities
    \sloppy 
    % Setup hyperref package
    \hypersetup{
      breaklinks=true,  % so long urls are correctly broken across lines
      colorlinks=true,
      urlcolor=urlcolor,
      linkcolor=linkcolor,
      citecolor=citecolor,
      }
    % Slightly bigger margins than the latex defaults
    
    \geometry{verbose,tmargin=1in,bmargin=1in,lmargin=1in,rmargin=1in}
    
    

    \begin{document}
    
    
    \maketitle
    
    

    
    \section{Working with Data}\label{working-with-data}

The purpose of this chapter is to familiarize the reader with some of
the basic of working with data in Julia. As would be expected, much of
the focus of this chapter is on or around dataframes, including
dataframe functions. Other topic covered include categorical data,
input-output (IO), and the split-apply-combine strategy.

    \subsection{Dataframes}\label{dataframes}

A dataframe is a tabular representation of data, similar to a
spreadsheet or a data matrix. As with a matrix, the observations are
rows and the variables are columns. Each row is a single (vectorvalued)
observation. For a single row, i.e., observation, each column represents
a single realization of a variable. At this stage, it may be helppful to
explicitly draw the analogy between a dataframe and the more formal
notation often used in statistics and data science.

Suppose we observe \(n\) realizations
\(\mathbf{x}_1,\cdots,\mathbf{x}_n\) of \(p\)-dimensional random
variables \(\mathbf{X}_1,\cdots,\mathbf{X}_n\), where
\(\mathbf{X}_i=(X_{i1},X_{i2},\cdots,X_{ip})^{\prime}\) for
\(i=1,\cdots,n\). In matrix form, this can be written

\begin{equation}
\mathscr{X}=\left(\mathbf{X}_{1}, \mathbf{X}_{2}, \ldots, \mathbf{X}_{n}\right)^{\prime}=\left(\begin{array}{c}{\mathbf{X}_{1}^{\prime}} \\ {\mathbf{X}_{2}^{\prime}} \\ {\vdots} \\ {\mathbf{X}_{n}^{\prime}}\end{array}\right)=\left(\begin{array}{cccc}{X_{11}} & {X_{12}} & {\cdots} & {X_{1 p}} \\ {X_{21}} & {X_{22}} & {\cdots} & {X_{2 p}} \\ {\vdots} & {\vdots} & {\ddots} & {\vdots} \\ {X_{n 1}} & {X_{n 2}} & {\cdots} & {X_{n p}}\end{array}\right).
\end{equation}

Now, \(\mathbf{X}_i\) is called a random vector and \(\mathscr{X}\) is
called an \(n\times p\) random matrix. A realization of \(\mathscr{X}\)
can be considered a data matrix. For completeness, note that a matrix
\(\mathbf{A}\) with all entries constant is called a constant matrix.

Consider, for example, data on the weight and height of 500 people. Let
\(\mathbf{x}_i=(x_{i1},x_{i2})^{\prime}\) be the associated observation
fro the \(i\)th person, \(i=1,2,\cdots,500\), where \(x_{i1}\)
represents their weight and \(x_{i2}\) represents their height. The
associated data matrix is then

\begin{equation}
\mathscr{X}=\left(\mathrm{x}_{1}, \mathrm{x}_{2}, \ldots, \mathrm{x}_{500}\right)^{\prime}=\left(\begin{array}{c}{\mathrm{x}_{1}^{\prime}} \\ {\mathrm{x}_{2}^{\prime}} \\ {\vdots} \\ {\mathrm{x}_{500}^{\prime}}\end{array}\right)=\left(\begin{array}{cc}{x_{11}} & {x_{12}} \\ {x_{21}} & {x_{22}} \\ {\vdots} & {\vdots} \\ {x_{500,1}} & {x_{500,2}}\end{array}\right)
\end{equation}

A dataframe is a computer representation of a data matrix. In Julia, the
\texttt{DataFrame} type is available through the \texttt{DataFrames.jl}
package. There are several convenient features of a \texttt{DataFrame},
including:

\begin{itemize}
\tightlist
\item
  columns can be different Julia types;
\item
  table cell entries can be missing;
\item
  metadata can be associated with a \texttt{DataFrame};
\item
  columns can be names;
\item
  tables can be subsetted by row, column or both.
\end{itemize}

The columns of a \texttt{DataFrame} are most ogten integers, floats or
strings, an dthey are specified by Julia symbols.

    \begin{Verbatim}[commandchars=\\\{\}]
{\color{incolor}In [{\color{incolor}1}]:} \PY{c}{\PYZsh{}\PYZsh{} symbol versus string}
        \PY{n}{fruit} \PY{o}{=} \PY{l+s}{\PYZdq{}}\PY{l+s}{a}\PY{l+s}{p}\PY{l+s}{p}\PY{l+s}{l}\PY{l+s}{e}\PY{l+s}{\PYZdq{}}
        \PY{n}{println}\PY{p}{(}\PY{l+s}{\PYZdq{}}\PY{l+s}{e}\PY{l+s}{v}\PY{l+s}{a}\PY{l+s}{l}\PY{l+s}{(}\PY{l+s}{:}\PY{l+s}{f}\PY{l+s}{r}\PY{l+s}{u}\PY{l+s}{i}\PY{l+s}{t}\PY{l+s}{)}\PY{l+s}{:}\PY{l+s}{ }\PY{l+s}{\PYZdq{}}\PY{p}{,} \PY{n}{eval}\PY{p}{(}\PY{o}{:}\PY{n}{fruit}\PY{p}{)}\PY{p}{)}
\end{Verbatim}

    \begin{Verbatim}[commandchars=\\\{\}]
eval(:fruit): apple

    \end{Verbatim}

    \begin{Verbatim}[commandchars=\\\{\}]
{\color{incolor}In [{\color{incolor}2}]:} \PY{n}{println}\PY{p}{(}\PY{l+s}{\PYZdq{}\PYZdq{}\PYZdq{}}\PY{l+s}{e}\PY{l+s}{v}\PY{l+s}{a}\PY{l+s}{l}\PY{l+s}{(}\PY{l+s}{\PYZdq{}}\PY{l+s}{a}\PY{l+s}{p}\PY{l+s}{p}\PY{l+s}{l}\PY{l+s}{e}\PY{l+s}{\PYZdq{}}\PY{l+s}{)}\PY{l+s}{:}\PY{l+s}{ }\PY{l+s}{\PYZdq{}\PYZdq{}\PYZdq{}}\PY{p}{,} \PY{n}{eval}\PY{p}{(}\PY{l+s}{\PYZdq{}}\PY{l+s}{a}\PY{l+s}{p}\PY{l+s}{p}\PY{l+s}{l}\PY{l+s}{e}\PY{l+s}{\PYZdq{}}\PY{p}{)}\PY{p}{)}
\end{Verbatim}

    \begin{Verbatim}[commandchars=\\\{\}]
eval("apple"): apple

    \end{Verbatim}

    In Julia, a symbol is how a vaariable name is represented as data; on
the other hand, a string represents itself. Note that
\texttt{df{[}:symbol{]}} is how a column is accessed with a symbol;
specifically, the data in the column represented by \texttt{symbol}
contained in the \texttt{DataFrame\ df} is being accessed. In Julia, a
\texttt{DataFrame} can be built all once in multiple phases.

    \begin{Verbatim}[commandchars=\\\{\}]
{\color{incolor}In [{\color{incolor}6}]:} \PY{c}{\PYZsh{}\PYZsh{} Some examples with DataFrames}
        
        \PY{k}{using} \PY{n}{DataFrames}\PY{p}{,} \PY{n}{Distributions}\PY{p}{,} \PY{n}{StatsBase}\PY{p}{,} \PY{n}{Random}
        
        \PY{n}{Random}\PY{o}{.}\PY{n}{seed!}\PY{p}{(}\PY{l+m+mi}{825}\PY{p}{)}
        
        \PY{n}{N}\PY{o}{=}\PY{l+m+mi}{50}
        
        \PY{c}{\PYZsh{}\PYZsh{} Create a sample dataframe}
        \PY{c}{\PYZsh{}\PYZsh{} Initially the DataFrame has N rows and 3 collumns}
        \PY{n}{df1} \PY{o}{=} \PY{n}{DataFrame}\PY{p}{(}
            \PY{n}{x1} \PY{o}{=} \PY{n}{rand}\PY{p}{(}\PY{n}{Normal}\PY{p}{(}\PY{l+m+mi}{2}\PY{p}{,}\PY{l+m+mi}{1}\PY{p}{)}\PY{p}{,}\PY{n}{N}\PY{p}{)}\PY{p}{,}
            \PY{n}{x2} \PY{o}{=} \PY{p}{[}\PY{n}{sample}\PY{p}{(}\PY{p}{[}\PY{l+s}{\PYZdq{}}\PY{l+s}{H}\PY{l+s}{i}\PY{l+s}{g}\PY{l+s}{h}\PY{l+s}{\PYZdq{}}\PY{p}{,}\PY{l+s}{\PYZdq{}}\PY{l+s}{M}\PY{l+s}{e}\PY{l+s}{d}\PY{l+s}{i}\PY{l+s}{u}\PY{l+s}{m}\PY{l+s}{\PYZdq{}}\PY{p}{,}\PY{l+s}{\PYZdq{}}\PY{l+s}{L}\PY{l+s}{o}\PY{l+s}{w}\PY{l+s}{\PYZdq{}}\PY{p}{]}\PY{p}{,}
                        \PY{n}{pweights}\PY{p}{(}\PY{p}{[}\PY{l+m+mf}{0.25}\PY{p}{,}\PY{l+m+mf}{0.45}\PY{p}{,}\PY{l+m+mf}{0.30}\PY{p}{]}\PY{p}{)}\PY{p}{)} \PY{k}{for} \PY{n}{i} \PY{o}{=}\PY{l+m+mi}{1}\PY{o}{:}\PY{n}{N}\PY{p}{]}\PY{p}{,}
            \PY{n}{x3} \PY{o}{=} \PY{n}{rand}\PY{p}{(}\PY{n}{Pareto}\PY{p}{(}\PY{l+m+mi}{2}\PY{p}{,}\PY{l+m+mi}{1}\PY{p}{)}\PY{p}{,}\PY{n}{N}\PY{p}{)}
        \PY{p}{)}
        
        \PY{c}{\PYZsh{}\PYZsh{} Add a 4th column, y, which is dependent on x3 and the level of x2}
        \PY{n}{df1}\PY{p}{[}\PY{o}{:}\PY{n}{y}\PY{p}{]} \PY{o}{=} \PY{p}{[}\PY{n}{df1}\PY{p}{[}\PY{n}{i}\PY{p}{,}\PY{o}{:}\PY{n}{x2}\PY{p}{]} \PY{o}{==} \PY{l+s}{\PYZdq{}}\PY{l+s}{M}\PY{l+s}{e}\PY{l+s}{d}\PY{l+s}{i}\PY{l+s}{u}\PY{l+s}{m}\PY{l+s}{\PYZdq{}} \PY{o}{?} \PY{o}{*}\PY{p}{(}\PY{l+m+mi}{2}\PY{p}{,} \PY{n}{df1}\PY{p}{[}\PY{n}{i}\PY{p}{,} \PY{o}{:}\PY{n}{x3}\PY{p}{]}\PY{p}{)} \PY{o}{:}
                        \PY{n}{df1}\PY{p}{[}\PY{n}{i}\PY{p}{,}\PY{o}{:}\PY{n}{x2}\PY{p}{]} \PY{o}{==} \PY{l+s}{\PYZdq{}}\PY{l+s}{H}\PY{l+s}{i}\PY{l+s}{g}\PY{l+s}{h}\PY{l+s}{\PYZdq{}} \PY{o}{?} \PY{o}{*}\PY{p}{(}\PY{l+m+mi}{4}\PY{p}{,} \PY{n}{df1}\PY{p}{[}\PY{n}{i}\PY{p}{,} \PY{o}{:}\PY{n}{x3}\PY{p}{]}\PY{p}{)} \PY{o}{:}
                            \PY{o}{*}\PY{p}{(}\PY{l+m+mf}{0.5}\PY{p}{,} \PY{n}{df1}\PY{p}{[}\PY{n}{i}\PY{p}{,}\PY{o}{:}\PY{n}{x3}\PY{p}{]}\PY{p}{)} \PY{k}{for} \PY{n}{i}\PY{o}{=}\PY{l+m+mi}{1}\PY{o}{:}\PY{n}{N}\PY{p}{]}
\end{Verbatim}

\begin{Verbatim}[commandchars=\\\{\}]
{\color{outcolor}Out[{\color{outcolor}6}]:} 50-element Array\{Float64,1\}:
         2.393325358524223 
         2.4310412249328515
         5.0422956234853125
         3.7345079211580465
         0.8588031634040387
         0.5231696808320803
         2.031479078828272 
         2.4806651435260076
         9.576090582790384 
         6.124545507841852 
         0.7720685108626731
         6.192654315744594 
         6.699890875731672 
         ⋮                 
         3.691184458092902 
         9.562195756696816 
         2.433614543200589 
         2.412383239462625 
         1.222717094180467 
         3.6746066646508866
         2.5425484862899155
         0.6933323954399253
         7.280430172887931 
         3.2074781385911297
         4.770360925053668 
         2.39414636559205  
\end{Verbatim}
            
    A \texttt{DataFrame} can be sliced the same a two-dimensional
\texttt{Array} is sliced, i.e.,
\texttt{df{[}row\_range,\ column\_range{]}}. These ranges can be
specified in a number of ways:

\begin{itemize}
\tightlist
\item
  Using \texttt{Int} indices individually or as arrays, e.g., \texttt{1}
  or \texttt{{[}4,6,9{]}}.
\item
  Using \texttt{:} to select indics in a dimension, e.g., \texttt{x:y}
  selects the range from \texttt{x} to \texttt{y} and \texttt{:} selects
  all indices in that dimension.
\item
  Via arrays of Boolean values, where \texttt{true} selects the elements
  at that index.
\end{itemize}

Note that columns can be selected by their symbols, either individually
or in an array \texttt{{[}:x1,\ :x2{]}}.

    \begin{Verbatim}[commandchars=\\\{\}]
{\color{incolor}In [{\color{incolor}7}]:} \PY{c}{\PYZsh{}\PYZsh{} Slicing DataFrames}
        \PY{n}{println}\PY{p}{(}\PY{l+s}{\PYZdq{}}\PY{l+s}{d}\PY{l+s}{f}\PY{l+s}{1}\PY{l+s}{[}\PY{l+s}{1}\PY{l+s}{:}\PY{l+s}{2}\PY{l+s}{,}\PY{l+s}{3}\PY{l+s}{:}\PY{l+s}{4}\PY{l+s}{]}\PY{l+s}{:}\PY{l+s}{ }\PY{l+s}{\PYZdq{}}\PY{p}{,}\PY{n}{df1}\PY{p}{[}\PY{l+m+mi}{1}\PY{o}{:}\PY{l+m+mi}{2}\PY{p}{,} \PY{l+m+mi}{3}\PY{o}{:}\PY{l+m+mi}{4}\PY{p}{]}\PY{p}{)}
\end{Verbatim}

    \begin{Verbatim}[commandchars=\\\{\}]
df1[1:2,3:4]: 2×2 DataFrame
│ Row │ x3      │ y       │
│     │ \textcolor{ansi-black-intense}{Float64} │ \textcolor{ansi-black-intense}{Float64} │
├─────┼─────────┼─────────┤
│ 1   │ 1.19666 │ 2.39333 │
│ 2   │ 1.21552 │ 2.43104 │

    \end{Verbatim}

    \begin{Verbatim}[commandchars=\\\{\}]
{\color{incolor}In [{\color{incolor}8}]:} \PY{n}{println}\PY{p}{(}\PY{l+s}{\PYZdq{}}\PY{l+s+se}{\PYZbs{}n}\PY{l+s}{d}\PY{l+s}{f}\PY{l+s}{1}\PY{l+s}{[}\PY{l+s}{1}\PY{l+s}{:}\PY{l+s}{2}\PY{l+s}{,}\PY{l+s}{ }\PY{l+s}{[}\PY{l+s}{:}\PY{l+s}{y}\PY{l+s}{,}\PY{l+s}{:}\PY{l+s}{x}\PY{l+s}{1}\PY{l+s}{]}\PY{l+s}{]}\PY{l+s}{:}\PY{l+s}{ }\PY{l+s}{\PYZdq{}}\PY{p}{,}\PY{n}{df1}\PY{p}{[}\PY{l+m+mi}{1}\PY{o}{:}\PY{l+m+mi}{2}\PY{p}{,} \PY{p}{[}\PY{o}{:}\PY{n}{y}\PY{p}{,}\PY{o}{:}\PY{n}{x1}\PY{p}{]}\PY{p}{]}\PY{p}{)}
\end{Verbatim}

    \begin{Verbatim}[commandchars=\\\{\}]

df1[1:2, [:y,:x1]]: 2×2 DataFrame
│ Row │ y       │ x1      │
│     │ \textcolor{ansi-black-intense}{Float64} │ \textcolor{ansi-black-intense}{Float64} │
├─────┼─────────┼─────────┤
│ 1   │ 2.39333 │ 1.81025 │
│ 2   │ 2.43104 │ 3.1707  │

    \end{Verbatim}

    \begin{Verbatim}[commandchars=\\\{\}]
{\color{incolor}In [{\color{incolor}9}]:} \PY{c}{\PYZsh{}\PYZsh{} Now, exclude columns x1 and x2}
        \PY{n}{keep} \PY{o}{=} \PY{n}{setdiff}\PY{p}{(}\PY{n}{names}\PY{p}{(}\PY{n}{df1}\PY{p}{)}\PY{p}{,} \PY{p}{[}\PY{o}{:}\PY{n}{x1}\PY{p}{,} \PY{o}{:}\PY{n}{x2}\PY{p}{]}\PY{p}{)}
        \PY{n}{println}\PY{p}{(}\PY{l+s}{\PYZdq{}}\PY{l+s+se}{\PYZbs{}n}\PY{l+s}{C}\PY{l+s}{o}\PY{l+s}{l}\PY{l+s}{u}\PY{l+s}{m}\PY{l+s}{s}\PY{l+s}{ }\PY{l+s}{t}\PY{l+s}{o}\PY{l+s}{ }\PY{l+s}{k}\PY{l+s}{e}\PY{l+s}{e}\PY{l+s}{p}\PY{l+s}{\PYZdq{}}\PY{p}{,} \PY{n}{keep}\PY{p}{)}
\end{Verbatim}

    \begin{Verbatim}[commandchars=\\\{\}]

Colums to keepSymbol[:x3, :y]

    \end{Verbatim}

    \begin{Verbatim}[commandchars=\\\{\}]
{\color{incolor}In [{\color{incolor}10}]:} \PY{n}{println}\PY{p}{(}\PY{l+s}{\PYZdq{}}\PY{l+s}{d}\PY{l+s}{f}\PY{l+s}{1}\PY{l+s}{[}\PY{l+s}{1}\PY{l+s}{:}\PY{l+s}{2}\PY{l+s}{,}\PY{l+s}{k}\PY{l+s}{e}\PY{l+s}{e}\PY{l+s}{p}\PY{l+s}{]}\PY{l+s}{:}\PY{l+s}{ }\PY{l+s}{\PYZdq{}}\PY{p}{,}\PY{n}{df1}\PY{p}{[}\PY{l+m+mi}{1}\PY{o}{:}\PY{l+m+mi}{2}\PY{p}{,} \PY{n}{keep}\PY{p}{]}\PY{p}{)}
\end{Verbatim}

    \begin{Verbatim}[commandchars=\\\{\}]
df1[1:2,keep]: 2×2 DataFrame
│ Row │ x3      │ y       │
│     │ \textcolor{ansi-black-intense}{Float64} │ \textcolor{ansi-black-intense}{Float64} │
├─────┼─────────┼─────────┤
│ 1   │ 1.19666 │ 2.39333 │
│ 2   │ 1.21552 │ 2.43104 │

    \end{Verbatim}

    In practical applications, missing data is common. In
\texttt{DataFrames.jl}, the \texttt{Missing} type is used to represent
missing values. In Julia, a singlton occurence of \texttt{Missing},
\texttt{missing} is used to represent missing data. Specifically,
\texttt{missing} is used to represent the value of a measurement when a
valid value could have been observed but was not. Note that
\texttt{missing} in Julia is analogous to \texttt{NA} in \emph{R}.

In the following code block, the array \texttt{v2} has type
\texttt{Union\{Float64,Missings.Missing\}}. In Julia, \texttt{Union}
types are an abstract type that contain objects of types included in its
arguments. In this example, \texttt{v2} can contain values of
\texttt{missing} or \texttt{Float64} numbers. Note that
\texttt{missings()} can be used togenerate arrays that will support
missing values; specifically, it will generate vectors of type
\texttt{Union} if another type is specified in the first argument of the
function call. Also, \texttt{ismissing(x)} is used to test whether
\texttt{x} is missing, where \texttt{x} is usually an element of a data
structure, e.g., \texttt{ismissing(v2{[}1{]})}.

    \begin{Verbatim}[commandchars=\\\{\}]
{\color{incolor}In [{\color{incolor}13}]:} \PY{n}{v1} \PY{o}{=} \PY{n}{missings}\PY{p}{(}\PY{l+m+mi}{2}\PY{p}{)}
         \PY{n}{println}\PY{p}{(}\PY{l+s}{\PYZdq{}}\PY{l+s}{v}\PY{l+s}{1}\PY{l+s}{:}\PY{l+s}{ }\PY{l+s}{\PYZdq{}}\PY{p}{,}\PY{n}{v1}\PY{p}{)}
\end{Verbatim}

    \begin{Verbatim}[commandchars=\\\{\}]
v1: Missing[missing, missing]

    \end{Verbatim}

    \begin{Verbatim}[commandchars=\\\{\}]
{\color{incolor}In [{\color{incolor}14}]:} \PY{n}{v2} \PY{o}{=} \PY{n}{missings}\PY{p}{(}\PY{k+kt}{Float64}\PY{p}{,}\PY{l+m+mi}{1}\PY{p}{,}\PY{l+m+mi}{3}\PY{p}{)}
         \PY{n}{v2}\PY{p}{[}\PY{l+m+mi}{2}\PY{p}{]} \PY{o}{=} \PY{n+nb}{pi}
         \PY{n}{println}\PY{p}{(}\PY{l+s}{\PYZdq{}}\PY{l+s}{v}\PY{l+s}{2}\PY{l+s}{:}\PY{l+s}{ }\PY{l+s+si}{\PYZdl{}}\PY{p}{(}\PY{n}{v2}\PY{p}{)}\PY{l+s}{\PYZdq{}}\PY{p}{)}
\end{Verbatim}

    \begin{Verbatim}[commandchars=\\\{\}]
v2: Union\{Missing, Float64\}[missing 3.14159 missing]

    \end{Verbatim}

    \begin{Verbatim}[commandchars=\\\{\}]
{\color{incolor}In [{\color{incolor}15}]:} \PY{c}{\PYZsh{}\PYZsh{} test for missing}
         \PY{n}{m1} \PY{o}{=} \PY{n}{map}\PY{p}{(}\PY{n}{ismissing}\PY{p}{,}\PY{n}{v2}\PY{p}{)}
         \PY{n}{println}\PY{p}{(}\PY{l+s}{\PYZdq{}}\PY{l+s}{m}\PY{l+s}{1}\PY{l+s}{:}\PY{l+s}{ }\PY{l+s+si}{\PYZdl{}}\PY{p}{(}\PY{n}{m1}\PY{p}{)}\PY{l+s}{\PYZdq{}}\PY{p}{)}
\end{Verbatim}

    \begin{Verbatim}[commandchars=\\\{\}]
m1: Bool[true false true]

    \end{Verbatim}

    \begin{Verbatim}[commandchars=\\\{\}]
{\color{incolor}In [{\color{incolor}16}]:} \PY{n}{println}\PY{p}{(}\PY{l+s}{\PYZdq{}}\PY{l+s}{P}\PY{l+s}{e}\PY{l+s}{r}\PY{l+s}{c}\PY{l+s}{e}\PY{l+s}{n}\PY{l+s}{t}\PY{l+s}{ }\PY{l+s}{m}\PY{l+s}{i}\PY{l+s}{s}\PY{l+s}{s}\PY{l+s}{i}\PY{l+s}{n}\PY{l+s}{g}\PY{l+s}{ }\PY{l+s}{v}\PY{l+s}{2}\PY{l+s}{:}\PY{l+s}{ }\PY{l+s}{\PYZdq{}}\PY{p}{,} \PY{o}{*}\PY{p}{(}\PY{n}{mean}\PY{p}{(}\PY{p}{[}\PY{n}{ismissing}\PY{p}{(}\PY{n}{i}\PY{p}{)} \PY{k}{for} \PY{n}{i} \PY{k+kp}{in} \PY{n}{v2}\PY{p}{]}\PY{p}{)}\PY{p}{,} \PY{l+m+mi}{100}\PY{p}{)}\PY{p}{)}
\end{Verbatim}

    \begin{Verbatim}[commandchars=\\\{\}]
Percent missing v2: 66.66666666666666

    \end{Verbatim}

    Note that most functions in Julia do not accept data of type
\texttt{Missings.Missing} as input. Therefore, users are often required
to remove them before they can use specific functions. Using
\texttt{skipmissing()} returns an iterator that excludes the missing
values and, when used in conjunction with \texttt{collect()}, gives an
array of non-missing values. This approach can be used with functions
that take non-missing values only.

    \begin{Verbatim}[commandchars=\\\{\}]
{\color{incolor}In [{\color{incolor}17}]:} \PY{c}{\PYZsh{}\PYZsh{} calculates the mean of the non\PYZhy{}missing values}
         \PY{n}{mean}\PY{p}{(}\PY{n}{skipmissing}\PY{p}{(}\PY{n}{v2}\PY{p}{)}\PY{p}{)}
\end{Verbatim}

\begin{Verbatim}[commandchars=\\\{\}]
{\color{outcolor}Out[{\color{outcolor}17}]:} 3.141592653589793
\end{Verbatim}
            
    \begin{Verbatim}[commandchars=\\\{\}]
{\color{incolor}In [{\color{incolor}18}]:} \PY{c}{\PYZsh{}\PYZsh{} collects the non\PYZhy{}missing values in an array}
         \PY{n}{collect}\PY{p}{(}\PY{n}{skipmissing}\PY{p}{(}\PY{n}{v2}\PY{p}{)}\PY{p}{)}
\end{Verbatim}

\begin{Verbatim}[commandchars=\\\{\}]
{\color{outcolor}Out[{\color{outcolor}18}]:} 1-element Array\{Float64,1\}:
          3.141592653589793
\end{Verbatim}
            
    \subsection{Categorical Data}\label{categorical-data}

In Julia, categorical data is represented by arrays of type
\texttt{CategoricalArray}, defined in the \texttt{CategoricalArrays.jl}
package. Note that \texttt{CategoricalArray} arrays are analogous to
factors in \emph{R}. \texttt{CategoricalArray} arrays have a number of
advantages over \texttt{String} arrays in a dataframe:

\begin{itemize}
\tightlist
\item
  They save memory be representing each unique value of the string array
  as an index.
\item
  Each index corresponds to a level.
\item
  After data cleaning, there are usually only a small number of levels.
\end{itemize}

\texttt{CcategoricalArray} arrays support missing values. The type
\texttt{CategoricalArray\{Union\{T,\ Missing\}\}} is used to represent
missing values. When indexing/slicing arrays of this type.
\texttt{missinng} is reutrned when it is present at that location.

    \begin{Verbatim}[commandchars=\\\{\}]
{\color{incolor}In [{\color{incolor}22}]:} \PY{c}{\PYZsh{}\PYZsh{} Number of entries for the categorical Arrays}
         \PY{n}{Nca} \PY{o}{=}\PY{l+m+mi}{10}
         
         \PY{c}{\PYZsh{}\PYZsh{} Empty array}
         \PY{n}{v3} \PY{o}{=} \PY{k+kt}{Array}\PY{p}{\PYZob{}}\PY{k+kt}{Union}\PY{p}{\PYZob{}}\PY{n}{String}\PY{p}{,}\PY{n}{Missing}\PY{p}{\PYZcb{}}\PY{p}{\PYZcb{}}\PY{p}{(}\PY{n}{undef}\PY{p}{,} \PY{n}{Nca}\PY{p}{)}
         
         \PY{c}{\PYZsh{}\PYZsh{}Array has string and missing values}
         \PY{n}{v3} \PY{o}{=} \PY{p}{[}\PY{n}{isodd}\PY{p}{(}\PY{n}{i}\PY{p}{)} \PY{o}{?} \PY{n}{sample}\PY{p}{(}\PY{p}{[}\PY{l+s}{\PYZdq{}}\PY{l+s}{H}\PY{l+s}{i}\PY{l+s}{g}\PY{l+s}{h}\PY{l+s}{\PYZdq{}}\PY{p}{,} \PY{l+s}{\PYZdq{}}\PY{l+s}{L}\PY{l+s}{o}\PY{l+s}{w}\PY{l+s}{\PYZdq{}}\PY{p}{]}\PY{p}{,} \PY{n}{pweights}\PY{p}{(}\PY{p}{[}\PY{l+m+mf}{0.35}\PY{p}{,}\PY{l+m+mf}{0.65}\PY{p}{]}\PY{p}{)}\PY{p}{)} \PY{o}{:} 
                 \PY{n}{missing} \PY{k}{for} \PY{n}{i} \PY{o}{=} \PY{l+m+mi}{1}\PY{o}{:}\PY{n}{Nca}\PY{p}{]}
         \PY{c}{\PYZsh{}\PYZsh{} v3c is of type CategoricalArray\PYZob{}Union\PYZob{}Missing,String\PYZcb{},1,UInt32\PYZcb{}}
         \PY{n}{v3c} \PY{o}{=} \PY{n}{categorical}\PY{p}{(}\PY{n}{v3}\PY{p}{)}
         
         \PY{c}{\PYZsh{}\PYZsh{} Levels should be [\PYZdq{}High\PYZdq{},\PYZdq{}Low\PYZdq{}]}
         \PY{n}{println}\PY{p}{(}\PY{l+s}{\PYZdq{}}\PY{l+s}{1}\PY{l+s}{.}\PY{l+s}{ }\PY{l+s}{l}\PY{l+s}{e}\PY{l+s}{v}\PY{l+s}{e}\PY{l+s}{l}\PY{l+s}{s}\PY{l+s}{(}\PY{l+s}{v}\PY{l+s}{3}\PY{l+s}{c}\PY{l+s}{)}\PY{l+s}{:}\PY{l+s}{ }\PY{l+s}{\PYZdq{}}\PY{p}{,} \PY{n}{levels}\PY{p}{(}\PY{n}{v3c}\PY{p}{)}\PY{p}{)}
         \PY{n}{println}\PY{p}{(}\PY{l+s}{\PYZdq{}}\PY{l+s}{1}\PY{l+s}{.}\PY{l+s}{ }\PY{l+s}{v}\PY{l+s}{3}\PY{l+s}{c}\PY{l+s}{:}\PY{l+s}{ }\PY{l+s+si}{\PYZdl{}}\PY{p}{(}\PY{n}{v3c}\PY{p}{)}\PY{l+s}{\PYZdq{}}\PY{p}{)}
\end{Verbatim}

    \begin{Verbatim}[commandchars=\\\{\}]
1. levels(v3c): ["High", "Low"]
1. v3c: Union\{Missing, CategoricalString\{UInt32\}\}["Low", missing, "High", missing, "Low", missing, "High", missing, "Low", missing]

    \end{Verbatim}

    \begin{Verbatim}[commandchars=\\\{\}]
{\color{incolor}In [{\color{incolor}20}]:} \PY{c}{\PYZsh{}\PYZsh{} Reordered levels \PYZhy{} does not change the data}
         \PY{n}{levels!}\PY{p}{(}\PY{n}{v3c}\PY{p}{,} \PY{p}{[}\PY{l+s}{\PYZdq{}}\PY{l+s}{L}\PY{l+s}{o}\PY{l+s}{w}\PY{l+s}{\PYZdq{}}\PY{p}{,}\PY{l+s}{\PYZdq{}}\PY{l+s}{H}\PY{l+s}{i}\PY{l+s}{g}\PY{l+s}{h}\PY{l+s}{\PYZdq{}}\PY{p}{]}\PY{p}{)}
         \PY{n}{println}\PY{p}{(}\PY{l+s}{\PYZdq{}}\PY{l+s}{2}\PY{l+s}{.}\PY{l+s}{l}\PY{l+s}{e}\PY{l+s}{v}\PY{l+s}{e}\PY{l+s}{l}\PY{l+s}{s}\PY{l+s}{(}\PY{l+s}{v}\PY{l+s}{3}\PY{l+s}{c}\PY{l+s}{)}\PY{l+s}{:}\PY{l+s}{ }\PY{l+s+si}{\PYZdl{}}\PY{p}{(}\PY{n}{levels}\PY{p}{(}\PY{n}{v3c}\PY{p}{)}\PY{p}{)}\PY{l+s}{\PYZdq{}}\PY{p}{)}
\end{Verbatim}

    \begin{Verbatim}[commandchars=\\\{\}]
2.levels(v3c): ["Low", "High"]

    \end{Verbatim}

    \begin{Verbatim}[commandchars=\\\{\}]
{\color{incolor}In [{\color{incolor}21}]:} \PY{n}{println}\PY{p}{(}\PY{l+s}{\PYZdq{}}\PY{l+s}{2}\PY{l+s}{.}\PY{l+s}{v}\PY{l+s}{3}\PY{l+s}{c}\PY{l+s}{:}\PY{l+s}{ }\PY{l+s+si}{\PYZdl{}}\PY{p}{(}\PY{n}{v3c}\PY{p}{)}\PY{l+s}{\PYZdq{}}\PY{p}{)}
\end{Verbatim}

    \begin{Verbatim}[commandchars=\\\{\}]
2.v3c: Union\{Missing, CategoricalString\{UInt32\}\}["High", missing, "Low", missing, "Low", missing, "High", missing, "Low", missing]

    \end{Verbatim}

    Here are several useful functinos that can be used with
\texttt{CatoegoricalArray} arrays:

\begin{itemize}
\tightlist
\item
  \texttt{levels()} returns the levels pf the \texttt{CategoricalArray}.
\item
  \texttt{levels()} changes the order of the array's levels.
\item
  \texttt{compress()} compresses the array saving memory.
\item
  \texttt{decompress()} decompresses the compressed array.
\item
  \texttt{categorical(ca)} converts the array \texttt{ca} into an array
  of type \texttt{CategoricalArray}.
\item
  \texttt{droplevels!(ca)} drops levels no longer present in the array
  \texttt{ca}. This is useful when a dataframe has been subsetted and
  some levels are no longer present in the data.
\item
  \texttt{recode(a,\ pairs)} recodes the levels of the array. New levels
  should be the same type as the original ones.
\item
  \texttt{recode!(new,\ orig,\ pairs)} recodes the levels in
  \texttt{orig} using the pairs and puts the \texttt{new} levels in
  \texttt{new}.
\end{itemize}


    % Add a bibliography block to the postdoc
    
    
    
    \end{document}
