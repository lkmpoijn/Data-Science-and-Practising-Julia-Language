
% Default to the notebook output style

    


% Inherit from the specified cell style.




    
\documentclass[11pt]{article}

    
    
    \usepackage[T1]{fontenc}
    % Nicer default font (+ math font) than Computer Modern for most use cases
    \usepackage{mathpazo}

    % Basic figure setup, for now with no caption control since it's done
    % automatically by Pandoc (which extracts ![](path) syntax from Markdown).
    \usepackage{graphicx}
    % We will generate all images so they have a width \maxwidth. This means
    % that they will get their normal width if they fit onto the page, but
    % are scaled down if they would overflow the margins.
    \makeatletter
    \def\maxwidth{\ifdim\Gin@nat@width>\linewidth\linewidth
    \else\Gin@nat@width\fi}
    \makeatother
    \let\Oldincludegraphics\includegraphics
    % Set max figure width to be 80% of text width, for now hardcoded.
    \renewcommand{\includegraphics}[1]{\Oldincludegraphics[width=.8\maxwidth]{#1}}
    % Ensure that by default, figures have no caption (until we provide a
    % proper Figure object with a Caption API and a way to capture that
    % in the conversion process - todo).
    \usepackage{caption}
    \DeclareCaptionLabelFormat{nolabel}{}
    \captionsetup{labelformat=nolabel}

    \usepackage{adjustbox} % Used to constrain images to a maximum size 
    \usepackage{xcolor} % Allow colors to be defined
    \usepackage{enumerate} % Needed for markdown enumerations to work
    \usepackage{geometry} % Used to adjust the document margins
    \usepackage{amsmath} % Equations
    \usepackage{amssymb} % Equations
    \usepackage{textcomp} % defines textquotesingle
    % Hack from http://tex.stackexchange.com/a/47451/13684:
    \AtBeginDocument{%
        \def\PYZsq{\textquotesingle}% Upright quotes in Pygmentized code
    }
    \usepackage{upquote} % Upright quotes for verbatim code
    \usepackage{eurosym} % defines \euro
    \usepackage[mathletters]{ucs} % Extended unicode (utf-8) support
    %\usepackage[utf8x]{inputenc} % Allow utf-8 characters in the tex document
    \usepackage{fancyvrb} % verbatim replacement that allows latex
    \usepackage{grffile} % extends the file name processing of package graphics 
                         % to support a larger range 
    % The hyperref package gives us a pdf with properly built
    % internal navigation ('pdf bookmarks' for the table of contents,
    % internal cross-reference links, web links for URLs, etc.)
    \usepackage{hyperref}
    \usepackage{longtable} % longtable support required by pandoc >1.10
    \usepackage{booktabs}  % table support for pandoc > 1.12.2
    \usepackage[inline]{enumitem} % IRkernel/repr support (it uses the enumerate* environment)
    \usepackage[normalem]{ulem} % ulem is needed to support strikethroughs (\sout)
                                % normalem makes italics be italics, not underlines
    \usepackage{mathrsfs}
    

    
    
    % Colors for the hyperref package
    \definecolor{urlcolor}{rgb}{0,.145,.698}
    \definecolor{linkcolor}{rgb}{.71,0.21,0.01}
    \definecolor{citecolor}{rgb}{.12,.54,.11}

    % ANSI colors
    \definecolor{ansi-black}{HTML}{3E424D}
    \definecolor{ansi-black-intense}{HTML}{282C36}
    \definecolor{ansi-red}{HTML}{E75C58}
    \definecolor{ansi-red-intense}{HTML}{B22B31}
    \definecolor{ansi-green}{HTML}{00A250}
    \definecolor{ansi-green-intense}{HTML}{007427}
    \definecolor{ansi-yellow}{HTML}{DDB62B}
    \definecolor{ansi-yellow-intense}{HTML}{B27D12}
    \definecolor{ansi-blue}{HTML}{208FFB}
    \definecolor{ansi-blue-intense}{HTML}{0065CA}
    \definecolor{ansi-magenta}{HTML}{D160C4}
    \definecolor{ansi-magenta-intense}{HTML}{A03196}
    \definecolor{ansi-cyan}{HTML}{60C6C8}
    \definecolor{ansi-cyan-intense}{HTML}{258F8F}
    \definecolor{ansi-white}{HTML}{C5C1B4}
    \definecolor{ansi-white-intense}{HTML}{A1A6B2}
    \definecolor{ansi-default-inverse-fg}{HTML}{FFFFFF}
    \definecolor{ansi-default-inverse-bg}{HTML}{000000}

    % commands and environments needed by pandoc snippets
    % extracted from the output of `pandoc -s`
    \providecommand{\tightlist}{%
      \setlength{\itemsep}{0pt}\setlength{\parskip}{0pt}}
    \DefineVerbatimEnvironment{Highlighting}{Verbatim}{commandchars=\\\{\}}
    % Add ',fontsize=\small' for more characters per line
    \newenvironment{Shaded}{}{}
    \newcommand{\KeywordTok}[1]{\textcolor[rgb]{0.00,0.44,0.13}{\textbf{{#1}}}}
    \newcommand{\DataTypeTok}[1]{\textcolor[rgb]{0.56,0.13,0.00}{{#1}}}
    \newcommand{\DecValTok}[1]{\textcolor[rgb]{0.25,0.63,0.44}{{#1}}}
    \newcommand{\BaseNTok}[1]{\textcolor[rgb]{0.25,0.63,0.44}{{#1}}}
    \newcommand{\FloatTok}[1]{\textcolor[rgb]{0.25,0.63,0.44}{{#1}}}
    \newcommand{\CharTok}[1]{\textcolor[rgb]{0.25,0.44,0.63}{{#1}}}
    \newcommand{\StringTok}[1]{\textcolor[rgb]{0.25,0.44,0.63}{{#1}}}
    \newcommand{\CommentTok}[1]{\textcolor[rgb]{0.38,0.63,0.69}{\textit{{#1}}}}
    \newcommand{\OtherTok}[1]{\textcolor[rgb]{0.00,0.44,0.13}{{#1}}}
    \newcommand{\AlertTok}[1]{\textcolor[rgb]{1.00,0.00,0.00}{\textbf{{#1}}}}
    \newcommand{\FunctionTok}[1]{\textcolor[rgb]{0.02,0.16,0.49}{{#1}}}
    \newcommand{\RegionMarkerTok}[1]{{#1}}
    \newcommand{\ErrorTok}[1]{\textcolor[rgb]{1.00,0.00,0.00}{\textbf{{#1}}}}
    \newcommand{\NormalTok}[1]{{#1}}
    
    % Additional commands for more recent versions of Pandoc
    \newcommand{\ConstantTok}[1]{\textcolor[rgb]{0.53,0.00,0.00}{{#1}}}
    \newcommand{\SpecialCharTok}[1]{\textcolor[rgb]{0.25,0.44,0.63}{{#1}}}
    \newcommand{\VerbatimStringTok}[1]{\textcolor[rgb]{0.25,0.44,0.63}{{#1}}}
    \newcommand{\SpecialStringTok}[1]{\textcolor[rgb]{0.73,0.40,0.53}{{#1}}}
    \newcommand{\ImportTok}[1]{{#1}}
    \newcommand{\DocumentationTok}[1]{\textcolor[rgb]{0.73,0.13,0.13}{\textit{{#1}}}}
    \newcommand{\AnnotationTok}[1]{\textcolor[rgb]{0.38,0.63,0.69}{\textbf{\textit{{#1}}}}}
    \newcommand{\CommentVarTok}[1]{\textcolor[rgb]{0.38,0.63,0.69}{\textbf{\textit{{#1}}}}}
    \newcommand{\VariableTok}[1]{\textcolor[rgb]{0.10,0.09,0.49}{{#1}}}
    \newcommand{\ControlFlowTok}[1]{\textcolor[rgb]{0.00,0.44,0.13}{\textbf{{#1}}}}
    \newcommand{\OperatorTok}[1]{\textcolor[rgb]{0.40,0.40,0.40}{{#1}}}
    \newcommand{\BuiltInTok}[1]{{#1}}
    \newcommand{\ExtensionTok}[1]{{#1}}
    \newcommand{\PreprocessorTok}[1]{\textcolor[rgb]{0.74,0.48,0.00}{{#1}}}
    \newcommand{\AttributeTok}[1]{\textcolor[rgb]{0.49,0.56,0.16}{{#1}}}
    \newcommand{\InformationTok}[1]{\textcolor[rgb]{0.38,0.63,0.69}{\textbf{\textit{{#1}}}}}
    \newcommand{\WarningTok}[1]{\textcolor[rgb]{0.38,0.63,0.69}{\textbf{\textit{{#1}}}}}
    
    
    % Define a nice break command that doesn't care if a line doesn't already
    % exist.
    \def\br{\hspace*{\fill} \\* }
    % Math Jax compatibility definitions
    \def\gt{>}
    \def\lt{<}
    \let\Oldtex\TeX
    \let\Oldlatex\LaTeX
    \renewcommand{\TeX}{\textrm{\Oldtex}}
    \renewcommand{\LaTeX}{\textrm{\Oldlatex}}
    % Document parameters
    % Document title
    \title{juliastart}
    
    
    
    
    

    % Pygments definitions
    
\makeatletter
\def\PY@reset{\let\PY@it=\relax \let\PY@bf=\relax%
    \let\PY@ul=\relax \let\PY@tc=\relax%
    \let\PY@bc=\relax \let\PY@ff=\relax}
\def\PY@tok#1{\csname PY@tok@#1\endcsname}
\def\PY@toks#1+{\ifx\relax#1\empty\else%
    \PY@tok{#1}\expandafter\PY@toks\fi}
\def\PY@do#1{\PY@bc{\PY@tc{\PY@ul{%
    \PY@it{\PY@bf{\PY@ff{#1}}}}}}}
\def\PY#1#2{\PY@reset\PY@toks#1+\relax+\PY@do{#2}}

\expandafter\def\csname PY@tok@w\endcsname{\def\PY@tc##1{\textcolor[rgb]{0.73,0.73,0.73}{##1}}}
\expandafter\def\csname PY@tok@c\endcsname{\let\PY@it=\textit\def\PY@tc##1{\textcolor[rgb]{0.25,0.50,0.50}{##1}}}
\expandafter\def\csname PY@tok@cp\endcsname{\def\PY@tc##1{\textcolor[rgb]{0.74,0.48,0.00}{##1}}}
\expandafter\def\csname PY@tok@k\endcsname{\let\PY@bf=\textbf\def\PY@tc##1{\textcolor[rgb]{0.00,0.50,0.00}{##1}}}
\expandafter\def\csname PY@tok@kp\endcsname{\def\PY@tc##1{\textcolor[rgb]{0.00,0.50,0.00}{##1}}}
\expandafter\def\csname PY@tok@kt\endcsname{\def\PY@tc##1{\textcolor[rgb]{0.69,0.00,0.25}{##1}}}
\expandafter\def\csname PY@tok@o\endcsname{\def\PY@tc##1{\textcolor[rgb]{0.40,0.40,0.40}{##1}}}
\expandafter\def\csname PY@tok@ow\endcsname{\let\PY@bf=\textbf\def\PY@tc##1{\textcolor[rgb]{0.67,0.13,1.00}{##1}}}
\expandafter\def\csname PY@tok@nb\endcsname{\def\PY@tc##1{\textcolor[rgb]{0.00,0.50,0.00}{##1}}}
\expandafter\def\csname PY@tok@nf\endcsname{\def\PY@tc##1{\textcolor[rgb]{0.00,0.00,1.00}{##1}}}
\expandafter\def\csname PY@tok@nc\endcsname{\let\PY@bf=\textbf\def\PY@tc##1{\textcolor[rgb]{0.00,0.00,1.00}{##1}}}
\expandafter\def\csname PY@tok@nn\endcsname{\let\PY@bf=\textbf\def\PY@tc##1{\textcolor[rgb]{0.00,0.00,1.00}{##1}}}
\expandafter\def\csname PY@tok@ne\endcsname{\let\PY@bf=\textbf\def\PY@tc##1{\textcolor[rgb]{0.82,0.25,0.23}{##1}}}
\expandafter\def\csname PY@tok@nv\endcsname{\def\PY@tc##1{\textcolor[rgb]{0.10,0.09,0.49}{##1}}}
\expandafter\def\csname PY@tok@no\endcsname{\def\PY@tc##1{\textcolor[rgb]{0.53,0.00,0.00}{##1}}}
\expandafter\def\csname PY@tok@nl\endcsname{\def\PY@tc##1{\textcolor[rgb]{0.63,0.63,0.00}{##1}}}
\expandafter\def\csname PY@tok@ni\endcsname{\let\PY@bf=\textbf\def\PY@tc##1{\textcolor[rgb]{0.60,0.60,0.60}{##1}}}
\expandafter\def\csname PY@tok@na\endcsname{\def\PY@tc##1{\textcolor[rgb]{0.49,0.56,0.16}{##1}}}
\expandafter\def\csname PY@tok@nt\endcsname{\let\PY@bf=\textbf\def\PY@tc##1{\textcolor[rgb]{0.00,0.50,0.00}{##1}}}
\expandafter\def\csname PY@tok@nd\endcsname{\def\PY@tc##1{\textcolor[rgb]{0.67,0.13,1.00}{##1}}}
\expandafter\def\csname PY@tok@s\endcsname{\def\PY@tc##1{\textcolor[rgb]{0.73,0.13,0.13}{##1}}}
\expandafter\def\csname PY@tok@sd\endcsname{\let\PY@it=\textit\def\PY@tc##1{\textcolor[rgb]{0.73,0.13,0.13}{##1}}}
\expandafter\def\csname PY@tok@si\endcsname{\let\PY@bf=\textbf\def\PY@tc##1{\textcolor[rgb]{0.73,0.40,0.53}{##1}}}
\expandafter\def\csname PY@tok@se\endcsname{\let\PY@bf=\textbf\def\PY@tc##1{\textcolor[rgb]{0.73,0.40,0.13}{##1}}}
\expandafter\def\csname PY@tok@sr\endcsname{\def\PY@tc##1{\textcolor[rgb]{0.73,0.40,0.53}{##1}}}
\expandafter\def\csname PY@tok@ss\endcsname{\def\PY@tc##1{\textcolor[rgb]{0.10,0.09,0.49}{##1}}}
\expandafter\def\csname PY@tok@sx\endcsname{\def\PY@tc##1{\textcolor[rgb]{0.00,0.50,0.00}{##1}}}
\expandafter\def\csname PY@tok@m\endcsname{\def\PY@tc##1{\textcolor[rgb]{0.40,0.40,0.40}{##1}}}
\expandafter\def\csname PY@tok@gh\endcsname{\let\PY@bf=\textbf\def\PY@tc##1{\textcolor[rgb]{0.00,0.00,0.50}{##1}}}
\expandafter\def\csname PY@tok@gu\endcsname{\let\PY@bf=\textbf\def\PY@tc##1{\textcolor[rgb]{0.50,0.00,0.50}{##1}}}
\expandafter\def\csname PY@tok@gd\endcsname{\def\PY@tc##1{\textcolor[rgb]{0.63,0.00,0.00}{##1}}}
\expandafter\def\csname PY@tok@gi\endcsname{\def\PY@tc##1{\textcolor[rgb]{0.00,0.63,0.00}{##1}}}
\expandafter\def\csname PY@tok@gr\endcsname{\def\PY@tc##1{\textcolor[rgb]{1.00,0.00,0.00}{##1}}}
\expandafter\def\csname PY@tok@ge\endcsname{\let\PY@it=\textit}
\expandafter\def\csname PY@tok@gs\endcsname{\let\PY@bf=\textbf}
\expandafter\def\csname PY@tok@gp\endcsname{\let\PY@bf=\textbf\def\PY@tc##1{\textcolor[rgb]{0.00,0.00,0.50}{##1}}}
\expandafter\def\csname PY@tok@go\endcsname{\def\PY@tc##1{\textcolor[rgb]{0.53,0.53,0.53}{##1}}}
\expandafter\def\csname PY@tok@gt\endcsname{\def\PY@tc##1{\textcolor[rgb]{0.00,0.27,0.87}{##1}}}
\expandafter\def\csname PY@tok@err\endcsname{\def\PY@bc##1{\setlength{\fboxsep}{0pt}\fcolorbox[rgb]{1.00,0.00,0.00}{1,1,1}{\strut ##1}}}
\expandafter\def\csname PY@tok@kc\endcsname{\let\PY@bf=\textbf\def\PY@tc##1{\textcolor[rgb]{0.00,0.50,0.00}{##1}}}
\expandafter\def\csname PY@tok@kd\endcsname{\let\PY@bf=\textbf\def\PY@tc##1{\textcolor[rgb]{0.00,0.50,0.00}{##1}}}
\expandafter\def\csname PY@tok@kn\endcsname{\let\PY@bf=\textbf\def\PY@tc##1{\textcolor[rgb]{0.00,0.50,0.00}{##1}}}
\expandafter\def\csname PY@tok@kr\endcsname{\let\PY@bf=\textbf\def\PY@tc##1{\textcolor[rgb]{0.00,0.50,0.00}{##1}}}
\expandafter\def\csname PY@tok@bp\endcsname{\def\PY@tc##1{\textcolor[rgb]{0.00,0.50,0.00}{##1}}}
\expandafter\def\csname PY@tok@fm\endcsname{\def\PY@tc##1{\textcolor[rgb]{0.00,0.00,1.00}{##1}}}
\expandafter\def\csname PY@tok@vc\endcsname{\def\PY@tc##1{\textcolor[rgb]{0.10,0.09,0.49}{##1}}}
\expandafter\def\csname PY@tok@vg\endcsname{\def\PY@tc##1{\textcolor[rgb]{0.10,0.09,0.49}{##1}}}
\expandafter\def\csname PY@tok@vi\endcsname{\def\PY@tc##1{\textcolor[rgb]{0.10,0.09,0.49}{##1}}}
\expandafter\def\csname PY@tok@vm\endcsname{\def\PY@tc##1{\textcolor[rgb]{0.10,0.09,0.49}{##1}}}
\expandafter\def\csname PY@tok@sa\endcsname{\def\PY@tc##1{\textcolor[rgb]{0.73,0.13,0.13}{##1}}}
\expandafter\def\csname PY@tok@sb\endcsname{\def\PY@tc##1{\textcolor[rgb]{0.73,0.13,0.13}{##1}}}
\expandafter\def\csname PY@tok@sc\endcsname{\def\PY@tc##1{\textcolor[rgb]{0.73,0.13,0.13}{##1}}}
\expandafter\def\csname PY@tok@dl\endcsname{\def\PY@tc##1{\textcolor[rgb]{0.73,0.13,0.13}{##1}}}
\expandafter\def\csname PY@tok@s2\endcsname{\def\PY@tc##1{\textcolor[rgb]{0.73,0.13,0.13}{##1}}}
\expandafter\def\csname PY@tok@sh\endcsname{\def\PY@tc##1{\textcolor[rgb]{0.73,0.13,0.13}{##1}}}
\expandafter\def\csname PY@tok@s1\endcsname{\def\PY@tc##1{\textcolor[rgb]{0.73,0.13,0.13}{##1}}}
\expandafter\def\csname PY@tok@mb\endcsname{\def\PY@tc##1{\textcolor[rgb]{0.40,0.40,0.40}{##1}}}
\expandafter\def\csname PY@tok@mf\endcsname{\def\PY@tc##1{\textcolor[rgb]{0.40,0.40,0.40}{##1}}}
\expandafter\def\csname PY@tok@mh\endcsname{\def\PY@tc##1{\textcolor[rgb]{0.40,0.40,0.40}{##1}}}
\expandafter\def\csname PY@tok@mi\endcsname{\def\PY@tc##1{\textcolor[rgb]{0.40,0.40,0.40}{##1}}}
\expandafter\def\csname PY@tok@il\endcsname{\def\PY@tc##1{\textcolor[rgb]{0.40,0.40,0.40}{##1}}}
\expandafter\def\csname PY@tok@mo\endcsname{\def\PY@tc##1{\textcolor[rgb]{0.40,0.40,0.40}{##1}}}
\expandafter\def\csname PY@tok@ch\endcsname{\let\PY@it=\textit\def\PY@tc##1{\textcolor[rgb]{0.25,0.50,0.50}{##1}}}
\expandafter\def\csname PY@tok@cm\endcsname{\let\PY@it=\textit\def\PY@tc##1{\textcolor[rgb]{0.25,0.50,0.50}{##1}}}
\expandafter\def\csname PY@tok@cpf\endcsname{\let\PY@it=\textit\def\PY@tc##1{\textcolor[rgb]{0.25,0.50,0.50}{##1}}}
\expandafter\def\csname PY@tok@c1\endcsname{\let\PY@it=\textit\def\PY@tc##1{\textcolor[rgb]{0.25,0.50,0.50}{##1}}}
\expandafter\def\csname PY@tok@cs\endcsname{\let\PY@it=\textit\def\PY@tc##1{\textcolor[rgb]{0.25,0.50,0.50}{##1}}}

\def\PYZbs{\char`\\}
\def\PYZus{\char`\_}
\def\PYZob{\char`\{}
\def\PYZcb{\char`\}}
\def\PYZca{\char`\^}
\def\PYZam{\char`\&}
\def\PYZlt{\char`\<}
\def\PYZgt{\char`\>}
\def\PYZsh{\char`\#}
\def\PYZpc{\char`\%}
\def\PYZdl{\char`\$}
\def\PYZhy{\char`\-}
\def\PYZsq{\char`\'}
\def\PYZdq{\char`\"}
\def\PYZti{\char`\~}
% for compatibility with earlier versions
\def\PYZat{@}
\def\PYZlb{[}
\def\PYZrb{]}
\makeatother


    % Exact colors from NB
    \definecolor{incolor}{rgb}{0.0, 0.0, 0.5}
    \definecolor{outcolor}{rgb}{0.545, 0.0, 0.0}



    
    % Prevent overflowing lines due to hard-to-break entities
    \sloppy 
    % Setup hyperref package
    \hypersetup{
      breaklinks=true,  % so long urls are correctly broken across lines
      colorlinks=true,
      urlcolor=urlcolor,
      linkcolor=linkcolor,
      citecolor=citecolor,
      }
    % Slightly bigger margins than the latex defaults
    
    \geometry{verbose,tmargin=1in,bmargin=1in,lmargin=1in,rmargin=1in}
    
    

    \begin{document}
    
    
    \maketitle
    
    

    
    \hypertarget{core-julia}{%
\section{Core Julia}\label{core-julia}}

This chapter consists of six sections:

\begin{enumerate}
\def\labelenumi{\arabic{enumi}.}
\tightlist
\item
  Variable names;
\item
  Operators;
\item
  Types;
\item
  Data Structure;
\item
  Control flow;
\item
  Functions.
\end{enumerate}

\hypertarget{variable-names}{%
\subsection{Variable names}\label{variable-names}}

Variable names are case sensitive and Unicode names (in UTF-8 encoding)
may be used. And names must begin with a letter, no matter lowercase or
uppercase, an underscore or a Unicode code point larger than 00A0, and
other Unicode points, even a latex symbol. We can also redefine the
constants here, such as \(\pi\). But bulid-in statements are not allowed
in this language. The examples are below.

    \begin{Verbatim}[commandchars=\\\{\}]
{\color{incolor}In [{\color{incolor}15}]:} \PY{c}{\PYZsh{}\PYZsh{} right variable names}
         \PY{n}{z} \PY{o}{=} \PY{l+m+mi}{100}
         \PY{n}{y} \PY{o}{=} \PY{l+m+mf}{1.0}
         \PY{n}{s} \PY{o}{=} \PY{l+s}{\PYZdq{}}\PY{l+s}{m}\PY{l+s}{y}\PY{l+s}{\PYZus{}}\PY{l+s}{v}\PY{l+s}{a}\PY{l+s}{r}\PY{l+s}{i}\PY{l+s}{a}\PY{l+s}{b}\PY{l+s}{l}\PY{l+s}{e}\PY{l+s}{\PYZdq{}}
         \PY{n}{data\PYZus{}science} \PY{o}{=} \PY{l+s}{\PYZdq{}}\PY{l+s}{t}\PY{l+s}{r}\PY{l+s}{u}\PY{l+s}{e}\PY{l+s}{\PYZdq{}}
         \PY{n}{datascience} \PY{o}{=} \PY{k+kc}{true}
         
         \PY{c}{\PYZsh{}\PYZsh{} wrong variable names are here if run them, return an error}
         \PY{c}{\PYZsh{}\PYZsh{} if = 1.2}
         \PY{c}{\PYZsh{}\PYZsh{} else = true}
         \PY{c}{\PYZsh{}\PYZsh{}}
\end{Verbatim}

\begin{Verbatim}[commandchars=\\\{\}]
{\color{outcolor}Out[{\color{outcolor}15}]:} true
\end{Verbatim}
            
    \hypertarget{operators}{%
\subsection{Operators}\label{operators}}

The operators are similar to \emph{R} and \emph{Python}. Four main
categories of operators in Julia:

\begin{enumerate}
\def\labelenumi{\arabic{enumi}.}
\tightlist
\item
  Arithmetic;
\item
  Updating;
\item
  Numeric comparison;
\item
  Bitwise.
\end{enumerate}

    \begin{Verbatim}[commandchars=\\\{\}]
{\color{incolor}In [{\color{incolor}16}]:} \PY{n}{x} \PY{o}{=} \PY{l+m+mi}{2}
         \PY{n}{y} \PY{o}{=} \PY{l+m+mi}{3}
         \PY{n}{z} \PY{o}{=} \PY{l+m+mi}{4}
         
         \PY{n}{x} \PY{o}{+} \PY{n}{y} 
         \PY{n}{x}\PY{o}{\PYZca{}}\PY{n}{y}
         \PY{n}{x} \PY{o}{+=} \PY{l+m+mi}{2}
         \PY{n}{x}
         \PY{n}{y}
\end{Verbatim}

\begin{Verbatim}[commandchars=\\\{\}]
{\color{outcolor}Out[{\color{outcolor}16}]:} 3
\end{Verbatim}
            
    When constructing expressing with multiple operators, the order in which
these operators are applied to the expression is known as operator
precedence.

With the parentheses () inclueded in the expression, we can control the
order by ourselves like the following code

    \begin{Verbatim}[commandchars=\\\{\}]
{\color{incolor}In [{\color{incolor}17}]:} \PY{n}{x}\PY{o}{*}\PY{n}{y}\PY{o}{+}\PY{n}{z}\PY{o}{\PYZca{}}\PY{l+m+mi}{2}
         \PY{n}{x}\PY{o}{*}\PY{p}{(}\PY{n}{y}\PY{o}{+}\PY{n}{z}\PY{o}{\PYZca{}}\PY{l+m+mi}{2}\PY{p}{)}
         \PY{p}{(}\PY{n}{x}\PY{o}{*}\PY{n}{y}\PY{p}{)}\PY{o}{+}\PY{p}{(}\PY{n}{z}\PY{o}{\PYZca{}}\PY{l+m+mi}{2}\PY{p}{)}
\end{Verbatim}

\begin{Verbatim}[commandchars=\\\{\}]
{\color{outcolor}Out[{\color{outcolor}17}]:} 28
\end{Verbatim}
            
    \hypertarget{types}{%
\subsection{Types}\label{types}}

\hypertarget{numeric}{%
\subsubsection{Numeric}\label{numeric}}

Julia offers full support for real and complex numbers. The internal
variable \texttt{Sys.WORD\_SIZE} displays the architecture type of the
computer. Minimum and Maximum can be showed by \texttt{typemin()} and
\texttt{typemax()}.

    \begin{Verbatim}[commandchars=\\\{\}]
{\color{incolor}In [{\color{incolor}18}]:} \PY{n}{Sys}\PY{o}{.}\PY{n+nb}{WORD\PYZus{}SIZE}
\end{Verbatim}

\begin{Verbatim}[commandchars=\\\{\}]
{\color{outcolor}Out[{\color{outcolor}18}]:} 64
\end{Verbatim}
            
    \begin{Verbatim}[commandchars=\\\{\}]
{\color{incolor}In [{\color{incolor}19}]:} \PY{n}{typemax}\PY{p}{(}\PY{k+kt}{Int}\PY{p}{)}
\end{Verbatim}

\begin{Verbatim}[commandchars=\\\{\}]
{\color{outcolor}Out[{\color{outcolor}19}]:} 9223372036854775807
\end{Verbatim}
            
    \begin{Verbatim}[commandchars=\\\{\}]
{\color{incolor}In [{\color{incolor}20}]:} \PY{n}{typemin}\PY{p}{(}\PY{k+kt}{Int}\PY{p}{)}
\end{Verbatim}

\begin{Verbatim}[commandchars=\\\{\}]
{\color{outcolor}Out[{\color{outcolor}20}]:} -9223372036854775808
\end{Verbatim}
            
    \begin{Verbatim}[commandchars=\\\{\}]
{\color{incolor}In [{\color{incolor}21}]:} \PY{n}{typemax}\PY{p}{(}\PY{k+kt}{Int64}\PY{p}{)}
\end{Verbatim}

\begin{Verbatim}[commandchars=\\\{\}]
{\color{outcolor}Out[{\color{outcolor}21}]:} 9223372036854775807
\end{Verbatim}
            
    \begin{Verbatim}[commandchars=\\\{\}]
{\color{incolor}In [{\color{incolor}22}]:} \PY{n}{typemax}\PY{p}{(}\PY{k+kt}{Float32}\PY{p}{)}
\end{Verbatim}

\begin{Verbatim}[commandchars=\\\{\}]
{\color{outcolor}Out[{\color{outcolor}22}]:} Inf32
\end{Verbatim}
            
    Some types use leftmost bit to control the sign, such as \texttt{Int64},
but others use that bit as value and without sign like \texttt{UInt128}.

Boolean values are 8-bit integers, with false being 0 and true being
1.Overflow errors will happen if the result is larger or smaller than
its allowable size.

    \begin{Verbatim}[commandchars=\\\{\}]
{\color{incolor}In [{\color{incolor}23}]:} \PY{n}{literal\PYZus{}int} \PY{o}{=} \PY{l+m+mi}{1}
         \PY{n}{println}\PY{p}{(}\PY{l+s}{\PYZdq{}}\PY{l+s}{t}\PY{l+s}{y}\PY{l+s}{p}\PY{l+s}{e}\PY{l+s}{o}\PY{l+s}{f}\PY{l+s}{(}\PY{l+s}{l}\PY{l+s}{i}\PY{l+s}{t}\PY{l+s}{e}\PY{l+s}{r}\PY{l+s}{a}\PY{l+s}{l}\PY{l+s}{\PYZus{}}\PY{l+s}{i}\PY{l+s}{n}\PY{l+s}{t}\PY{l+s}{)}\PY{l+s}{:}\PY{l+s}{\PYZdq{}}\PY{p}{,}\PY{n}{typeof}\PY{p}{(}\PY{n}{literal\PYZus{}int}\PY{p}{)}\PY{p}{)}
\end{Verbatim}

    \begin{Verbatim}[commandchars=\\\{\}]
typeof(literal\_int):Int64

    \end{Verbatim}

    \begin{Verbatim}[commandchars=\\\{\}]
{\color{incolor}In [{\color{incolor}24}]:} \PY{n}{x} \PY{o}{=} \PY{n}{typemax}\PY{p}{(}\PY{k+kt}{Int64}\PY{p}{)}
\end{Verbatim}

\begin{Verbatim}[commandchars=\\\{\}]
{\color{outcolor}Out[{\color{outcolor}24}]:} 9223372036854775807
\end{Verbatim}
            
    \begin{Verbatim}[commandchars=\\\{\}]
{\color{incolor}In [{\color{incolor}25}]:} \PY{n}{x} \PY{o}{+=} \PY{l+m+mi}{1}
\end{Verbatim}

\begin{Verbatim}[commandchars=\\\{\}]
{\color{outcolor}Out[{\color{outcolor}25}]:} -9223372036854775808
\end{Verbatim}
            
    \hypertarget{floats}{%
\subsubsection{Floats}\label{floats}}

Floats are similar to scientific notation. They are made up of three
components: asigned interger whose length determines the precision, the
base used to represent te number and a signed integer htat changes the
magnitude of floating point number (the exponent). \texttt{Float64}
literals are distinguished by having an \texttt{e} before hte power, and
can be defined in hexadecimal. \texttt{Float32} literals are
distinguished by having an \texttt{f} in place of the \texttt{e}. THere
are three \texttt{Float64} values that do not occur on the real line:

\begin{enumerate}
\def\labelenumi{\arabic{enumi}.}
\tightlist
\item
  \texttt{Inf}, positive indinity: a value larger than all finite
  floating point numbers, equal to itself, and greater than every other
  floating point value but \texttt{NaN};
\item
  \texttt{-Inf}, negative infinity: a value less than all finite
  floating point numbers, equal to itself, and less than every other
  floating point value but \texttt{NaN};
\item
  \texttt{NaN}, not a number: a value not equal to any floating point
  value, and not \texttt{==},\texttt{\textless{}} or
  \texttt{\textgreater{}} than any floating point value, including
  itself.
\end{enumerate}

Some tips:

\begin{enumerate}
\def\labelenumi{\arabic{enumi}.}
\tightlist
\item
  digit separation using an \texttt{\_};
\end{enumerate}

    \begin{Verbatim}[commandchars=\\\{\}]
{\color{incolor}In [{\color{incolor}26}]:} \PY{n}{x1} \PY{o}{=} \PY{l+m+mf}{1.0}
         \PY{n}{x64} \PY{o}{=} \PY{l+m+mf}{15e\PYZhy{}15}
         \PY{n}{x32} \PY{o}{=} \PY{l+m+mf}{2.5f\PYZhy{}4}
         \PY{n}{println}\PY{p}{(}\PY{l+s}{\PYZdq{}}\PY{l+s}{x}\PY{l+s}{1}\PY{l+s}{ }\PY{l+s}{i}\PY{l+s}{s}\PY{l+s}{ }\PY{l+s}{\PYZdq{}}\PY{p}{,}\PY{n}{typeof}\PY{p}{(}\PY{n}{x1}\PY{p}{)}\PY{p}{)}
         \PY{n}{println}\PY{p}{(}\PY{l+s}{\PYZdq{}}\PY{l+s+se}{\PYZbs{}n}\PY{l+s}{x}\PY{l+s}{6}\PY{l+s}{4}\PY{l+s}{ }\PY{l+s}{i}\PY{l+s}{s}\PY{l+s}{ }\PY{l+s}{\PYZdq{}}\PY{p}{,} \PY{n}{typeof}\PY{p}{(}\PY{n}{x64}\PY{p}{)}\PY{p}{)}
         \PY{n}{println}\PY{p}{(}\PY{l+s}{\PYZdq{}}\PY{l+s+se}{\PYZbs{}n}\PY{l+s}{x}\PY{l+s}{3}\PY{l+s}{2}\PY{l+s}{ }\PY{l+s}{i}\PY{l+s}{s}\PY{l+s}{ }\PY{l+s}{\PYZdq{}}\PY{p}{,} \PY{n}{typeof}\PY{p}{(}\PY{n}{x32}\PY{p}{)}\PY{p}{)}
\end{Verbatim}

    \begin{Verbatim}[commandchars=\\\{\}]
x1 is Float64

x64 is Float64

x32 is Float32

    \end{Verbatim}

    \begin{Verbatim}[commandchars=\\\{\}]
{\color{incolor}In [{\color{incolor}27}]:} \PY{l+m+mf}{9.2\PYZus{}4}\PY{o}{==}\PY{l+m+mf}{9.24}
         \PY{c}{\PYZsh{}\PYZsh{} digit separation using an \PYZus{}}
\end{Verbatim}

\begin{Verbatim}[commandchars=\\\{\}]
{\color{outcolor}Out[{\color{outcolor}27}]:} true
\end{Verbatim}
            
    In machine, it is defined that the smallest value is \(1+z\neq 1\). In
Julia, the value of epsilon for a particular machine can be found via
the \texttt{eps()} function.

The spacing between floating point numbers and the value of machine
epsilon is important to understand because it can help avoid certain
types of errors.

There are also float underflow errors, which occur when the result of a
calculation is smaller than machine epsilon or when numbers of similar
precision are subtracted.

    \begin{Verbatim}[commandchars=\\\{\}]
{\color{incolor}In [{\color{incolor}28}]:} \PY{n}{eps}\PY{p}{(}\PY{p}{)}
\end{Verbatim}

\begin{Verbatim}[commandchars=\\\{\}]
{\color{outcolor}Out[{\color{outcolor}28}]:} 2.220446049250313e-16
\end{Verbatim}
            
    \begin{Verbatim}[commandchars=\\\{\}]
{\color{incolor}In [{\color{incolor}29}]:} \PY{n}{n1} \PY{o}{=} \PY{p}{[}\PY{l+m+mf}{1e\PYZhy{}25}\PY{p}{,}\PY{l+m+mf}{1e\PYZhy{}5}\PY{p}{,}\PY{l+m+mf}{1.}\PY{p}{,}\PY{l+m+mf}{1e5}\PY{p}{,}\PY{l+m+mf}{1e25}\PY{p}{]}
         \PY{k}{for} \PY{n}{i} \PY{k+kp}{in} \PY{n}{n1}
             \PY{n}{println}\PY{p}{(}\PY{o}{*}\PY{p}{(}\PY{n}{i}\PY{p}{,}\PY{n}{eps}\PY{p}{(}\PY{p}{)}\PY{p}{)}\PY{p}{)}
         \PY{k}{end}
\end{Verbatim}

    \begin{Verbatim}[commandchars=\\\{\}]
2.2204460492503132e-41
2.2204460492503133e-21
2.220446049250313e-16
2.220446049250313e-11
2.2204460492503133e9

    \end{Verbatim}

    \hypertarget{strings}{%
\subsubsection{Strings}\label{strings}}

In Julia, a string is a sequence of Unicode code points, using UTF-8
encoding. Characters in strings have an index value within the string.
It is worth noting that Julia indices start at \textbf{position 1},
similar to \textbf{R} but different to \textbf{Python}. The key word
\texttt{end} can be used to represent the last index. Herein, we will
deal with ASCII characters only. Note that \texttt{String} is the
built-in type for string and string literals,and \texttt{Char} is the
built-in type used to represent single chararcters. In fact,
\texttt{Char} is a numeric value representing a Unicode code point. The
value of a string cannot be changed, i.e., strings are immutable, and a
new string must be built from another string. Strings are defined by
double or triple quotes.

    \begin{Verbatim}[commandchars=\\\{\}]
{\color{incolor}In [{\color{incolor}30}]:} \PY{n}{s1} \PY{o}{=} \PY{l+s}{\PYZdq{}}\PY{l+s}{H}\PY{l+s}{i}\PY{l+s}{\PYZdq{}}
\end{Verbatim}

\begin{Verbatim}[commandchars=\\\{\}]
{\color{outcolor}Out[{\color{outcolor}30}]:} "Hi"
\end{Verbatim}
            
    \begin{Verbatim}[commandchars=\\\{\}]
{\color{incolor}In [{\color{incolor}31}]:} \PY{n}{s2} \PY{o}{=} \PY{l+s}{\PYZdq{}\PYZdq{}\PYZdq{}}\PY{l+s}{I}\PY{l+s}{ }\PY{l+s}{h}\PY{l+s}{a}\PY{l+s}{v}\PY{l+s}{e}\PY{l+s}{ }\PY{l+s}{a}\PY{l+s}{ }\PY{l+s}{\PYZdq{}}\PY{l+s}{q}\PY{l+s}{u}\PY{l+s}{o}\PY{l+s}{t}\PY{l+s}{e}\PY{l+s}{\PYZdq{}}\PY{l+s}{ }\PY{l+s}{c}\PY{l+s}{h}\PY{l+s}{a}\PY{l+s}{r}\PY{l+s}{a}\PY{l+s}{c}\PY{l+s}{t}\PY{l+s}{e}\PY{l+s}{r}\PY{l+s}{\PYZdq{}\PYZdq{}\PYZdq{}}
\end{Verbatim}

\begin{Verbatim}[commandchars=\\\{\}]
{\color{outcolor}Out[{\color{outcolor}31}]:} "I have a \textbackslash{}"quote\textbackslash{}" character"
\end{Verbatim}
            
    String can be sliced using range indexing, e.g.,
\texttt{my\_string{[}4:6{]}} would return a substring of
\texttt{my\_string} containing the 4th, 5th and 6th charaters of
\texttt{my\_string}. Concatenation can be done in two ways: using the
\texttt{string()} function or with the \texttt{*} operator. Note this is
a somewhat unusual feature of Julia. many other language use \texttt{+}
to perform concatenation. String interpolation takes place when a string
literal is defined with a variable inside its instantiation. The
cariable is prepended with \texttt{\$}. By using variables insider the
string's definition, complex strings can be built in a readbale form,
without multiple string multiplications.

    \begin{Verbatim}[commandchars=\\\{\}]
{\color{incolor}In [{\color{incolor}32}]:} \PY{c}{\PYZsh{}\PYZsh{} some examples of strings}
         \PY{n}{str} \PY{o}{=} \PY{l+s}{\PYZdq{}}\PY{l+s}{D}\PY{l+s}{a}\PY{l+s}{t}\PY{l+s}{a}\PY{l+s}{ }\PY{l+s}{s}\PY{l+s}{c}\PY{l+s}{i}\PY{l+s}{e}\PY{l+s}{n}\PY{l+s}{c}\PY{l+s}{e}\PY{l+s}{ }\PY{l+s}{i}\PY{l+s}{s}\PY{l+s}{ }\PY{l+s}{f}\PY{l+s}{u}\PY{l+s}{n}\PY{l+s}{!}\PY{l+s}{\PYZdq{}}
         \PY{n}{str}\PY{p}{[}\PY{l+m+mi}{1}\PY{p}{]}
\end{Verbatim}

\begin{Verbatim}[commandchars=\\\{\}]
{\color{outcolor}Out[{\color{outcolor}32}]:} 'D': ASCII/Unicode U+0044 (category Lu: Letter, uppercase)
\end{Verbatim}
            
    \begin{Verbatim}[commandchars=\\\{\}]
{\color{incolor}In [{\color{incolor}33}]:} \PY{n}{str}\PY{p}{[}\PY{k}{end}\PY{p}{]}
\end{Verbatim}

\begin{Verbatim}[commandchars=\\\{\}]
{\color{outcolor}Out[{\color{outcolor}33}]:} '!': ASCII/Unicode U+0021 (category Po: Punctuation, other)
\end{Verbatim}
            
    \begin{Verbatim}[commandchars=\\\{\}]
{\color{incolor}In [{\color{incolor}34}]:} \PY{n}{str}\PY{p}{[}\PY{l+m+mi}{4}\PY{o}{:}\PY{l+m+mi}{7}\PY{p}{]}
\end{Verbatim}

\begin{Verbatim}[commandchars=\\\{\}]
{\color{outcolor}Out[{\color{outcolor}34}]:} "a sc"
\end{Verbatim}
            
    \begin{Verbatim}[commandchars=\\\{\}]
{\color{incolor}In [{\color{incolor}35}]:} \PY{n}{str}\PY{p}{[}\PY{k}{end}\PY{o}{\PYZhy{}}\PY{l+m+mi}{3}\PY{o}{:}\PY{k}{end}\PY{p}{]}
\end{Verbatim}

\begin{Verbatim}[commandchars=\\\{\}]
{\color{outcolor}Out[{\color{outcolor}35}]:} "fun!"
\end{Verbatim}
            
    \begin{Verbatim}[commandchars=\\\{\}]
{\color{incolor}In [{\color{incolor}36}]:} \PY{n}{string}\PY{p}{(}\PY{n}{str}\PY{p}{,}\PY{l+s}{\PYZdq{}}\PY{l+s}{ }\PY{l+s}{S}\PY{l+s}{u}\PY{l+s}{r}\PY{l+s}{e}\PY{l+s}{ }\PY{l+s}{i}\PY{l+s}{s}\PY{l+s}{ }\PY{l+s}{:}\PY{l+s}{)}\PY{l+s}{\PYZdq{}}\PY{p}{)}
\end{Verbatim}

\begin{Verbatim}[commandchars=\\\{\}]
{\color{outcolor}Out[{\color{outcolor}36}]:} "Data science is fun! Sure is :)"
\end{Verbatim}
            
    \begin{Verbatim}[commandchars=\\\{\}]
{\color{incolor}In [{\color{incolor}37}]:} \PY{n}{str} \PY{o}{*} \PY{l+s}{\PYZdq{}}\PY{l+s}{S}\PY{l+s}{u}\PY{l+s}{r}\PY{l+s}{e}\PY{l+s}{ }\PY{l+s}{i}\PY{l+s}{s}\PY{l+s}{ }\PY{l+s}{:}\PY{l+s}{)}\PY{l+s}{\PYZdq{}}
\end{Verbatim}

\begin{Verbatim}[commandchars=\\\{\}]
{\color{outcolor}Out[{\color{outcolor}37}]:} "Data science is fun!Sure is :)"
\end{Verbatim}
            
    \begin{Verbatim}[commandchars=\\\{\}]
{\color{incolor}In [{\color{incolor}38}]:} \PY{c}{\PYZsh{} Interpolation}
         \PY{l+s}{\PYZdq{}}\PY{l+s}{1}\PY{l+s}{+}\PY{l+s}{2}\PY{l+s}{=}\PY{l+s+si}{\PYZdl{}}\PY{p}{(}\PY{l+m+mi}{1}\PY{o}{+}\PY{l+m+mi}{2}\PY{p}{)}\PY{l+s}{\PYZdq{}}
\end{Verbatim}

\begin{Verbatim}[commandchars=\\\{\}]
{\color{outcolor}Out[{\color{outcolor}38}]:} "1+2=3"
\end{Verbatim}
            
    \begin{Verbatim}[commandchars=\\\{\}]
{\color{incolor}In [{\color{incolor}39}]:} \PY{n}{word1}\PY{o}{=}\PY{l+s}{\PYZdq{}}\PY{l+s}{J}\PY{l+s}{u}\PY{l+s}{l}\PY{l+s}{i}\PY{l+s}{a}\PY{l+s}{\PYZdq{}}
         \PY{n}{word2}\PY{o}{=}\PY{l+s}{\PYZdq{}}\PY{l+s}{d}\PY{l+s}{a}\PY{l+s}{t}\PY{l+s}{a}\PY{l+s}{\PYZdq{}}
         \PY{n}{word3}\PY{o}{=}\PY{l+s}{\PYZdq{}}\PY{l+s}{s}\PY{l+s}{c}\PY{l+s}{i}\PY{l+s}{e}\PY{l+s}{n}\PY{l+s}{c}\PY{l+s}{e}\PY{l+s}{\PYZdq{}}
         \PY{l+s}{\PYZdq{}}\PY{l+s+si}{\PYZdl{}word1}\PY{l+s}{ }\PY{l+s}{i}\PY{l+s}{s}\PY{l+s}{ }\PY{l+s}{g}\PY{l+s}{r}\PY{l+s}{e}\PY{l+s}{a}\PY{l+s}{t}\PY{l+s}{ }\PY{l+s}{f}\PY{l+s}{o}\PY{l+s}{r}\PY{l+s}{ }\PY{l+s+si}{\PYZdl{}word2}\PY{l+s}{ }\PY{l+s+si}{\PYZdl{}word3}\PY{l+s}{\PYZdq{}}
\end{Verbatim}

\begin{Verbatim}[commandchars=\\\{\}]
{\color{outcolor}Out[{\color{outcolor}39}]:} "Julia is great for data science"
\end{Verbatim}
            
    Strings can be comared lexicographically using comarsion operators,
e.g.~\texttt{==}, \texttt{\textgreater{}}, etc. Lexicograohical
comparison involves sequentially comparing string elements with the same
position, until one pair of elements falsiies the comparison, or the end
of the string is reached. Some useful string functions are:

\begin{itemize}
\tightlist
\item
  \texttt{findfirst(par,str)} returns the indices of the characters in
  the string \texttt{str} matching the pattern \texttt{pat}.
\item
  \texttt{occursin(substr,\ str)} returns \texttt{true/false} depending
  on the presence/absence of \texttt{substr} in \texttt{str}.
\item
  `repeat(str,n)' generates a new string that is the original string
  \texttt{str} repeated \texttt{n} times.
\item
  `length(str)' returns the number of charaters in the string
  \texttt{str}.
\item
  `replace(str,ptn =\textgreater{} rep)' searches string \texttt{str}
  for the pattern \texttt{ptn} and, if it is present, replaces it with
  \texttt{rep}.
\end{itemize}

    \begin{Verbatim}[commandchars=\\\{\}]
{\color{incolor}In [{\color{incolor}40}]:} \PY{c}{\PYZsh{} Lexicographical comparison}
         \PY{n}{s1} \PY{o}{=} \PY{l+s}{\PYZdq{}}\PY{l+s}{a}\PY{l+s}{b}\PY{l+s}{c}\PY{l+s}{d}\PY{l+s}{\PYZdq{}}
         \PY{n}{s2} \PY{o}{=} \PY{l+s}{\PYZdq{}}\PY{l+s}{a}\PY{l+s}{b}\PY{l+s}{d}\PY{l+s}{e}\PY{l+s}{\PYZdq{}}
         
         \PY{n}{s1} \PY{o}{==} \PY{n}{s2}
\end{Verbatim}

\begin{Verbatim}[commandchars=\\\{\}]
{\color{outcolor}Out[{\color{outcolor}40}]:} false
\end{Verbatim}
            
    \begin{Verbatim}[commandchars=\\\{\}]
{\color{incolor}In [{\color{incolor}41}]:} \PY{n}{s1} \PY{o}{\PYZlt{}} \PY{n}{s2}
\end{Verbatim}

\begin{Verbatim}[commandchars=\\\{\}]
{\color{outcolor}Out[{\color{outcolor}41}]:} true
\end{Verbatim}
            
    \begin{Verbatim}[commandchars=\\\{\}]
{\color{incolor}In [{\color{incolor}42}]:} \PY{n}{s1} \PY{o}{\PYZgt{}} \PY{n}{s2}
\end{Verbatim}

\begin{Verbatim}[commandchars=\\\{\}]
{\color{outcolor}Out[{\color{outcolor}42}]:} false
\end{Verbatim}
            
    \begin{Verbatim}[commandchars=\\\{\}]
{\color{incolor}In [{\color{incolor}43}]:} \PY{c}{\PYZsh{} String functions}
         \PY{n}{str} \PY{o}{=} \PY{l+s}{\PYZdq{}}\PY{l+s}{D}\PY{l+s}{a}\PY{l+s}{t}\PY{l+s}{a}\PY{l+s}{ }\PY{l+s}{s}\PY{l+s}{c}\PY{l+s}{i}\PY{l+s}{e}\PY{l+s}{n}\PY{l+s}{c}\PY{l+s}{e}\PY{l+s}{ }\PY{l+s}{i}\PY{l+s}{s}\PY{l+s}{ }\PY{l+s}{f}\PY{l+s}{u}\PY{l+s}{n}\PY{l+s}{!}\PY{l+s}{\PYZdq{}}
         
         \PY{n}{findfirst}\PY{p}{(}\PY{l+s}{\PYZdq{}}\PY{l+s}{D}\PY{l+s}{a}\PY{l+s}{t}\PY{l+s}{a}\PY{l+s}{\PYZdq{}}\PY{p}{,}\PY{n}{str}\PY{p}{)}
\end{Verbatim}

\begin{Verbatim}[commandchars=\\\{\}]
{\color{outcolor}Out[{\color{outcolor}43}]:} 1:4
\end{Verbatim}
            
    \begin{Verbatim}[commandchars=\\\{\}]
{\color{incolor}In [{\color{incolor}44}]:} \PY{n}{occursin}\PY{p}{(}\PY{l+s}{\PYZdq{}}\PY{l+s}{a}\PY{l+s}{t}\PY{l+s}{a}\PY{l+s}{\PYZdq{}}\PY{p}{,}\PY{n}{str}\PY{p}{)}
\end{Verbatim}

\begin{Verbatim}[commandchars=\\\{\}]
{\color{outcolor}Out[{\color{outcolor}44}]:} true
\end{Verbatim}
            
    \begin{Verbatim}[commandchars=\\\{\}]
{\color{incolor}In [{\color{incolor}45}]:} \PY{n}{replace}\PY{p}{(}\PY{n}{str}\PY{p}{,}\PY{l+s}{\PYZdq{}}\PY{l+s}{f}\PY{l+s}{u}\PY{l+s}{n}\PY{l+s}{\PYZdq{}}\PY{o}{=\PYZgt{}}\PY{l+s}{\PYZdq{}}\PY{l+s}{g}\PY{l+s}{r}\PY{l+s}{e}\PY{l+s}{a}\PY{l+s}{t}\PY{l+s}{\PYZdq{}}\PY{p}{)}
\end{Verbatim}

\begin{Verbatim}[commandchars=\\\{\}]
{\color{outcolor}Out[{\color{outcolor}45}]:} "Data science is great!"
\end{Verbatim}
            
    Julia fully supports regular expressions (regexes). Regexes in Julia are
fully Perl compatible and fully compatible and are used to hunt for
patterns in string data. They are defined as string with a leading
\texttt{r} outside the quotes. Regular expressions are commonly used
with the following functions:

\begin{itemize}
\tightlist
\item
  \texttt{occursin(regex,str)} returns \texttt{true/false} if the regex
  has a match inthe string \texttt{str}.
\item
  \texttt{mathc(regex,str)} returns the first match of \texttt{regex} in
  the string. If there is no match, it returns the special Julia value
  \texttt{nothing}.
\item
  \texttt{eachmatch(regex,\ str)} returns all the matches of
  \texttt{regex} inthe string \texttt{str} as an array.
\end{itemize}

Regexes are a very powerful programming tool for working with text data.
However, an in-depth discussion of them is beyond the scope of this
book, and intersted readers are encouraged to consult
\href{http://scans.hebis.de/10/73/59/10735965_toc.pdf}{Friedl(2006)} for
further details.

    \begin{Verbatim}[commandchars=\\\{\}]
{\color{incolor}In [{\color{incolor}46}]:} \PY{c}{\PYZsh{} Regular expresions}
         
         \PY{c}{\PYZsh{} match alpha\PYZhy{}numeric charaters at the start of the str}
         \PY{n}{occursin}\PY{p}{(}\PY{l+s+sr}{r\PYZdq{}}\PY{l+s+sr}{[}\PY{l+s+sr}{a}\PY{l+s+sr}{\PYZhy{}}\PY{l+s+sr}{z}\PY{l+s+sr}{A}\PY{l+s+sr}{\PYZhy{}}\PY{l+s+sr}{z}\PY{l+s+sr}{0}\PY{l+s+sr}{\PYZhy{}}\PY{l+s+sr}{9}\PY{l+s+sr}{]}\PY{l+s+sr}{\PYZdq{}}\PY{p}{,}\PY{n}{str}\PY{p}{)}
\end{Verbatim}

\begin{Verbatim}[commandchars=\\\{\}]
{\color{outcolor}Out[{\color{outcolor}46}]:} true
\end{Verbatim}
            
    \begin{Verbatim}[commandchars=\\\{\}]
{\color{incolor}In [{\color{incolor}47}]:} \PY{n}{occursin}\PY{p}{(}\PY{l+s+sr}{r\PYZdq{}}\PY{l+s+sr}{\PYZca{}}\PY{l+s+sr}{[}\PY{l+s+sr}{a}\PY{l+s+sr}{\PYZhy{}}\PY{l+s+sr}{z}\PY{l+s+sr}{A}\PY{l+s+sr}{\PYZhy{}}\PY{l+s+sr}{z}\PY{l+s+sr}{0}\PY{l+s+sr}{\PYZhy{}}\PY{l+s+sr}{9}\PY{l+s+sr}{]}\PY{l+s+sr}{\PYZdq{}}\PY{p}{,}\PY{n}{str}\PY{p}{)}
\end{Verbatim}

\begin{Verbatim}[commandchars=\\\{\}]
{\color{outcolor}Out[{\color{outcolor}47}]:} true
\end{Verbatim}
            
    \begin{Verbatim}[commandchars=\\\{\}]
{\color{incolor}In [{\color{incolor}48}]:} \PY{n}{occursin}\PY{p}{(}\PY{l+s+sr}{r\PYZdq{}}\PY{l+s+sr}{[}\PY{l+s+sr}{a}\PY{l+s+sr}{\PYZhy{}}\PY{l+s+sr}{z}\PY{l+s+sr}{A}\PY{l+s+sr}{\PYZhy{}}\PY{l+s+sr}{z}\PY{l+s+sr}{0}\PY{l+s+sr}{\PYZhy{}}\PY{l+s+sr}{9}\PY{l+s+sr}{]}\PY{l+s+sr}{\PYZdl{}}\PY{l+s+sr}{\PYZdq{}}\PY{p}{,}\PY{n}{str}\PY{p}{)}
\end{Verbatim}

\begin{Verbatim}[commandchars=\\\{\}]
{\color{outcolor}Out[{\color{outcolor}48}]:} false
\end{Verbatim}
            
    \begin{Verbatim}[commandchars=\\\{\}]
{\color{incolor}In [{\color{incolor}49}]:} \PY{n}{occursin}\PY{p}{(}\PY{l+s+sr}{r\PYZdq{}}\PY{l+s+sr}{[}\PY{l+s+sr}{\PYZca{}}\PY{l+s+sr}{a}\PY{l+s+sr}{\PYZhy{}}\PY{l+s+sr}{z}\PY{l+s+sr}{A}\PY{l+s+sr}{\PYZhy{}}\PY{l+s+sr}{Z}\PY{l+s+sr}{0}\PY{l+s+sr}{\PYZhy{}}\PY{l+s+sr}{9}\PY{l+s+sr}{]}\PY{l+s+sr}{\PYZdq{}}\PY{p}{,}\PY{n}{str}\PY{p}{)}
\end{Verbatim}

\begin{Verbatim}[commandchars=\\\{\}]
{\color{outcolor}Out[{\color{outcolor}49}]:} true
\end{Verbatim}
            
    \begin{Verbatim}[commandchars=\\\{\}]
{\color{incolor}In [{\color{incolor}50}]:} \PY{c}{\PYZsh{}\PYZsh{} matches the first non\PYZhy{}alpha\PYZhy{}numeric character in the string}
         \PY{n}{match}\PY{p}{(}\PY{l+s+sr}{r\PYZdq{}}\PY{l+s+sr}{[}\PY{l+s+sr}{\PYZca{}}\PY{l+s+sr}{a}\PY{l+s+sr}{\PYZhy{}}\PY{l+s+sr}{z}\PY{l+s+sr}{A}\PY{l+s+sr}{\PYZhy{}}\PY{l+s+sr}{Z}\PY{l+s+sr}{0}\PY{l+s+sr}{\PYZhy{}}\PY{l+s+sr}{9}\PY{l+s+sr}{]}\PY{l+s+sr}{\PYZdq{}}\PY{p}{,}\PY{n}{str}\PY{p}{)}
\end{Verbatim}

\begin{Verbatim}[commandchars=\\\{\}]
{\color{outcolor}Out[{\color{outcolor}50}]:} RegexMatch(" ")
\end{Verbatim}
            
    \begin{Verbatim}[commandchars=\\\{\}]
{\color{incolor}In [{\color{incolor}51}]:} \PY{c}{\PYZsh{}\PYZsh{} matches all the non\PYZhy{}alpha\PYZhy{}numeric characters in the string}
         \PY{n}{collect}\PY{p}{(}\PY{n}{eachmatch}\PY{p}{(}\PY{l+s+sr}{r\PYZdq{}}\PY{l+s+sr}{[}\PY{l+s+sr}{\PYZca{}}\PY{l+s+sr}{a}\PY{l+s+sr}{\PYZhy{}}\PY{l+s+sr}{z}\PY{l+s+sr}{A}\PY{l+s+sr}{\PYZhy{}}\PY{l+s+sr}{Z}\PY{l+s+sr}{0}\PY{l+s+sr}{\PYZhy{}}\PY{l+s+sr}{9}\PY{l+s+sr}{]}\PY{l+s+sr}{\PYZdq{}}\PY{p}{,}\PY{n}{str}\PY{p}{)}\PY{p}{)}
\end{Verbatim}

\begin{Verbatim}[commandchars=\\\{\}]
{\color{outcolor}Out[{\color{outcolor}51}]:} 4-element Array\{RegexMatch,1\}:
          RegexMatch(" ")
          RegexMatch(" ")
          RegexMatch(" ")
          RegexMatch("!")
\end{Verbatim}
            
    \hypertarget{tuples}{%
\subsubsection{Tuples}\label{tuples}}

Tuples are a Julia type. They are an abstraction of function arguments
without the function. The arguments are defined in a specific order and
have well-defined types. Tuples can have any number of parameters, and
they do not have field names. Fields are accessed by their index, and
turples are defined using brackets \texttt{()} and commas. A very
usefual feature of tuples in Julia is that each element of a tuples in
Julia is that each element of a tuple can have its own type. Variable
values can be assigned directly from atuple where the value of each
variable corresponds to a value in the tuple.

    \begin{Verbatim}[commandchars=\\\{\}]
{\color{incolor}In [{\color{incolor}52}]:} \PY{c}{\PYZsh{} A tuple comprising only floats}
         \PY{n}{tup1} \PY{o}{=} \PY{p}{(}\PY{l+m+mf}{3.0}\PY{p}{,}\PY{l+m+mf}{0.1}\PY{p}{,}\PY{l+m+mf}{0.8}\PY{p}{,}\PY{l+m+mf}{1.9}\PY{p}{)}
         \PY{n}{typeof}\PY{p}{(}\PY{n}{tup1}\PY{p}{)}
\end{Verbatim}

\begin{Verbatim}[commandchars=\\\{\}]
{\color{outcolor}Out[{\color{outcolor}52}]:} NTuple\{4,Float64\}
\end{Verbatim}
            
    \begin{Verbatim}[commandchars=\\\{\}]
{\color{incolor}In [{\color{incolor}53}]:} \PY{c}{\PYZsh{}\PYZsh{} A tuple comprising strings and floats}
         \PY{n}{tup2} \PY{o}{=} \PY{p}{(}\PY{l+s}{\PYZdq{}}\PY{l+s}{D}\PY{l+s}{a}\PY{l+s}{t}\PY{l+s}{a}\PY{l+s}{\PYZdq{}}\PY{p}{,}\PY{l+m+mf}{2.5}\PY{p}{,}\PY{l+s}{\PYZdq{}}\PY{l+s}{S}\PY{l+s}{c}\PY{l+s}{i}\PY{l+s}{e}\PY{l+s}{n}\PY{l+s}{c}\PY{l+s}{e}\PY{l+s}{\PYZdq{}}\PY{p}{,} \PY{l+m+mf}{8.8}\PY{p}{)}
         \PY{n}{typeof}\PY{p}{(}\PY{n}{tup2}\PY{p}{)}
\end{Verbatim}

\begin{Verbatim}[commandchars=\\\{\}]
{\color{outcolor}Out[{\color{outcolor}53}]:} Tuple\{String,Float64,String,Float64\}
\end{Verbatim}
            
    \begin{Verbatim}[commandchars=\\\{\}]
{\color{incolor}In [{\color{incolor}54}]:} \PY{c}{\PYZsh{}\PYZsh{} variable assignment}
         \PY{n}{a}\PY{p}{,}\PY{n}{b}\PY{p}{,}\PY{n}{c} \PY{o}{=} \PY{p}{(}\PY{l+s}{\PYZdq{}}\PY{l+s}{F}\PY{l+s}{a}\PY{l+s}{s}\PY{l+s}{t}\PY{l+s}{\PYZdq{}}\PY{p}{,} \PY{l+m+mi}{1}\PY{p}{,} \PY{l+m+mf}{5.2}\PY{p}{)}
         \PY{n}{a}
\end{Verbatim}

\begin{Verbatim}[commandchars=\\\{\}]
{\color{outcolor}Out[{\color{outcolor}54}]:} "Fast"
\end{Verbatim}
            
    \begin{Verbatim}[commandchars=\\\{\}]
{\color{incolor}In [{\color{incolor}55}]:} \PY{n}{b}
\end{Verbatim}

\begin{Verbatim}[commandchars=\\\{\}]
{\color{outcolor}Out[{\color{outcolor}55}]:} 1
\end{Verbatim}
            
    \begin{Verbatim}[commandchars=\\\{\}]
{\color{incolor}In [{\color{incolor}56}]:} \PY{n}{c}
\end{Verbatim}

\begin{Verbatim}[commandchars=\\\{\}]
{\color{outcolor}Out[{\color{outcolor}56}]:} 5.2
\end{Verbatim}
            
    \begin{Verbatim}[commandchars=\\\{\}]
{\color{incolor}In [{\color{incolor}57}]:} \PY{p}{(}\PY{n}{a}\PY{p}{,}\PY{n}{b}\PY{p}{,}\PY{n}{c}\PY{p}{)}\PY{o}{=}\PY{p}{(}\PY{l+s}{\PYZdq{}}\PY{l+s}{s}\PY{l+s}{a}\PY{l+s}{f}\PY{l+s}{a}\PY{l+s}{\PYZdq{}}\PY{p}{,}\PY{l+m+mi}{2341}\PY{p}{,}\PY{l+m+mf}{123.412}\PY{p}{)}
         \PY{n}{typeof}\PY{p}{(}\PY{n}{a}\PY{p}{)}
\end{Verbatim}

\begin{Verbatim}[commandchars=\\\{\}]
{\color{outcolor}Out[{\color{outcolor}57}]:} String
\end{Verbatim}
            
    \hypertarget{data-structures}{%
\subsection{Data Structures}\label{data-structures}}

\hypertarget{arrays}{%
\subsubsection{Arrays}\label{arrays}}

An array is a multidimensional grid that stores objects of any type. To
improve preformance, arrays should contain only one specific type, e.g.,
\texttt{Int}. Arrays do not require vectroizing for fast array
computations. The array implementation used by Julia is written in Julia
and relies on the comiler for performace. The compiler uses type
inference to make optimized code for array indexing, which makes
programs more readable and easier to maintain. Arrays are a subtype of
the \texttt{AbstractArray} type. As such, they are laid out as a
contiguous block of memory. this is not true of other members of the
\texttt{AbstractArray} type, such as \texttt{SparseMatrixCSC} and
\texttt{SubArray}.

The type and dimensions of an array can be specified using
\texttt{Array\{T\}(D)}, where \texttt{T} is any valid Julia type and
\texttt{D} is the dimension of the array. The first entry in the tuple
\texttt{D} is a singleton that specifies how the array values are
initialiezed. Users can specify \texttt{undef} to create an
uninitialized array, \texttt{nothing} to create arrays with no value, or
\texttt{missing} to create arrays of missing values. Arrays with
different types can be created with type \texttt{Any}.

    \begin{Verbatim}[commandchars=\\\{\}]
{\color{incolor}In [{\color{incolor}58}]:} \PY{c}{\PYZsh{} A vector of length 5 containing integers}
         \PY{n}{a1} \PY{o}{=} \PY{k+kt}{Array}\PY{p}{\PYZob{}}\PY{k+kt}{Int64}\PY{p}{\PYZcb{}}\PY{p}{(}\PY{n}{undef}\PY{p}{,}\PY{l+m+mi}{5}\PY{p}{)}
         \PY{n}{typeof}\PY{p}{(}\PY{n}{a1}\PY{p}{)}
\end{Verbatim}

\begin{Verbatim}[commandchars=\\\{\}]
{\color{outcolor}Out[{\color{outcolor}58}]:} Array\{Int64,1\}
\end{Verbatim}
            
    \begin{Verbatim}[commandchars=\\\{\}]
{\color{incolor}In [{\color{incolor}59}]:} \PY{c}{\PYZsh{} A 2\PYZbs{}times2 matrix containting integers}
         \PY{n}{a2} \PY{o}{=} \PY{k+kt}{Array}\PY{p}{\PYZob{}}\PY{k+kt}{Int64}\PY{p}{\PYZcb{}}\PY{p}{(}\PY{n}{undef}\PY{p}{,}\PY{p}{(}\PY{l+m+mi}{2}\PY{p}{,}\PY{l+m+mi}{2}\PY{p}{)}\PY{p}{)}
\end{Verbatim}

\begin{Verbatim}[commandchars=\\\{\}]
{\color{outcolor}Out[{\color{outcolor}59}]:} 2×2 Array\{Int64,2\}:
          296955136  296955296
          296955280  112897568
\end{Verbatim}
            
    \begin{Verbatim}[commandchars=\\\{\}]
{\color{incolor}In [{\color{incolor}60}]:} \PY{c}{\PYZsh{} A 2\PYZbs{}times 2 matrix containing Any type}
         \PY{n}{a2} \PY{o}{=} \PY{k+kt}{Array}\PY{p}{\PYZob{}}\PY{k+kt}{Any}\PY{p}{\PYZcb{}}\PY{p}{(}\PY{n}{undef}\PY{p}{,}\PY{p}{(}\PY{l+m+mi}{2}\PY{p}{,}\PY{l+m+mi}{2}\PY{p}{)}\PY{p}{)}
\end{Verbatim}

\begin{Verbatim}[commandchars=\\\{\}]
{\color{outcolor}Out[{\color{outcolor}60}]:} 2×2 Array\{Any,2\}:
          \#undef  \#undef
          \#undef  \#undef
\end{Verbatim}
            
    In Julia, \texttt{{[}{]}} can also be used to generate arrays. In fact,
the \texttt{Vector()}, \texttt{Matrix()} and \texttt{collect()}
functions can also be used

    \begin{Verbatim}[commandchars=\\\{\}]
{\color{incolor}In [{\color{incolor}61}]:} \PY{c}{\PYZsh{}\PYZsh{} A three\PYZhy{}element row \PYZdq{}Vector\PYZdq{}}
         \PY{n}{a4} \PY{o}{=} \PY{p}{[}\PY{l+m+mi}{1}\PY{p}{,}\PY{l+m+mi}{2}\PY{p}{,}\PY{l+m+mi}{3}\PY{p}{]}
         \PY{n}{typeof}\PY{p}{(}\PY{n}{a4}\PY{p}{)}
\end{Verbatim}

\begin{Verbatim}[commandchars=\\\{\}]
{\color{outcolor}Out[{\color{outcolor}61}]:} Array\{Int64,1\}
\end{Verbatim}
            
    The array \texttt{a4} above does not have a second dimension, i.e., it
is neither a \(1\times3\) vector nor a \(3\times1\) vector. In other
words, Julia makes a distinction between \texttt{Array\{T,1\}} and
\texttt{Array\{T,2\}}.

    \begin{Verbatim}[commandchars=\\\{\}]
{\color{incolor}In [{\color{incolor}62}]:} \PY{c}{\PYZsh{}\PYZsh{} A 1\PYZbs{}times 3 colum vector  \PYZhy{}\PYZhy{} a two\PYZhy{}dimensional array}
         \PY{n}{a5} \PY{o}{=} \PY{p}{[}\PY{l+m+mi}{1} \PY{l+m+mi}{2} \PY{l+m+mi}{3}\PY{p}{]}
         \PY{n}{typeof}\PY{p}{(}\PY{n}{a5}\PY{p}{)}
\end{Verbatim}

\begin{Verbatim}[commandchars=\\\{\}]
{\color{outcolor}Out[{\color{outcolor}62}]:} Array\{Int64,2\}
\end{Verbatim}
            
    \begin{Verbatim}[commandchars=\\\{\}]
{\color{incolor}In [{\color{incolor}63}]:} \PY{c}{\PYZsh{}\PYZsh{} A 2\PYZbs{}times 3 matrix, where ; is used to separate rows}
         \PY{n}{a6} \PY{o}{=} \PY{p}{[}\PY{l+m+mi}{80} \PY{l+m+mi}{81} \PY{l+m+mi}{82} \PY{p}{;} \PY{l+m+mi}{90} \PY{l+m+mi}{91} \PY{l+m+mi}{92}\PY{p}{]}
         \PY{n}{typeof}\PY{p}{(}\PY{n}{a6}\PY{p}{)}
\end{Verbatim}

\begin{Verbatim}[commandchars=\\\{\}]
{\color{outcolor}Out[{\color{outcolor}63}]:} Array\{Int64,2\}
\end{Verbatim}
            
    \begin{Verbatim}[commandchars=\\\{\}]
{\color{incolor}In [{\color{incolor}64}]:} \PY{c}{\PYZsh{}\PYZsh{} Arrays containing elements of a specific type can be constructed like:}
         \PY{n}{a7} \PY{o}{=} \PY{k+kt}{Float64}\PY{p}{[}\PY{l+m+mf}{3.0} \PY{l+m+mf}{5.0} \PY{p}{;} \PY{l+m+mf}{1.1} \PY{l+m+mf}{3.5}\PY{p}{]}
\end{Verbatim}

\begin{Verbatim}[commandchars=\\\{\}]
{\color{outcolor}Out[{\color{outcolor}64}]:} 2×2 Array\{Float64,2\}:
          3.0  5.0
          1.1  3.5
\end{Verbatim}
            
    \begin{Verbatim}[commandchars=\\\{\}]
{\color{incolor}In [{\color{incolor}65}]:} \PY{c}{\PYZsh{}\PYZsh{} Arrays can be explicitly created like this:}
         \PY{k+kt}{Vector}\PY{p}{(}\PY{n}{undef}\PY{p}{,}\PY{l+m+mi}{3}\PY{p}{)}
\end{Verbatim}

\begin{Verbatim}[commandchars=\\\{\}]
{\color{outcolor}Out[{\color{outcolor}65}]:} 3-element Array\{Any,1\}:
          \#undef
          \#undef
          \#undef
\end{Verbatim}
            
    \begin{Verbatim}[commandchars=\\\{\}]
{\color{incolor}In [{\color{incolor}66}]:} \PY{k+kt}{Matrix}\PY{p}{(}\PY{n}{undef}\PY{p}{,} \PY{l+m+mi}{2}\PY{p}{,}\PY{l+m+mi}{2}\PY{p}{)}
\end{Verbatim}

\begin{Verbatim}[commandchars=\\\{\}]
{\color{outcolor}Out[{\color{outcolor}66}]:} 2×2 Array\{Any,2\}:
          \#undef  \#undef
          \#undef  \#undef
\end{Verbatim}
            
    \begin{Verbatim}[commandchars=\\\{\}]
{\color{incolor}In [{\color{incolor}67}]:} \PY{c}{\PYZsh{} A 3\PYZhy{}element Float array}
         \PY{n}{a3} \PY{o}{=} \PY{n}{collect}\PY{p}{(}\PY{k+kt}{Float64}\PY{p}{,} \PY{l+m+mi}{3}\PY{o}{:}\PY{o}{\PYZhy{}}\PY{l+m+mi}{1}\PY{o}{:}\PY{l+m+mi}{1}\PY{p}{)}
\end{Verbatim}

\begin{Verbatim}[commandchars=\\\{\}]
{\color{outcolor}Out[{\color{outcolor}67}]:} 3-element Array\{Float64,1\}:
          3.0
          2.0
          1.0
\end{Verbatim}
            
    Julia has many bulit-in functions that generate specific kinds of
arrays. Here are some useful ones:

\begin{itemize}
\tightlist
\item
  \texttt{zeros(T,\ d1,\ ..)} is a \texttt{d1}-dimensional array of all
  zeros.
\item
  \texttt{ones(T,\ d1,\ ..)} is a \texttt{d1}-dimensional array of all
  ones.
\item
  \texttt{rand(T,\ d1,\ ..)}: if \texttt{T} is Float a
  \texttt{d1}-dimensional array of random numbers between 0 and 1 is
  returned; if an array is specified as the first argument, \texttt{d1}
  random elements from the array are returned.
\item
  \texttt{randn(T,\ d1,\ ..)} is a \texttt{d1}-dimensional array of
  random numbers from the standard normal distribution with mean zero
  and standard deviation 1.
\item
  \texttt{MatrixT(I,\ (n,n))} is the \texttt{n}\(\times\)\texttt{n}
  identity matrix. The identity operator \texttt{I} is available in the
  `LinearAlgebra.jl' package.
\item
  \texttt{fill!(A,\ x)} is the array \texttt{A} filled with value
  \texttt{x}.
\end{itemize}

Note that, in the above, \texttt{d1} can be a tuple specifying multiple
dimensions.

Arrays can easily be concatenated in Julia. There are two functions
commonly used to concatenate arrays:

\begin{itemize}
\tightlist
\item
  \texttt{vcat(A1,\ A2,\ ..)} concatenates arrays vertically, i.e.,
  stacks \texttt{A1} on top of \texttt{A2}.
\item
  \texttt{hcat(A1,\ A2,\ ..)} concatenates arrays horizontally, i.e.,
  adds \texttt{A2} to the right of \texttt{A1}.
\end{itemize}

Of course, concatenations requires that the relevant dimensions match.

The following code blocks illustrates some useful array functions as we
as slicing. Slicing for arrays works similarly to slicing for strings.

    \begin{Verbatim}[commandchars=\\\{\}]
{\color{incolor}In [{\color{incolor}68}]:} \PY{c}{\PYZsh{}\PYZsh{} Create a 2x2 identity matrix}
         \PY{k}{using} \PY{n}{LinearAlgebra}
         \PY{n}{imat} \PY{o}{=} \PY{k+kt}{Matrix}\PY{p}{\PYZob{}}\PY{k+kt}{Int8}\PY{p}{\PYZcb{}}\PY{p}{(}\PY{n+nb}{I}\PY{p}{,} \PY{p}{(}\PY{l+m+mi}{2}\PY{p}{,}\PY{l+m+mi}{2}\PY{p}{)}\PY{p}{)}
\end{Verbatim}

\begin{Verbatim}[commandchars=\\\{\}]
{\color{outcolor}Out[{\color{outcolor}68}]:} 2×2 Array\{Int8,2\}:
          1  0
          0  1
\end{Verbatim}
            
    \begin{Verbatim}[commandchars=\\\{\}]
{\color{incolor}In [{\color{incolor}69}]:} \PY{c}{\PYZsh{}\PYZsh{} return rando numbers between 0 and 1}
         \PY{n}{rand}\PY{p}{(}\PY{l+m+mi}{2}\PY{p}{)}
\end{Verbatim}

\begin{Verbatim}[commandchars=\\\{\}]
{\color{outcolor}Out[{\color{outcolor}69}]:} 2-element Array\{Float64,1\}:
          0.9011806932440589
          0.7846081397345164
\end{Verbatim}
            
    \begin{Verbatim}[commandchars=\\\{\}]
{\color{incolor}In [{\color{incolor}70}]:} \PY{n}{B} \PY{o}{=} \PY{p}{[}\PY{l+m+mi}{80} \PY{l+m+mi}{81} \PY{l+m+mi}{82} \PY{p}{;} \PY{l+m+mi}{90} \PY{l+m+mi}{91} \PY{l+m+mi}{92}\PY{p}{]}
\end{Verbatim}

\begin{Verbatim}[commandchars=\\\{\}]
{\color{outcolor}Out[{\color{outcolor}70}]:} 2×3 Array\{Int64,2\}:
          80  81  82
          90  91  92
\end{Verbatim}
            
    \begin{Verbatim}[commandchars=\\\{\}]
{\color{incolor}In [{\color{incolor}71}]:} \PY{n}{rand}\PY{p}{(}\PY{n}{B}\PY{p}{,}\PY{l+m+mi}{2}\PY{p}{)}
\end{Verbatim}

\begin{Verbatim}[commandchars=\\\{\}]
{\color{outcolor}Out[{\color{outcolor}71}]:} 2-element Array\{Int64,1\}:
          90
          82
\end{Verbatim}
            
    \begin{Verbatim}[commandchars=\\\{\}]
{\color{incolor}In [{\color{incolor}72}]:} \PY{c}{\PYZsh{}\PYZsh{} The number of elements in B}
         \PY{n}{length}\PY{p}{(}\PY{n}{B}\PY{p}{)}
\end{Verbatim}

\begin{Verbatim}[commandchars=\\\{\}]
{\color{outcolor}Out[{\color{outcolor}72}]:} 6
\end{Verbatim}
            
    \begin{Verbatim}[commandchars=\\\{\}]
{\color{incolor}In [{\color{incolor}73}]:} \PY{c}{\PYZsh{}\PYZsh{} The dimensions of B}
         \PY{n}{size}\PY{p}{(}\PY{n}{B}\PY{p}{)}
\end{Verbatim}

\begin{Verbatim}[commandchars=\\\{\}]
{\color{outcolor}Out[{\color{outcolor}73}]:} (2, 3)
\end{Verbatim}
            
    \begin{Verbatim}[commandchars=\\\{\}]
{\color{incolor}In [{\color{incolor}74}]:} \PY{c}{\PYZsh{}\PYZsh{} the number of dimensions of B}
         \PY{n}{ndims}\PY{p}{(}\PY{n}{B}\PY{p}{)}
\end{Verbatim}

\begin{Verbatim}[commandchars=\\\{\}]
{\color{outcolor}Out[{\color{outcolor}74}]:} 2
\end{Verbatim}
            
    \begin{Verbatim}[commandchars=\\\{\}]
{\color{incolor}In [{\color{incolor}75}]:} \PY{c}{\PYZsh{}\PYZsh{} A new array with the same elements (data) as B but different dimensions }
         \PY{n}{reshape}\PY{p}{(}\PY{n}{B}\PY{p}{,} \PY{p}{(}\PY{l+m+mi}{3}\PY{p}{,}\PY{l+m+mi}{2}\PY{p}{)}\PY{p}{)}
\end{Verbatim}

\begin{Verbatim}[commandchars=\\\{\}]
{\color{outcolor}Out[{\color{outcolor}75}]:} 3×2 Array\{Int64,2\}:
          80  91
          90  82
          81  92
\end{Verbatim}
            
    \begin{Verbatim}[commandchars=\\\{\}]
{\color{incolor}In [{\color{incolor}76}]:} \PY{n}{ndims}\PY{p}{(}\PY{n}{B}\PY{p}{)}
\end{Verbatim}

\begin{Verbatim}[commandchars=\\\{\}]
{\color{outcolor}Out[{\color{outcolor}76}]:} 2
\end{Verbatim}
            
    \begin{Verbatim}[commandchars=\\\{\}]
{\color{incolor}In [{\color{incolor}77}]:} \PY{c}{\PYZsh{}\PYZsh{} A copy of B, where elements are recursively copied}
         \PY{n}{B2} \PY{o}{=} \PY{n}{deepcopy}\PY{p}{(}\PY{n}{B}\PY{p}{)}
\end{Verbatim}

\begin{Verbatim}[commandchars=\\\{\}]
{\color{outcolor}Out[{\color{outcolor}77}]:} 2×3 Array\{Int64,2\}:
          80  81  82
          90  91  92
\end{Verbatim}
            
    \begin{Verbatim}[commandchars=\\\{\}]
{\color{incolor}In [{\color{incolor}78}]:} \PY{c}{\PYZsh{}\PYZsh{} When slicing, a slice is specified for each dimension}
         \PY{c}{\PYZsh{}\PYZsh{} The first two rows of the first column done two ways}
         \PY{n}{B}\PY{p}{[}\PY{l+m+mi}{1}\PY{o}{:}\PY{l+m+mi}{2}\PY{p}{,} \PY{p}{]}
\end{Verbatim}

\begin{Verbatim}[commandchars=\\\{\}]
{\color{outcolor}Out[{\color{outcolor}78}]:} 2-element Array\{Int64,1\}:
          80
          90
\end{Verbatim}
            
    \begin{Verbatim}[commandchars=\\\{\}]
{\color{incolor}In [{\color{incolor}79}]:} \PY{n}{B}\PY{p}{[}\PY{l+m+mi}{1}\PY{o}{:}\PY{l+m+mi}{2}\PY{p}{,}\PY{l+m+mi}{1}\PY{p}{]}
\end{Verbatim}

\begin{Verbatim}[commandchars=\\\{\}]
{\color{outcolor}Out[{\color{outcolor}79}]:} 2-element Array\{Int64,1\}:
          80
          90
\end{Verbatim}
            
    \begin{Verbatim}[commandchars=\\\{\}]
{\color{incolor}In [{\color{incolor}80}]:} \PY{c}{\PYZsh{}\PYZsh{} The first two rows of the second column}
         \PY{n}{B}\PY{p}{[}\PY{l+m+mi}{1}\PY{o}{:}\PY{l+m+mi}{2}\PY{p}{,}\PY{l+m+mi}{2}\PY{p}{]}
\end{Verbatim}

\begin{Verbatim}[commandchars=\\\{\}]
{\color{outcolor}Out[{\color{outcolor}80}]:} 2-element Array\{Int64,1\}:
          81
          91
\end{Verbatim}
            
    \begin{Verbatim}[commandchars=\\\{\}]
{\color{incolor}In [{\color{incolor}81}]:} \PY{c}{\PYZsh{} The first row}
         \PY{n}{B}\PY{p}{[}\PY{l+m+mi}{1}\PY{p}{,}\PY{o}{:}\PY{p}{]}
\end{Verbatim}

\begin{Verbatim}[commandchars=\\\{\}]
{\color{outcolor}Out[{\color{outcolor}81}]:} 3-element Array\{Int64,1\}:
          80
          81
          82
\end{Verbatim}
            
    \begin{Verbatim}[commandchars=\\\{\}]
{\color{incolor}In [{\color{incolor}82}]:} \PY{c}{\PYZsh{}\PYZsh{} The third element}
         \PY{n}{B}\PY{p}{[}\PY{l+m+mi}{3}\PY{p}{]}
\end{Verbatim}

\begin{Verbatim}[commandchars=\\\{\}]
{\color{outcolor}Out[{\color{outcolor}82}]:} 81
\end{Verbatim}
            
    \begin{Verbatim}[commandchars=\\\{\}]
{\color{incolor}In [{\color{incolor}83}]:} \PY{c}{\PYZsh{} Another way to build an array is using comprehensions}
         \PY{n}{A1} \PY{o}{=} \PY{p}{[}\PY{n}{sqrt}\PY{p}{(}\PY{n}{i}\PY{p}{)} \PY{k}{for} \PY{n}{i} \PY{k+kp}{in} \PY{p}{[}\PY{l+m+mi}{16}\PY{p}{,}\PY{l+m+mi}{25}\PY{p}{,}\PY{l+m+mi}{64}\PY{p}{]}\PY{p}{]}
\end{Verbatim}

\begin{Verbatim}[commandchars=\\\{\}]
{\color{outcolor}Out[{\color{outcolor}83}]:} 3-element Array\{Float64,1\}:
          4.0
          5.0
          8.0
\end{Verbatim}
            
    \begin{Verbatim}[commandchars=\\\{\}]
{\color{incolor}In [{\color{incolor}84}]:} \PY{n}{A2} \PY{o}{=} \PY{p}{[}\PY{n}{i}\PY{o}{\PYZca{}}\PY{l+m+mi}{2} \PY{k}{for} \PY{n}{i} \PY{k+kp}{in} \PY{p}{[}\PY{l+m+mi}{1}\PY{p}{,}\PY{l+m+mi}{2}\PY{p}{,}\PY{l+m+mi}{3}\PY{p}{]}\PY{p}{]}
\end{Verbatim}

\begin{Verbatim}[commandchars=\\\{\}]
{\color{outcolor}Out[{\color{outcolor}84}]:} 3-element Array\{Int64,1\}:
          1
          4
          9
\end{Verbatim}
            
    From a couple of examples in the above code block, we can see that Julia
counts array elements by column, i.e., the \(k\)th element of the
\(n\times m\) matrix \(X\) is the \(k\)th element of the \(nm\)-vector
vec(\(X\)). Array comprehensions, illistrated above, are another more
sophisticated way of building arrays. The generate the items in the
array with a function and a loop. These items are then collected into an
array by the brackets \texttt{{[}{]}} that surround the loop and
function.

    \hypertarget{dictionaries}{%
\subsubsection{Dictionaries}\label{dictionaries}}

In Julia, dictionaries are defined as associative collections consisting
of a key value pair, i.e., the key is associated with a specific value.
These key-value pairs have their own type in Julia,
\texttt{Pairtypeof(Key)}, \texttt{typeof(value)} which creates a
\texttt{Pair} object. Alternatively, the \texttt{=\textgreater{}} symbol
can be used to separate the key and value to create the same
\texttt{Pair} object. One use of \texttt{Pair} objects is in the
instantiation of dictionaries. Dictionaries in Julia can be used
analogously to lists in \emph{R}. Dictionaries are created using the
keyword \texttt{Dict} and types can be specified for both the key and
the value. The keys are hashed and are always unique.

    \begin{Verbatim}[commandchars=\\\{\}]
{\color{incolor}In [{\color{incolor}85}]:} \PY{c}{\PYZsh{}\PYZsh{} Three dictionaries, D0 is empty, D1 and D1 are the same}
         \PY{n}{D0} \PY{o}{=} \PY{k+kt}{Dict}\PY{p}{(}\PY{p}{)}
         \PY{n}{D1} \PY{o}{=} \PY{k+kt}{Dict}\PY{p}{(}\PY{l+m+mi}{1} \PY{o}{=\PYZgt{}} \PY{l+s}{\PYZdq{}}\PY{l+s}{r}\PY{l+s}{e}\PY{l+s}{d}\PY{l+s}{\PYZdq{}}\PY{p}{,} \PY{l+m+mi}{2} \PY{o}{=\PYZgt{}} \PY{l+s}{\PYZdq{}}\PY{l+s}{w}\PY{l+s}{h}\PY{l+s}{i}\PY{l+s}{t}\PY{l+s}{e}\PY{l+s}{\PYZdq{}}\PY{p}{)}
         \PY{n}{D2} \PY{o}{=} \PY{k+kt}{Dict}\PY{p}{\PYZob{}}\PY{k+kt}{Integer}\PY{p}{,} \PY{n}{String}\PY{p}{\PYZcb{}}\PY{p}{(}\PY{l+m+mi}{1} \PY{o}{=\PYZgt{}} \PY{l+s}{\PYZdq{}}\PY{l+s}{r}\PY{l+s}{e}\PY{l+s}{d}\PY{l+s}{\PYZdq{}}\PY{p}{,} \PY{l+m+mi}{2} \PY{o}{=\PYZgt{}} \PY{l+s}{\PYZdq{}}\PY{l+s}{w}\PY{l+s}{h}\PY{l+s}{i}\PY{l+s}{t}\PY{l+s}{e}\PY{l+s}{\PYZdq{}}\PY{p}{)}
         \PY{n}{typeof}\PY{p}{(}\PY{n}{D2}\PY{p}{)}
\end{Verbatim}

\begin{Verbatim}[commandchars=\\\{\}]
{\color{outcolor}Out[{\color{outcolor}85}]:} Dict\{Integer,String\}
\end{Verbatim}
            
    \begin{Verbatim}[commandchars=\\\{\}]
{\color{incolor}In [{\color{incolor}86}]:} \PY{c}{\PYZsh{}\PYZsh{} Dictionaries can be created using a loop}
         \PY{n}{food} \PY{o}{=} \PY{p}{[}\PY{l+s}{\PYZdq{}}\PY{l+s}{s}\PY{l+s}{a}\PY{l+s}{l}\PY{l+s}{m}\PY{l+s}{o}\PY{l+s}{n}\PY{l+s}{\PYZdq{}}\PY{p}{,} \PY{l+s}{\PYZdq{}}\PY{l+s}{m}\PY{l+s}{a}\PY{l+s}{p}\PY{l+s}{l}\PY{l+s}{e}\PY{l+s}{ }\PY{l+s}{s}\PY{l+s}{y}\PY{l+s}{r}\PY{l+s}{u}\PY{l+s}{p}\PY{l+s}{\PYZdq{}}\PY{p}{,} \PY{l+s}{\PYZdq{}}\PY{l+s}{t}\PY{l+s}{o}\PY{l+s}{u}\PY{l+s}{r}\PY{l+s}{i}\PY{l+s}{e}\PY{l+s}{r}\PY{l+s}{e}\PY{l+s}{\PYZdq{}}\PY{p}{]}
         
         \PY{n}{food\PYZus{}dict} \PY{o}{=} \PY{k+kt}{Dict}\PY{p}{\PYZob{}}\PY{k+kt}{Int}\PY{p}{,} \PY{n}{String}\PY{p}{\PYZcb{}}\PY{p}{(}\PY{p}{)}
         
         \PY{c}{\PYZsh{}\PYZsh{} keys are the foods index in the array}
         \PY{k}{for} \PY{p}{(}\PY{n}{n}\PY{p}{,} \PY{n}{fd}\PY{p}{)} \PY{k+kp}{in} \PY{n}{enumerate}\PY{p}{(}\PY{n}{food}\PY{p}{)}
             \PY{n}{food\PYZus{}dict}\PY{p}{[}\PY{n}{n}\PY{p}{]} \PY{o}{=} \PY{n}{fd}
         \PY{k}{end}
         \PY{n}{food\PYZus{}dict}
\end{Verbatim}

\begin{Verbatim}[commandchars=\\\{\}]
{\color{outcolor}Out[{\color{outcolor}86}]:} Dict\{Int64,String\} with 3 entries:
           2 => "maple syrup"
           3 => "touriere"
           1 => "salmon"
\end{Verbatim}
            
    \begin{Verbatim}[commandchars=\\\{\}]
{\color{incolor}In [{\color{incolor}87}]:} \PY{c}{\PYZsh{}\PYZsh{} Dictionaries can also be created using the generator syntax}
         \PY{n}{wine} \PY{o}{=} \PY{p}{[}\PY{l+s}{\PYZdq{}}\PY{l+s}{r}\PY{l+s}{e}\PY{l+s}{d}\PY{l+s}{\PYZdq{}}\PY{p}{,} \PY{l+s}{\PYZdq{}}\PY{l+s}{w}\PY{l+s}{h}\PY{l+s}{i}\PY{l+s}{t}\PY{l+s}{e}\PY{l+s}{\PYZdq{}}\PY{p}{,} \PY{l+s}{\PYZdq{}}\PY{l+s}{r}\PY{l+s}{o}\PY{l+s}{s}\PY{l+s}{e}\PY{l+s}{\PYZdq{}}\PY{p}{]}
         \PY{n}{wine\PYZus{}dict} \PY{o}{=} \PY{k+kt}{Dict}\PY{p}{\PYZob{}}\PY{k+kt}{Int}\PY{p}{,} \PY{n}{String}\PY{p}{\PYZcb{}}\PY{p}{(}\PY{n}{i} \PY{o}{=\PYZgt{}} \PY{n}{wine}\PY{p}{[}\PY{n}{i}\PY{p}{]} \PY{k}{for} \PY{n}{i} \PY{k+kp}{in} \PY{l+m+mi}{1}\PY{o}{:}\PY{n}{length}\PY{p}{(}\PY{n}{wine}\PY{p}{)}\PY{p}{)}
\end{Verbatim}

\begin{Verbatim}[commandchars=\\\{\}]
{\color{outcolor}Out[{\color{outcolor}87}]:} Dict\{Int64,String\} with 3 entries:
           2 => "white"
           3 => "rose"
           1 => "red"
\end{Verbatim}
            
    Values can be accessed using \texttt{{[}{]}} with a value of dictionary
key inserted between them or \texttt{get()}. The presence of a key can
be checked using \texttt{haskey()} and a particular key can be accessed
using \texttt{getkey()}. Keys can also be modified, as illustrated in
the below code block. Here, we also demonstrate adding and deleting
entries from a dictionary as well as various ways of manipulating keys
and values. Note that the following code block builds on the previous.

    \begin{Verbatim}[commandchars=\\\{\}]
{\color{incolor}In [{\color{incolor}88}]:} \PY{c}{\PYZsh{}\PYZsh{} Values can be accessed similarly to an array, but by key:}
         \PY{n}{food\PYZus{}dict}\PY{p}{[}\PY{l+m+mi}{1}\PY{p}{]}
\end{Verbatim}

\begin{Verbatim}[commandchars=\\\{\}]
{\color{outcolor}Out[{\color{outcolor}88}]:} "salmon"
\end{Verbatim}
            
    \begin{Verbatim}[commandchars=\\\{\}]
{\color{incolor}In [{\color{incolor}89}]:} \PY{c}{\PYZsh{}\PYZsh{} The get() function can alos be used; note that \PYZdq{}unknown\PYZdq{} is  }
         \PY{c}{\PYZsh{}\PYZsh{} the value returned here if the key is not in the dictionary}
         \PY{n}{get}\PY{p}{(}\PY{n}{food\PYZus{}dict}\PY{p}{,}\PY{l+m+mi}{1}\PY{p}{,}\PY{l+s}{\PYZdq{}}\PY{l+s}{u}\PY{l+s}{n}\PY{l+s}{k}\PY{l+s}{n}\PY{l+s}{o}\PY{l+s}{w}\PY{l+s}{n}\PY{l+s}{\PYZdq{}}\PY{p}{)}
\end{Verbatim}

\begin{Verbatim}[commandchars=\\\{\}]
{\color{outcolor}Out[{\color{outcolor}89}]:} "salmon"
\end{Verbatim}
            
    \begin{Verbatim}[commandchars=\\\{\}]
{\color{incolor}In [{\color{incolor}90}]:} \PY{n}{get}\PY{p}{(}\PY{n}{food\PYZus{}dict}\PY{p}{,}\PY{l+m+mi}{7}\PY{p}{,}\PY{l+s}{\PYZdq{}}\PY{l+s}{u}\PY{l+s}{n}\PY{l+s}{k}\PY{l+s}{n}\PY{l+s}{o}\PY{l+s}{w}\PY{l+s}{n}\PY{l+s}{\PYZdq{}}\PY{p}{)}
\end{Verbatim}

\begin{Verbatim}[commandchars=\\\{\}]
{\color{outcolor}Out[{\color{outcolor}90}]:} "unknown"
\end{Verbatim}
            
    \begin{Verbatim}[commandchars=\\\{\}]
{\color{incolor}In [{\color{incolor}91}]:} \PY{c}{\PYZsh{}\PYZsh{} we can also check directly for the presence of a particular key}
         \PY{n}{haskey}\PY{p}{(}\PY{n}{food\PYZus{}dict}\PY{p}{,}\PY{l+m+mi}{2}\PY{p}{)}
\end{Verbatim}

\begin{Verbatim}[commandchars=\\\{\}]
{\color{outcolor}Out[{\color{outcolor}91}]:} true
\end{Verbatim}
            
    \begin{Verbatim}[commandchars=\\\{\}]
{\color{incolor}In [{\color{incolor}92}]:} \PY{n}{haskey}\PY{p}{(}\PY{n}{food\PYZus{}dict}\PY{p}{,}\PY{l+m+mi}{9}\PY{p}{)}
\end{Verbatim}

\begin{Verbatim}[commandchars=\\\{\}]
{\color{outcolor}Out[{\color{outcolor}92}]:} false
\end{Verbatim}
            
    \begin{Verbatim}[commandchars=\\\{\}]
{\color{incolor}In [{\color{incolor}93}]:} \PY{c}{\PYZsh{}\PYZsh{} The getkey() function can also be used; note that 999 is the }
         \PY{c}{\PYZsh{}\PYZsh{} value returned here if the key is not in the dictionary}
         \PY{n}{getkey}\PY{p}{(}\PY{n}{food\PYZus{}dict}\PY{p}{,}\PY{l+m+mi}{1}\PY{p}{,}\PY{l+m+mi}{999}\PY{p}{)}
\end{Verbatim}

\begin{Verbatim}[commandchars=\\\{\}]
{\color{outcolor}Out[{\color{outcolor}93}]:} 1
\end{Verbatim}
            
    \begin{Verbatim}[commandchars=\\\{\}]
{\color{incolor}In [{\color{incolor}94}]:} \PY{c}{\PYZsh{}\PYZsh{} A new value can be associated with an existing key}
         \PY{n}{food\PYZus{}dict}\PY{p}{[}\PY{l+m+mi}{1}\PY{p}{]}
         \PY{n}{food\PYZus{}dict}\PY{p}{[}\PY{l+m+mi}{1}\PY{p}{]} \PY{o}{=} \PY{l+s}{\PYZdq{}}\PY{l+s}{l}\PY{l+s}{o}\PY{l+s}{b}\PY{l+s}{s}\PY{l+s}{t}\PY{l+s}{e}\PY{l+s}{r}\PY{l+s}{\PYZdq{}}
\end{Verbatim}

\begin{Verbatim}[commandchars=\\\{\}]
{\color{outcolor}Out[{\color{outcolor}94}]:} "lobster"
\end{Verbatim}
            
    \begin{Verbatim}[commandchars=\\\{\}]
{\color{incolor}In [{\color{incolor}95}]:} \PY{c}{\PYZsh{}\PYZsh{} Two common ways to add new entries:}
         \PY{n}{food\PYZus{}dict}\PY{p}{[}\PY{l+m+mi}{4}\PY{p}{]} \PY{o}{=} \PY{l+s}{\PYZdq{}}\PY{l+s}{b}\PY{l+s}{a}\PY{l+s}{n}\PY{l+s}{n}\PY{l+s}{o}\PY{l+s}{c}\PY{l+s}{k}\PY{l+s}{\PYZdq{}}
         \PY{n}{get!}\PY{p}{(}\PY{n}{food\PYZus{}dict}\PY{p}{,}\PY{l+m+mi}{4}\PY{p}{,}\PY{l+s}{\PYZdq{}}\PY{l+s}{m}\PY{l+s}{i}\PY{l+s}{s}\PY{l+s}{s}\PY{l+s}{\PYZdq{}}\PY{p}{)}
         \PY{n}{food\PYZus{}dict}\PY{p}{[}\PY{l+m+mi}{5}\PY{p}{]}
\end{Verbatim}

    \begin{Verbatim}[commandchars=\\\{\}]

        KeyError: key 5 not found

        

        Stacktrace:

         [1] getindex(::Dict\{Int64,String\}, ::Int64) at .\textbackslash{}dict.jl:478

         [2] top-level scope at In[95]:4

    \end{Verbatim}

    \begin{Verbatim}[commandchars=\\\{\}]
{\color{incolor}In [{\color{incolor}96}]:} \PY{c}{\PYZsh{}\PYZsh{} The advantage of get!() is that will not add the new entry if}
         \PY{c}{\PYZsh{}\PYZsh{} a value is already associated with the key}
         \PY{n}{get!}\PY{p}{(}\PY{n}{food\PYZus{}dict}\PY{p}{,}\PY{l+m+mi}{4}\PY{p}{,}\PY{l+s}{\PYZdq{}}\PY{l+s}{f}\PY{l+s}{a}\PY{l+s}{k}\PY{l+s}{e}\PY{l+s}{\PYZdq{}}\PY{p}{)}
\end{Verbatim}

\begin{Verbatim}[commandchars=\\\{\}]
{\color{outcolor}Out[{\color{outcolor}96}]:} "bannock"
\end{Verbatim}
            
    \begin{Verbatim}[commandchars=\\\{\}]
{\color{incolor}In [{\color{incolor}97}]:} \PY{c}{\PYZsh{}\PYZsh{} Just deleting entries by key is straightforward}
         \PY{n}{delete!}\PY{p}{(}\PY{n}{food\PYZus{}dict}\PY{p}{,}\PY{l+m+mi}{4}\PY{p}{)}
\end{Verbatim}

\begin{Verbatim}[commandchars=\\\{\}]
{\color{outcolor}Out[{\color{outcolor}97}]:} Dict\{Int64,String\} with 3 entries:
           2 => "maple syrup"
           3 => "touriere"
           1 => "lobster"
\end{Verbatim}
            
    \begin{Verbatim}[commandchars=\\\{\}]
{\color{incolor}In [{\color{incolor}98}]:} \PY{c}{\PYZsh{}\PYZsh{} But we can also delete by key and return the value assocated with the key;}
         \PY{c}{\PYZsh{}\PYZsh{} note that 999 is reurned here if the key is not present}
         \PY{n}{deleted\PYZus{}fd\PYZus{}value} \PY{o}{=} \PY{n}{pop!}\PY{p}{(}\PY{n}{food\PYZus{}dict}\PY{p}{,}\PY{l+m+mi}{3}\PY{p}{,}\PY{l+m+mi}{999}\PY{p}{)}
         \PY{n}{food\PYZus{}dict}
\end{Verbatim}

\begin{Verbatim}[commandchars=\\\{\}]
{\color{outcolor}Out[{\color{outcolor}98}]:} Dict\{Int64,String\} with 2 entries:
           2 => "maple syrup"
           1 => "lobster"
\end{Verbatim}
            
    \begin{Verbatim}[commandchars=\\\{\}]
{\color{incolor}In [{\color{incolor}99}]:} \PY{c}{\PYZsh{}\PYZsh{} Keys can be coerced into arrays}
         \PY{n}{collect}\PY{p}{(}\PY{n}{keys}\PY{p}{(}\PY{n}{food\PYZus{}dict}\PY{p}{)}\PY{p}{)}
\end{Verbatim}

\begin{Verbatim}[commandchars=\\\{\}]
{\color{outcolor}Out[{\color{outcolor}99}]:} 2-element Array\{Int64,1\}:
          2
          1
\end{Verbatim}
            
    \begin{Verbatim}[commandchars=\\\{\}]
{\color{incolor}In [{\color{incolor}100}]:} \PY{n}{collect}\PY{p}{(}\PY{n}{values}\PY{p}{(}\PY{n}{food\PYZus{}dict}\PY{p}{)}\PY{p}{)}
\end{Verbatim}

\begin{Verbatim}[commandchars=\\\{\}]
{\color{outcolor}Out[{\color{outcolor}100}]:} 2-element Array\{String,1\}:
           "maple syrup"
           "lobster"    
\end{Verbatim}
            
    \begin{Verbatim}[commandchars=\\\{\}]
{\color{incolor}In [{\color{incolor}101}]:} \PY{c}{\PYZsh{}\PYZsh{} We can also iterate over both keys and values}
          \PY{k}{for} \PY{p}{(}\PY{n}{k}\PY{p}{,} \PY{n}{v}\PY{p}{)} \PY{k+kp}{in} \PY{n}{food\PYZus{}dict}
              \PY{n}{println}\PY{p}{(}\PY{l+s}{\PYZdq{}}\PY{l+s}{f}\PY{l+s}{o}\PY{l+s}{o}\PY{l+s}{d}\PY{l+s}{\PYZus{}}\PY{l+s}{d}\PY{l+s}{i}\PY{l+s}{c}\PY{l+s}{t}\PY{l+s}{:}\PY{l+s}{ }\PY{l+s}{k}\PY{l+s}{e}\PY{l+s}{y}\PY{l+s}{:}\PY{l+s}{ }\PY{l+s}{\PYZdq{}}\PY{p}{,} \PY{n}{k}\PY{p}{,} \PY{l+s}{\PYZdq{}}\PY{l+s}{v}\PY{l+s}{a}\PY{l+s}{l}\PY{l+s}{u}\PY{l+s}{e}\PY{l+s}{:}\PY{l+s}{ }\PY{l+s}{\PYZdq{}}\PY{p}{,} \PY{n}{v}\PY{p}{)}
          \PY{k}{end}
\end{Verbatim}

    \begin{Verbatim}[commandchars=\\\{\}]
food\_dict: key: 2value: maple syrup
food\_dict: key: 1value: lobster

    \end{Verbatim}

    \begin{Verbatim}[commandchars=\\\{\}]
{\color{incolor}In [{\color{incolor}102}]:} \PY{c}{\PYZsh{}\PYZsh{} We can also just loop over keys}
          \PY{k}{for} \PY{n}{k} \PY{k+kp}{in} \PY{n}{keys}\PY{p}{(}\PY{n}{food\PYZus{}dict}\PY{p}{)}
              \PY{n}{println}\PY{p}{(}\PY{l+s}{\PYZdq{}}\PY{l+s}{f}\PY{l+s}{o}\PY{l+s}{o}\PY{l+s}{d}\PY{l+s}{\PYZus{}}\PY{l+s}{d}\PY{l+s}{i}\PY{l+s}{c}\PY{l+s}{t}\PY{l+s}{:}\PY{l+s}{ }\PY{l+s}{k}\PY{l+s}{e}\PY{l+s}{y}\PY{l+s}{s}\PY{l+s}{:}\PY{l+s}{ }\PY{l+s}{\PYZdq{}}\PY{p}{,}\PY{n}{k}\PY{p}{)}
          \PY{k}{end}
\end{Verbatim}

    \begin{Verbatim}[commandchars=\\\{\}]
food\_dict: keys: 2
food\_dict: keys: 1

    \end{Verbatim}

    \begin{Verbatim}[commandchars=\\\{\}]
{\color{incolor}In [{\color{incolor}103}]:} \PY{c}{\PYZsh{}\PYZsh{} Or could also just loop over values}
          \PY{k}{for} \PY{n}{v} \PY{k+kp}{in} \PY{n}{values}\PY{p}{(}\PY{n}{food\PYZus{}dict}\PY{p}{)}
              \PY{n}{println}\PY{p}{(}\PY{l+s}{\PYZdq{}}\PY{l+s}{f}\PY{l+s}{o}\PY{l+s}{o}\PY{l+s}{d}\PY{l+s}{\PYZus{}}\PY{l+s}{d}\PY{l+s}{i}\PY{l+s}{c}\PY{l+s}{t}\PY{l+s}{:}\PY{l+s}{ }\PY{l+s}{v}\PY{l+s}{a}\PY{l+s}{l}\PY{l+s}{u}\PY{l+s}{e}\PY{l+s}{:}\PY{l+s}{\PYZdq{}}\PY{p}{,} \PY{n}{v}\PY{p}{)}
          \PY{k}{end}
\end{Verbatim}

    \begin{Verbatim}[commandchars=\\\{\}]
food\_dict: value:maple syrup
food\_dict: value:lobster

    \end{Verbatim}

    \hypertarget{control-flow}{%
\subsection{Control Flow}\label{control-flow}}

\hypertarget{compound-expressions}{%
\subsubsection{Compound Expressions}\label{compound-expressions}}

In Julia, a compound expression is one expression that is uesd to
sequentially evalute a group of subexpressions. The value of the last
subexpression is returned as the value of the expression. THere area two
to achieve this: \texttt{Begin} blocks and chains.

    \begin{Verbatim}[commandchars=\\\{\}]
{\color{incolor}In [{\color{incolor}104}]:} \PY{c}{\PYZsh{}\PYZsh{} A begin block}
          \PY{n}{b1} \PY{o}{=} \PY{k}{begin}
              \PY{n}{c} \PY{o}{=} \PY{l+m+mi}{20}
              \PY{n}{d} \PY{o}{=} \PY{l+m+mi}{5}
              \PY{n}{c} \PY{o}{*} \PY{n}{d}
          \PY{k}{end}
          \PY{n}{println}\PY{p}{(}\PY{l+s}{\PYZdq{}}\PY{l+s}{b}\PY{l+s}{1}\PY{l+s}{:}\PY{l+s}{ }\PY{l+s}{\PYZdq{}}\PY{p}{,} \PY{n}{b1}\PY{p}{)}
\end{Verbatim}

    \begin{Verbatim}[commandchars=\\\{\}]
b1: 100

    \end{Verbatim}

    \begin{Verbatim}[commandchars=\\\{\}]
{\color{incolor}In [{\color{incolor}105}]:} \PY{c}{\PYZsh{}\PYZsh{} A chain}
          \PY{n}{b2} \PY{o}{=} \PY{p}{(}\PY{n}{c}\PY{o}{=} \PY{l+m+mi}{20} \PY{p}{;} \PY{n}{d} \PY{o}{=} \PY{l+m+mi}{5} \PY{p}{;} \PY{n}{c} \PY{o}{*} \PY{n}{d}\PY{p}{)}
          \PY{n}{println}\PY{p}{(}\PY{l+s}{\PYZdq{}}\PY{l+s}{b}\PY{l+s}{2}\PY{l+s}{:}\PY{l+s}{ }\PY{l+s}{\PYZdq{}}\PY{p}{,} \PY{n}{b2}\PY{p}{)}
\end{Verbatim}

    \begin{Verbatim}[commandchars=\\\{\}]
b2: 100

    \end{Verbatim}

    \hypertarget{conditional-evaluation}{%
\subsubsection{Conditional Evaluation}\label{conditional-evaluation}}

Conditional evaluation allows parts of a program to evaluated, or not,
based on the value of a Boolean expression, i.e., an expression that
produces a true/false value. In Julia, conditional evaluation takes the
form of an \texttt{if-elseif-else} construct, which is evaluated until
the first BOolean expression evaluates to true or the else statement is
reached. When is given Boolean expression evaluates to true, the
associated block of code executed. No other code blocks or condition
expressions within the \texttt{if-elseif-else} construct returns the
value of the last executed statement. Programmers can use as many
\texttt{elseif} blocks as they wish, including none, i.e., an
\texttt{if-else} construct. In Julia, \texttt{if}, \texttt{elseif} and
\texttt{else} statements do not require parentheses; in fact, their use
is discouraged.

    \begin{Verbatim}[commandchars=\\\{\}]
{\color{incolor}In [{\color{incolor}106}]:} \PY{c}{\PYZsh{} An if\PYZhy{}else construct}
          \PY{n}{k} \PY{o}{=} \PY{l+m+mi}{1}
          \PY{k}{if} \PY{n}{k} \PY{o}{==} \PY{l+m+mi}{0}
              \PY{l+s}{\PYZdq{}}\PY{l+s}{z}\PY{l+s}{e}\PY{l+s}{r}\PY{l+s}{o}\PY{l+s}{\PYZdq{}}
          \PY{k}{else}
              \PY{l+s}{\PYZdq{}}\PY{l+s}{n}\PY{l+s}{o}\PY{l+s}{t}\PY{l+s}{ }\PY{l+s}{z}\PY{l+s}{e}\PY{l+s}{r}\PY{l+s}{o}\PY{l+s}{\PYZdq{}}
          \PY{k}{end}
\end{Verbatim}

\begin{Verbatim}[commandchars=\\\{\}]
{\color{outcolor}Out[{\color{outcolor}106}]:} "not zero"
\end{Verbatim}
            
    \begin{Verbatim}[commandchars=\\\{\}]
{\color{incolor}In [{\color{incolor}107}]:} \PY{c}{\PYZsh{}\PYZsh{} An if\PYZhy{}elseif\PYZhy{}else construct}
          \PY{n}{k} \PY{o}{=} \PY{l+m+mi}{11}
          \PY{k}{if} \PY{n}{k} \PY{o}{\PYZpc{}} \PY{l+m+mi}{3} \PY{o}{==} \PY{l+m+mi}{0}
              \PY{l+m+mi}{0}
          \PY{k}{elseif} \PY{n}{k} \PY{o}{\PYZpc{}} \PY{l+m+mi}{3} \PY{o}{==} \PY{l+m+mi}{1}
              \PY{l+m+mi}{1}
          \PY{k}{else}
              \PY{l+m+mi}{2}
          \PY{k}{end}
\end{Verbatim}

\begin{Verbatim}[commandchars=\\\{\}]
{\color{outcolor}Out[{\color{outcolor}107}]:} 2
\end{Verbatim}
            
    An alternative approach to conditional evaluation is via shortcircuit
evaluation. This construct has the form \texttt{a\ ?\ b\ :\ c}, where
\texttt{a} is a Boolean expression, \texttt{b} is evaluated if
\texttt{a} is true, and \texttt{c} is evaluated if \texttt{a} is false.
Note that \texttt{?\ :} is called the ``ternary operator'', it
assocaites from right to left, and it can be useful for short
conditional statements. Ternary operators cna be chained together to
accommodate situations analogous to an \texttt{if-elseif-else} construct
with one or more \texttt{ifelse} blocks.

    \begin{Verbatim}[commandchars=\\\{\}]
{\color{incolor}In [{\color{incolor}108}]:} \PY{c}{\PYZsh{} A short\PYZhy{}circuit evaluation}
          \PY{n}{b} \PY{o}{=} \PY{l+m+mi}{10}\PY{p}{;} \PY{n}{c} \PY{o}{=} \PY{l+m+mi}{20}\PY{p}{;}
          \PY{n}{println}\PY{p}{(}\PY{l+s}{\PYZdq{}}\PY{l+s}{S}\PY{l+s}{C}\PY{l+s}{E}\PY{l+s}{:}\PY{l+s}{ }\PY{l+s}{b}\PY{l+s}{ }\PY{l+s}{\PYZlt{}}\PY{l+s}{ }\PY{l+s}{c}\PY{l+s}{:}\PY{l+s}{ }\PY{l+s}{\PYZdq{}}\PY{p}{,} \PY{n}{b} \PY{o}{\PYZlt{}} \PY{n}{c} \PY{o}{?} \PY{l+s}{\PYZdq{}}\PY{l+s}{l}\PY{l+s}{e}\PY{l+s}{s}\PY{l+s}{s}\PY{l+s}{ }\PY{l+s}{t}\PY{l+s}{h}\PY{l+s}{a}\PY{l+s}{n}\PY{l+s}{\PYZdq{}} \PY{o}{:} \PY{l+s}{\PYZdq{}}\PY{l+s}{n}\PY{l+s}{o}\PY{l+s}{t}\PY{l+s}{ }\PY{l+s}{l}\PY{l+s}{e}\PY{l+s}{s}\PY{l+s}{s}\PY{l+s}{ }\PY{l+s}{t}\PY{l+s}{h}\PY{l+s}{a}\PY{l+s}{n}\PY{l+s}{\PYZdq{}}\PY{p}{)}
\end{Verbatim}

    \begin{Verbatim}[commandchars=\\\{\}]
SCE: b < c: less than

    \end{Verbatim}

    \begin{Verbatim}[commandchars=\\\{\}]
{\color{incolor}In [{\color{incolor}109}]:} \PY{c}{\PYZsh{} A short\PYZhy{}circuit evaluation with nesting}
          \PY{n}{d} \PY{o}{=} \PY{l+m+mi}{10}\PY{p}{;} \PY{n}{f} \PY{o}{=} \PY{l+m+mi}{10}\PY{p}{;}
          \PY{n}{println}\PY{p}{(}\PY{l+s}{\PYZdq{}}\PY{l+s}{S}\PY{l+s}{C}\PY{l+s}{E}\PY{l+s}{:}\PY{l+s}{ }\PY{l+s}{c}\PY{l+s}{h}\PY{l+s}{a}\PY{l+s}{i}\PY{l+s}{n}\PY{l+s}{e}\PY{l+s}{d}\PY{l+s}{ }\PY{l+s}{d}\PY{l+s}{ }\PY{l+s}{v}\PY{l+s}{s}\PY{l+s}{ }\PY{l+s}{e}\PY{l+s}{:}\PY{l+s}{ }\PY{l+s}{\PYZdq{}}\PY{p}{,} 
                  \PY{n}{d} \PY{o}{\PYZlt{}} \PY{n}{f} \PY{o}{?} \PY{l+s}{\PYZdq{}}\PY{l+s}{l}\PY{l+s}{e}\PY{l+s}{s}\PY{l+s}{s}\PY{l+s}{ }\PY{l+s}{t}\PY{l+s}{h}\PY{l+s}{a}\PY{l+s}{n}\PY{l+s}{ }\PY{l+s}{\PYZdq{}} \PY{o}{:} 
                  \PY{n}{d} \PY{o}{\PYZgt{}} \PY{n}{f} \PY{o}{?} \PY{l+s}{\PYZdq{}}\PY{l+s}{g}\PY{l+s}{r}\PY{l+s}{e}\PY{l+s}{a}\PY{l+s}{t}\PY{l+s}{e}\PY{l+s}{r}\PY{l+s}{ }\PY{l+s}{t}\PY{l+s}{h}\PY{l+s}{a}\PY{l+s}{n}\PY{l+s}{\PYZdq{}} \PY{o}{:} \PY{l+s}{\PYZdq{}}\PY{l+s}{e}\PY{l+s}{q}\PY{l+s}{u}\PY{l+s}{a}\PY{l+s}{l}\PY{l+s}{\PYZdq{}}\PY{p}{)}
\end{Verbatim}

    \begin{Verbatim}[commandchars=\\\{\}]
SCE: chained d vs e: equal

    \end{Verbatim}

    Note that we do not use e in the above example because it is a literal
in Julia (the exponential function); while it can be overwritten, it is
best practice to avoid doing so.

    \begin{Verbatim}[commandchars=\\\{\}]
{\color{incolor}In [{\color{incolor}110}]:} \PY{n+nb}{e}
\end{Verbatim}

    \begin{Verbatim}[commandchars=\\\{\}]

        UndefVarError: e not defined

        

        Stacktrace:

         [1] top-level scope at In[110]:1

    \end{Verbatim}

    \begin{Verbatim}[commandchars=\\\{\}]
{\color{incolor}In [{\color{incolor}111}]:} \PY{k}{using} \PY{n}{Base}\PY{o}{.}\PY{n}{MathConstants}
          \PY{n+nb}{e}
\end{Verbatim}

\begin{Verbatim}[commandchars=\\\{\}]
{\color{outcolor}Out[{\color{outcolor}111}]:} ℯ = 2.7182818284590{\ldots}
\end{Verbatim}
            
    \hypertarget{loops}{%
\subsubsection{Loops}\label{loops}}

\hypertarget{basics}{%
\paragraph{Basics}\label{basics}}

Two looping constructs exist in Julia: \texttt{for} loops and
\texttt{while} loops. Theses loops can iterate over any container, such
as a string or an array. The body of loop ends with the \texttt{end}
keyword. Variables reference inside variables defined outside the body
of the lop, pre-append them with the global keyword inside the body of
the loop. A \texttt{for} loop can operate over a range object
representing a sequence of number, e.g., \texttt{1:5}, which it uses to
get each index to loop through the range of values in the range,
assigning each one to an indexing variable. THe indexing variable only
exists inside the loop. when looping over a container, \texttt{for}
loops can access the elements of the container directly using the
\texttt{in} operator. Rather than using simple nesting, nested
\texttt{for} loops can be written as a single outer loop with multiple
indexing variables forming a Cartesian product, e.g., if there are two
indexing variables then, for each value of the first index, each value
of the second index is evaluated.

    \begin{Verbatim}[commandchars=\\\{\}]
{\color{incolor}In [{\color{incolor}112}]:} \PY{n}{str} \PY{o}{=} \PY{l+s}{\PYZdq{}}\PY{l+s}{J}\PY{l+s}{u}\PY{l+s}{l}\PY{l+s}{i}\PY{l+s}{a}\PY{l+s}{\PYZdq{}}
          
          \PY{c}{\PYZsh{}\PYZsh{} A for loop for a string, iterating by index}
          \PY{k}{for} \PY{n}{i} \PY{o}{=} \PY{l+m+mi}{1}\PY{o}{:}\PY{n}{length}\PY{p}{(}\PY{n}{str}\PY{p}{)}
              \PY{n}{print}\PY{p}{(}\PY{n}{str}\PY{p}{[}\PY{n}{i}\PY{p}{]}\PY{p}{)}
          \PY{k}{end}
\end{Verbatim}

    \begin{Verbatim}[commandchars=\\\{\}]
Julia
    \end{Verbatim}

    \begin{Verbatim}[commandchars=\\\{\}]
{\color{incolor}In [{\color{incolor}113}]:} \PY{c}{\PYZsh{}\PYZsh{} A for loop for a string, iterating by container element}
          \PY{k}{for} \PY{n}{s} \PY{k+kp}{in} \PY{n}{str}
              \PY{n}{print}\PY{p}{(}\PY{n}{s}\PY{p}{)}
          \PY{k}{end}
\end{Verbatim}

    \begin{Verbatim}[commandchars=\\\{\}]
Julia
    \end{Verbatim}

    \begin{Verbatim}[commandchars=\\\{\}]
{\color{incolor}In [{\color{incolor}114}]:} \PY{c}{\PYZsh{}\PYZsh{} A nested for loop}
          \PY{k}{for} \PY{n}{i} \PY{k+kp}{in} \PY{n}{str}\PY{p}{,} \PY{n}{j} \PY{o}{=} \PY{n}{length}\PY{p}{(}\PY{n}{str}\PY{p}{)}
              \PY{n}{println}\PY{p}{(}\PY{p}{(}\PY{n}{i}\PY{p}{,}\PY{n}{j}\PY{p}{)}\PY{p}{)}
          \PY{k}{end}
\end{Verbatim}

    \begin{Verbatim}[commandchars=\\\{\}]
('J', 5)
('u', 5)
('l', 5)
('i', 5)
('a', 5)

    \end{Verbatim}

    \begin{Verbatim}[commandchars=\\\{\}]
{\color{incolor}In [{\color{incolor}115}]:} \PY{n}{odd} \PY{o}{=} \PY{p}{[}\PY{l+m+mi}{1}\PY{p}{,}\PY{l+m+mi}{3}\PY{p}{,}\PY{l+m+mi}{5}\PY{p}{]}
          \PY{n}{even} \PY{o}{=} \PY{p}{[}\PY{l+m+mi}{2}\PY{p}{,}\PY{l+m+mi}{4}\PY{p}{,}\PY{l+m+mi}{6}\PY{p}{]}
          \PY{k}{for} \PY{n}{i} \PY{k+kp}{in} \PY{n}{odd}\PY{p}{,} \PY{n}{j} \PY{k+kp}{in} \PY{n}{even}
              \PY{n}{println}\PY{p}{(}\PY{l+s}{\PYZdq{}}\PY{l+s}{i}\PY{l+s}{*}\PY{l+s}{j}\PY{l+s}{:}\PY{l+s}{ }\PY{l+s+si}{\PYZdl{}}\PY{p}{(}\PY{n}{i}\PY{o}{*}\PY{n}{j}\PY{p}{)}\PY{l+s}{\PYZdq{}}\PY{p}{)}
          \PY{k}{end}
\end{Verbatim}

    \begin{Verbatim}[commandchars=\\\{\}]
i*j: 2
i*j: 4
i*j: 6
i*j: 6
i*j: 12
i*j: 18
i*j: 10
i*j: 20
i*j: 30

    \end{Verbatim}

    A while loop evaluates a conditional expression and, as long as it is
true, the loop evaluates the code in the body of the loop. To ensure
that the loop will end at some stage, an operation inside the loop has
to falsify the conditional expression. Programmers must ensure that a
\texttt{while} loop will falsify the conditional expression, otherwise
the loop will become ``infinity'' and never finish executing.

    \begin{Verbatim}[commandchars=\\\{\}]
{\color{incolor}In [{\color{incolor}116}]:} \PY{c}{\PYZsh{}\PYZsh{} Example of an infinity while loop (nothing inside the loop can falsify}
          \PY{c}{\PYZsh{}\PYZsh{} the condition x\PYZlt{}10)}
          \PY{n}{n}\PY{o}{=}\PY{l+m+mi}{0}
          \PY{n}{x}\PY{o}{=}\PY{l+m+mi}{1}
          \PY{k}{while} \PY{n}{x}\PY{o}{\PYZlt{}}\PY{l+m+mi}{10}
              \PY{n}{x}\PY{o}{=}\PY{n}{x}\PY{o}{+}\PY{l+m+mi}{1}
              \PY{n}{print}\PY{p}{(}\PY{l+m+mi}{10}\PY{p}{)}
          \PY{k}{end}
\end{Verbatim}

    \begin{Verbatim}[commandchars=\\\{\}]
101010101010101010
    \end{Verbatim}

    \begin{Verbatim}[commandchars=\\\{\}]
{\color{incolor}In [{\color{incolor}117}]:} \PY{c}{\PYZsh{}\PYZsh{} A while loop to estimate the median using an MM algorithm}
          \PY{k}{using} \PY{n}{Distributions}\PY{p}{,} \PY{n}{Random}
          \PY{n}{Random}\PY{o}{.}\PY{n}{seed!}\PY{p}{(}\PY{l+m+mi}{1234}\PY{p}{)}
          
          \PY{n}{iter} \PY{o}{=} \PY{l+m+mi}{0}
          \PY{n}{N} \PY{o}{=} \PY{l+m+mi}{100}
          \PY{n}{x} \PY{o}{=} \PY{n}{rand}\PY{p}{(}\PY{n}{Normal}\PY{p}{(}\PY{l+m+mi}{2}\PY{p}{,}\PY{l+m+mi}{1}\PY{p}{)}\PY{p}{,}\PY{n}{N}\PY{p}{)}
          \PY{n}{psi} \PY{o}{=} \PY{n}{fill!}\PY{p}{(}\PY{k+kt}{Vector}\PY{p}{\PYZob{}}\PY{k+kt}{Float64}\PY{p}{\PYZcb{}}\PY{p}{(}\PY{n}{undef}\PY{p}{,}\PY{l+m+mi}{2}\PY{p}{)}\PY{p}{,}\PY{l+m+mf}{1e9}\PY{p}{)}
          
          \PY{k}{while} \PY{p}{(}\PY{k+kc}{true}\PY{p}{)}
              \PY{k+kd}{global} \PY{n}{iter}\PY{p}{,} \PY{n}{x}\PY{p}{,} \PY{n}{psi}
              \PY{n}{iter} \PY{o}{+=} \PY{l+m+mi}{1} 
              \PY{k}{if} \PY{n}{iter} \PY{o}{==} \PY{l+m+mi}{25}
                  \PY{n}{println}\PY{p}{(}\PY{l+s}{\PYZdq{}}\PY{l+s}{M}\PY{l+s}{a}\PY{l+s}{x}\PY{l+s}{ }\PY{l+s}{i}\PY{l+s}{t}\PY{l+s}{e}\PY{l+s}{r}\PY{l+s}{a}\PY{l+s}{t}\PY{l+s}{i}\PY{l+s}{o}\PY{l+s}{n}\PY{l+s}{ }\PY{l+s}{r}\PY{l+s}{e}\PY{l+s}{a}\PY{l+s}{c}\PY{l+s}{h}\PY{l+s}{e}\PY{l+s}{d}\PY{l+s}{ }\PY{l+s}{a}\PY{l+s}{t}\PY{l+s}{ }\PY{l+s}{i}\PY{l+s}{t}\PY{l+s}{e}\PY{l+s}{r}\PY{l+s}{ }\PY{l+s}{=}\PY{l+s}{ }\PY{l+s+si}{\PYZdl{}iter}\PY{l+s}{\PYZdq{}}\PY{p}{)}
                  \PY{k}{break}
              \PY{k}{end}
              \PY{n}{num}\PY{p}{,} \PY{n}{den} \PY{o}{=} \PY{p}{(}\PY{l+m+mi}{0}\PY{p}{,}\PY{l+m+mi}{0}\PY{p}{)}
              \PY{c}{\PYZsh{}\PYZsh{} elementwise operations in wgt}
              \PY{n}{wgt} \PY{o}{=} \PY{p}{(}\PY{n}{abs}\PY{o}{.}\PY{p}{(}\PY{n}{x} \PY{o}{.\PYZhy{}} \PY{n}{psi}\PY{p}{[}\PY{l+m+mi}{2}\PY{p}{]}\PY{p}{)}\PY{p}{)}\PY{o}{.\PYZca{}}\PY{o}{\PYZhy{}}\PY{l+m+mi}{1}
              \PY{n}{num} \PY{o}{=} \PY{n}{sum}\PY{p}{(}\PY{n}{wgt} \PY{o}{.*} \PY{n}{x}\PY{p}{)}
              \PY{n}{den} \PY{o}{=} \PY{n}{sum}\PY{p}{(}\PY{n}{wgt}\PY{p}{)}
              \PY{n}{psi} \PY{o}{=} \PY{n}{circshift}\PY{p}{(}\PY{n}{psi}\PY{p}{,} \PY{l+m+mi}{1}\PY{p}{)}
              \PY{n}{psi}\PY{p}{[}\PY{l+m+mi}{2}\PY{p}{]} \PY{o}{=} \PY{n}{num}\PY{o}{/}\PY{n}{den}
              
              \PY{n}{dif} \PY{o}{=} \PY{n}{abs}\PY{p}{(}\PY{n}{psi}\PY{p}{[}\PY{l+m+mi}{2}\PY{p}{]}\PY{o}{\PYZhy{}}\PY{n}{psi}\PY{p}{[}\PY{l+m+mi}{1}\PY{p}{]}\PY{p}{)}
              \PY{k}{if} \PY{n}{dif} \PY{o}{\PYZlt{}} \PY{l+m+mf}{0.001}
                  \PY{n}{print}\PY{p}{(}\PY{l+s}{\PYZdq{}}\PY{l+s}{C}\PY{l+s}{o}\PY{l+s}{n}\PY{l+s}{v}\PY{l+s}{e}\PY{l+s}{r}\PY{l+s}{g}\PY{l+s}{e}\PY{l+s}{d}\PY{l+s}{ }\PY{l+s}{a}\PY{l+s}{t}\PY{l+s}{ }\PY{l+s}{i}\PY{l+s}{t}\PY{l+s}{e}\PY{l+s}{r}\PY{l+s}{a}\PY{l+s}{t}\PY{l+s}{i}\PY{l+s}{o}\PY{l+s}{n}\PY{l+s}{ }\PY{l+s+si}{\PYZdl{}iter}\PY{l+s}{\PYZdq{}}\PY{p}{)}
                  \PY{k}{break}
              \PY{k}{end}
          \PY{k}{end}
\end{Verbatim}

    \begin{Verbatim}[commandchars=\\\{\}]
Converged at iteration 2
    \end{Verbatim}

    \hypertarget{loop-termination}{%
\paragraph{Loop termination}\label{loop-termination}}

When writing loops, it often advantageous to allow a loop to terminate
early, before it has completed. In the case of a \texttt{while} loop,
the loop would be broken before the test condition is falsified. when
iterating over an iterable object with a \texttt{for} loop, it is
stopped before the end of the object is reached. The \texttt{break}
keyword can accomplish both tasks. The following code block has two
loops, a \texttt{while} loop that calculates the square of the index
variable and stops when the square is graeter than 16. Note that without
the \texttt{break} keyword, this is an infinite loop. The second loop
dose the same thing, but uses a \texttt{for} loop to do it. The
\texttt{for} loop terminates before the end of the iterable range object
is reached.

    \begin{Verbatim}[commandchars=\\\{\}]
{\color{incolor}In [{\color{incolor}118}]:} \PY{c}{\PYZsh{}\PYZsh{} break keyword}
          \PY{n}{i} \PY{o}{=} \PY{l+m+mi}{0}
          \PY{k}{while} \PY{k+kc}{true}
              \PY{k+kd}{global} \PY{n}{i}
              \PY{n}{sq} \PY{o}{=} \PY{n}{i}\PY{o}{\PYZca{}}\PY{l+m+mi}{2}
              \PY{n}{println}\PY{p}{(}\PY{l+s}{\PYZdq{}}\PY{l+s}{i}\PY{l+s}{:}\PY{l+s}{ }\PY{l+s+si}{\PYZdl{}i}\PY{l+s}{ }\PY{l+s}{\PYZhy{}}\PY{l+s}{\PYZhy{}}\PY{l+s}{\PYZhy{}}\PY{l+s}{ }\PY{l+s}{s}\PY{l+s}{q}\PY{l+s}{:}\PY{l+s}{ }\PY{l+s}{=}\PY{l+s}{ }\PY{l+s+si}{\PYZdl{}sq}\PY{l+s}{\PYZdq{}}\PY{p}{)}
              \PY{k}{if} \PY{n}{sq} \PY{o}{\PYZgt{}} \PY{l+m+mi}{16}
                  \PY{k}{break}
              \PY{k}{end}
              \PY{n}{i} \PY{o}{+=} \PY{l+m+mi}{1}
          \PY{k}{end}
\end{Verbatim}

    \begin{Verbatim}[commandchars=\\\{\}]
i: 0 --- sq: = 0
i: 1 --- sq: = 1
i: 2 --- sq: = 4
i: 3 --- sq: = 9
i: 4 --- sq: = 16
i: 5 --- sq: = 25

    \end{Verbatim}

    \begin{Verbatim}[commandchars=\\\{\}]
{\color{incolor}In [{\color{incolor}119}]:} \PY{k}{for} \PY{n}{i} \PY{o}{=} \PY{l+m+mi}{1}\PY{o}{:}\PY{l+m+mi}{10}
              \PY{n}{sq} \PY{o}{=} \PY{n}{i}\PY{o}{\PYZca{}}\PY{l+m+mi}{2}
              \PY{n}{println}\PY{p}{(}\PY{l+s}{\PYZdq{}}\PY{l+s}{i}\PY{l+s}{:}\PY{l+s}{ }\PY{l+s+si}{\PYZdl{}i}\PY{l+s}{ }\PY{l+s}{\PYZhy{}}\PY{l+s}{\PYZhy{}}\PY{l+s}{\PYZhy{}}\PY{l+s}{ }\PY{l+s}{s}\PY{l+s}{q}\PY{l+s}{:}\PY{l+s}{ }\PY{l+s+si}{\PYZdl{}sq}\PY{l+s}{\PYZdq{}}\PY{p}{)}
              \PY{k}{if} \PY{n}{sq} \PY{o}{\PYZgt{}} \PY{l+m+mi}{16}
                  \PY{k}{break}
              \PY{k}{end}
          \PY{k}{end}
\end{Verbatim}

    \begin{Verbatim}[commandchars=\\\{\}]
i: 1 --- sq: 1
i: 2 --- sq: 4
i: 3 --- sq: 9
i: 4 --- sq: 16
i: 5 --- sq: 25

    \end{Verbatim}

    In some situations, it might be the case that a programmer wants to move
from the current iteration of a loop immediately into the next iteration
before the current one is finished. This can be accomplished using the
\texttt{continue} keyword.

    \begin{Verbatim}[commandchars=\\\{\}]
{\color{incolor}In [{\color{incolor}120}]:} \PY{c}{\PYZsh{}\PYZsh{} continue keyword}
          \PY{k}{for} \PY{n}{i} \PY{k+kp}{in} \PY{l+m+mi}{1}\PY{o}{:}\PY{l+m+mi}{5}
              \PY{k}{if} \PY{n}{i} \PY{o}{\PYZpc{}} \PY{l+m+mi}{2} \PY{o}{==} \PY{l+m+mi}{0}
                  \PY{k}{continue}
              \PY{k}{end}
              \PY{n}{sq} \PY{o}{=} \PY{n}{i}\PY{o}{\PYZca{}}\PY{l+m+mi}{2}
              \PY{n}{println}\PY{p}{(}\PY{l+s}{\PYZdq{}}\PY{l+s}{i}\PY{l+s}{:}\PY{l+s}{ }\PY{l+s+si}{\PYZdl{}i}\PY{l+s}{ }\PY{l+s}{\PYZhy{}}\PY{l+s}{\PYZhy{}}\PY{l+s}{\PYZhy{}}\PY{l+s}{ }\PY{l+s}{s}\PY{l+s}{q}\PY{l+s}{:}\PY{l+s}{ }\PY{l+s+si}{\PYZdl{}sq}\PY{l+s}{\PYZdq{}}\PY{p}{)}
          \PY{k}{end}
\end{Verbatim}

    \begin{Verbatim}[commandchars=\\\{\}]
i: 1 --- sq: 1
i: 3 --- sq: 9
i: 5 --- sq: 25

    \end{Verbatim}

    In real world scenarios, continue could be used multiple times in a loop
and there could be more complex code after the \texttt{continue}
keyword.

    \hypertarget{exception-handling}{%
\paragraph{Exception handling}\label{exception-handling}}

Exceptions are unexpected contions that can occur in a program while it
is carrying out its computations. The program may not be able to carry
out the required computations or return a sensible value to its caller.
Usually, exceptions teminate the function or program that generates it
and prints some sort of diagnostic message to standard output. An
example of this is given in the following code block, where we try and
take the logarithm of a negative number and the \texttt{log()} function
throws an exception.

    \begin{Verbatim}[commandchars=\\\{\}]
{\color{incolor}In [{\color{incolor}121}]:} \PY{c}{\PYZsh{}\PYZsh{} generate an exception}
          \PY{n}{log}\PY{p}{(}\PY{o}{\PYZhy{}}\PY{l+m+mi}{1}\PY{p}{)}
\end{Verbatim}

    \begin{Verbatim}[commandchars=\\\{\}]

        DomainError with -1.0:
    log will only return a complex result if called with a complex argument. Try log(Complex(x)).

        

        Stacktrace:

         [1] throw\_complex\_domainerror(::Symbol, ::Float64) at .\textbackslash{}math.jl:31

         [2] log(::Float64) at .\textbackslash{}special\textbackslash{}log.jl:285

         [3] log(::Int64) at .\textbackslash{}special\textbackslash{}log.jl:395

         [4] top-level scope at In[121]:1

    \end{Verbatim}

    In the above code block the \texttt{log()} function threw a
\texttt{DomainError} exception. Julia has a number of built-in
exceptions that can be thrown captured by Julia program. Any exception
can be explicitly thrown using \texttt{throw()} function.

    \begin{Verbatim}[commandchars=\\\{\}]
{\color{incolor}In [{\color{incolor}122}]:} \PY{c}{\PYZsh{}\PYZsh{} throw()}
          \PY{k}{for} \PY{n}{i} \PY{k+kp}{in} \PY{p}{[}\PY{l+m+mi}{1}\PY{p}{,}\PY{l+m+mi}{2}\PY{p}{,}\PY{o}{\PYZhy{}}\PY{l+m+mi}{1}\PY{p}{,}\PY{l+m+mi}{3}\PY{p}{]}
              \PY{k}{if} \PY{n}{i} \PY{o}{\PYZlt{}} \PY{l+m+mi}{0}
                  \PY{n}{throw}\PY{p}{(}\PY{k+kt}{DomainError}\PY{p}{(}\PY{p}{)}\PY{p}{)}
              \PY{k}{else}
                  \PY{n}{println}\PY{p}{(}\PY{l+s}{\PYZdq{}}\PY{l+s}{i}\PY{l+s}{:}\PY{l+s+si}{\PYZdl{}}\PY{p}{(}\PY{n}{log}\PY{p}{(}\PY{n}{i}\PY{p}{)}\PY{p}{)}\PY{l+s}{\PYZdq{}}\PY{p}{)}
              \PY{k}{end}
          \PY{k}{end}
\end{Verbatim}

    \begin{Verbatim}[commandchars=\\\{\}]
i:0.0
i:0.6931471805599453

    \end{Verbatim}

    \begin{Verbatim}[commandchars=\\\{\}]

        MethodError: no method matching DomainError()
    Closest candidates are:
      DomainError(!Matched::Any) at boot.jl:256
      DomainError(!Matched::Any, !Matched::Any) at boot.jl:257

        

        Stacktrace:

         [1] top-level scope at .\textbackslash{}In[122]:4

    \end{Verbatim}

    \begin{Verbatim}[commandchars=\\\{\}]
{\color{incolor}In [{\color{incolor}123}]:} \PY{c}{\PYZsh{}\PYZsh{} error}
          \PY{k}{for} \PY{n}{i} \PY{k+kp}{in} \PY{p}{[}\PY{l+m+mi}{1}\PY{p}{,}\PY{l+m+mi}{2}\PY{p}{,}\PY{o}{\PYZhy{}}\PY{l+m+mi}{1}\PY{p}{,}\PY{l+m+mi}{3}\PY{p}{]}
              \PY{k}{if} \PY{n}{i}\PY{o}{\PYZlt{}}\PY{l+m+mi}{0}
                  \PY{n}{error}\PY{p}{(}\PY{l+s}{\PYZdq{}}\PY{l+s}{i}\PY{l+s}{ }\PY{l+s}{i}\PY{l+s}{s}\PY{l+s}{ }\PY{l+s}{a}\PY{l+s}{ }\PY{l+s}{n}\PY{l+s}{e}\PY{l+s}{g}\PY{l+s}{a}\PY{l+s}{t}\PY{l+s}{i}\PY{l+s}{v}\PY{l+s}{e}\PY{l+s}{ }\PY{l+s}{n}\PY{l+s}{u}\PY{l+s}{m}\PY{l+s}{b}\PY{l+s}{e}\PY{l+s}{r}\PY{l+s}{\PYZdq{}}\PY{p}{)}
              \PY{k}{else}
                  \PY{n}{println}\PY{p}{(}\PY{l+s}{\PYZdq{}}\PY{l+s}{i}\PY{l+s}{:}\PY{l+s+si}{\PYZdl{}}\PY{p}{(}\PY{n}{log}\PY{p}{(}\PY{n}{i}\PY{p}{)}\PY{p}{)}\PY{l+s}{\PYZdq{}}\PY{p}{)}
              \PY{k}{end}
          \PY{k}{end}
\end{Verbatim}

    \begin{Verbatim}[commandchars=\\\{\}]
i:0.0
i:0.6931471805599453

    \end{Verbatim}

    \begin{Verbatim}[commandchars=\\\{\}]

        i is a negative number

        

        Stacktrace:

         [1] error(::String) at .\textbackslash{}error.jl:33

         [2] top-level scope at .\textbackslash{}In[123]:4

    \end{Verbatim}

    In the previous code block, we throw the \texttt{DomainError()}
exception when the input to \texttt{log()} is negative. Note that
\texttt{DomainError()} requires the bracket \texttt{()} to return an
exception object of type \texttt{ErrorException} that will immediately
stop all execution of the Julia program.

If we want to test for an excepiton and handle it gracefully, we can use
a \texttt{try-each} statment to do this. These statements allow us to
catch an exception, store it in a variable if required, and try an
alternative way of processing the input that generated the exception.

    \begin{Verbatim}[commandchars=\\\{\}]
{\color{incolor}In [{\color{incolor}124}]:} \PY{c}{\PYZsh{}\PYZsh{} try/catch}
          \PY{k}{for} \PY{n}{i} \PY{k+kp}{in} \PY{p}{[}\PY{l+m+mi}{1}\PY{p}{,}\PY{l+m+mi}{2}\PY{p}{,}\PY{o}{\PYZhy{}}\PY{l+m+mi}{1}\PY{p}{,}\PY{l+s}{\PYZdq{}}\PY{l+s}{A}\PY{l+s}{\PYZdq{}}\PY{p}{]}
              \PY{k}{try} \PY{n}{log}\PY{p}{(}\PY{n}{i}\PY{p}{)}
              \PY{k}{catch} \PY{n}{ex}
                  \PY{k}{if} \PY{n}{isa}\PY{p}{(}\PY{n}{ex}\PY{p}{,} \PY{k+kt}{DomainError}\PY{p}{)}
                      \PY{n}{println}\PY{p}{(}\PY{l+s}{\PYZdq{}}\PY{l+s}{i}\PY{l+s}{:}\PY{l+s}{ }\PY{l+s+si}{\PYZdl{}i}\PY{l+s}{ }\PY{l+s}{\PYZhy{}}\PY{l+s}{\PYZhy{}}\PY{l+s}{\PYZhy{}}\PY{l+s}{ }\PY{l+s}{D}\PY{l+s}{o}\PY{l+s}{m}\PY{l+s}{a}\PY{l+s}{i}\PY{l+s}{n}\PY{l+s}{ }\PY{l+s}{E}\PY{l+s}{r}\PY{l+s}{r}\PY{l+s}{o}\PY{l+s}{r}\PY{l+s}{\PYZdq{}}\PY{p}{)}
                      \PY{n}{log}\PY{p}{(}\PY{n}{abs}\PY{p}{(}\PY{n}{i}\PY{p}{)}\PY{p}{)}
                  \PY{k}{else}
                      \PY{n}{println}\PY{p}{(}\PY{l+s}{\PYZdq{}}\PY{l+s}{i}\PY{l+s}{:}\PY{l+s}{ }\PY{l+s+si}{\PYZdl{}i}\PY{l+s}{\PYZdq{}}\PY{p}{)}
                      \PY{n}{println}\PY{p}{(}\PY{n}{ex}\PY{p}{)}
                      \PY{n}{error}\PY{p}{(}\PY{l+s}{\PYZdq{}}\PY{l+s}{N}\PY{l+s}{o}\PY{l+s}{t}\PY{l+s}{ }\PY{l+s}{a}\PY{l+s}{ }\PY{l+s}{D}\PY{l+s}{o}\PY{l+s}{m}\PY{l+s}{a}\PY{l+s}{i}\PY{l+s}{n}\PY{l+s}{E}\PY{l+s}{r}\PY{l+s}{r}\PY{l+s}{o}\PY{l+s}{r}\PY{l+s}{\PYZdq{}}\PY{p}{)}
                  \PY{k}{end}
              \PY{k}{end}
          \PY{k}{end}
\end{Verbatim}

    \begin{Verbatim}[commandchars=\\\{\}]
i: -1 --- Domain Error
i: A
MethodError(log, ("A",), 0x000000000000641c)

    \end{Verbatim}

    \begin{Verbatim}[commandchars=\\\{\}]

        Not a DomainError

        

        Stacktrace:

         [1] error(::String) at .\textbackslash{}error.jl:33

         [2] top-level scope at .\textbackslash{}In[124]:11

    \end{Verbatim}

    In the previous code block, the exception is stored in the \texttt{ex}
variable and when the error is not a \texttt{DomainError()}, its value
is returned along with the \texttt{ErrorException} defined by the call
to \texttt{error()}. Note that \texttt{try-catch} blocks can degrade the
preformance code, it is better to use standard condtitional evaluation
to handle known exceptions.

    \hypertarget{functions}{%
\subsection{Functions}\label{functions}}

A function is an object that takes argument values as a tuple and maps
them to return value. Functions are first-class objects in Julia. They
can be:

\begin{itemize}
\tightlist
\item
  assigned to variables;
\item
  called from these variables;
\item
  passed as arguments to other functions;
\item
  returned a svalues from a function.
\end{itemize}

A first-class object is one that accomodates all operations other
objects support. Operations typically supported by first-class bojects
in all programming languages are listed above. The basic syntax of a
function is illustrated in the following code block.

    \begin{Verbatim}[commandchars=\\\{\}]
{\color{incolor}In [{\color{incolor}125}]:} \PY{k}{function} \PY{n}{add}\PY{p}{(}\PY{n}{x}\PY{p}{,}\PY{n}{y}\PY{p}{)}
              \PY{k}{return} \PY{p}{(}\PY{n}{x}\PY{o}{+}\PY{n}{y}\PY{p}{)}
          \PY{k}{end}
\end{Verbatim}

\begin{Verbatim}[commandchars=\\\{\}]
{\color{outcolor}Out[{\color{outcolor}125}]:} add (generic function with 1 method)
\end{Verbatim}
            
    In Julia, function names are all lowercase, without underscores, but can
include Unicode charaters. It is best practice to avoid abbreviations,
e.g., \texttt{fibonacci()} is preferable to \texttt{fib()}. The body of
the function is the part contained on the line between the
\texttt{function} and \texttt{end} keywords. Parenthesis syntax is used
to call a function, e.g.~\texttt{add(3,5)} returns \texttt{8}. Because
functions are objects, they can be passed around like any value and,
when passed, the parentheses are omitted.

    \begin{Verbatim}[commandchars=\\\{\}]
{\color{incolor}In [{\color{incolor}126}]:} \PY{n}{addnew} \PY{o}{=} \PY{n}{add}
          \PY{n}{addnew}\PY{p}{(}\PY{l+m+mi}{3}\PY{p}{,}\PY{l+m+mi}{5}\PY{p}{)}
\end{Verbatim}

\begin{Verbatim}[commandchars=\\\{\}]
{\color{outcolor}Out[{\color{outcolor}126}]:} 8
\end{Verbatim}
            
    Functions may also be written in assignment form, in which case the body
of the function must be a single expression. This can be a very useful
approach for simple functions because it makes code mych easier to read.

    \begin{Verbatim}[commandchars=\\\{\}]
{\color{incolor}In [{\color{incolor}127}]:} \PY{n}{add2}\PY{p}{(}\PY{n}{x}\PY{p}{,}\PY{n}{y}\PY{p}{)} \PY{o}{=} \PY{n}{x}\PY{o}{+}\PY{n}{y}
\end{Verbatim}

\begin{Verbatim}[commandchars=\\\{\}]
{\color{outcolor}Out[{\color{outcolor}127}]:} add2 (generic function with 1 method)
\end{Verbatim}
            
    Argument passing is done by refernece. Modeifications to the input data
structure (e.g., array) inside the function will be visible outside it.
If function inputs are not to be modified by a function, a copy of the
input(s) should be made inside the functino before doing any
modification. \emph{Python} and other dynamic languages handle their
functino arguments in a similar way.

    \begin{Verbatim}[commandchars=\\\{\}]
{\color{incolor}In [{\color{incolor}128}]:} \PY{c}{\PYZsh{}\PYZsh{} Argument passing}
          \PY{k}{function} \PY{n}{f1!}\PY{p}{(}\PY{n}{x}\PY{p}{)}
              \PY{n}{x}\PY{p}{[}\PY{l+m+mi}{1}\PY{p}{]} \PY{o}{=} \PY{l+m+mi}{9999}
              \PY{k}{return}\PY{p}{(}\PY{n}{x}\PY{p}{)}
          \PY{k}{end}
          
          \PY{n}{ia} \PY{o}{=} \PY{k+kt}{Int64}\PY{p}{[}\PY{l+m+mi}{0}\PY{p}{,}\PY{l+m+mi}{1}\PY{p}{,}\PY{l+m+mi}{2}\PY{p}{]}
          \PY{n}{println}\PY{p}{(}\PY{l+s}{\PYZdq{}}\PY{l+s}{A}\PY{l+s}{r}\PY{l+s}{r}\PY{l+s}{a}\PY{l+s}{y}\PY{l+s}{ }\PY{l+s}{i}\PY{l+s}{a}\PY{l+s}{:}\PY{l+s}{ }\PY{l+s}{\PYZdq{}}\PY{p}{,} \PY{n}{ia}\PY{p}{)}
\end{Verbatim}

    \begin{Verbatim}[commandchars=\\\{\}]
Array ia: [0, 1, 2]

    \end{Verbatim}

    \begin{Verbatim}[commandchars=\\\{\}]
{\color{incolor}In [{\color{incolor}129}]:} \PY{n}{f1!}\PY{p}{(}\PY{n}{ia}\PY{p}{)}
          \PY{n}{println}\PY{p}{(}\PY{l+s}{\PYZdq{}}\PY{l+s}{A}\PY{l+s}{r}\PY{l+s}{g}\PY{l+s}{u}\PY{l+s}{m}\PY{l+s}{e}\PY{l+s}{n}\PY{l+s}{t}\PY{l+s}{ }\PY{l+s}{p}\PY{l+s}{a}\PY{l+s}{s}\PY{l+s}{s}\PY{l+s}{i}\PY{l+s}{n}\PY{l+s}{g}\PY{l+s}{ }\PY{l+s}{b}\PY{l+s}{e}\PY{l+s}{ }\PY{l+s}{r}\PY{l+s}{e}\PY{l+s}{f}\PY{l+s}{e}\PY{l+s}{r}\PY{l+s}{e}\PY{l+s}{n}\PY{l+s}{c}\PY{l+s}{e}\PY{l+s}{:}\PY{l+s}{ }\PY{l+s}{\PYZdq{}}\PY{p}{,} \PY{n}{ia}\PY{p}{)}
\end{Verbatim}

    \begin{Verbatim}[commandchars=\\\{\}]
Argument passing be reference: [9999, 1, 2]

    \end{Verbatim}

    By default, the last expression that is evaluated in the body of a
function is its reutrn value. However, when the function body contains
one or more \texttt{return} keywords, it returns immdiately when a
\texttt{return} keyword is evaluated. The \texttt{return} keywird
usually wraps an expression that provides a value when returned. When
used with the control flow statements, the return keyword can be
especially useful.

    \begin{Verbatim}[commandchars=\\\{\}]
{\color{incolor}In [{\color{incolor}130}]:} \PY{c}{\PYZsh{}\PYZsh{} A function with multiple option for return}
          \PY{k}{function} \PY{n}{gt}\PY{p}{(}\PY{n}{g1}\PY{p}{,} \PY{n}{g2}\PY{p}{)}
              \PY{k}{if} \PY{p}{(}\PY{n}{g1} \PY{o}{\PYZgt{}} \PY{n}{g2}\PY{p}{)}
                  \PY{k}{return}\PY{p}{(}\PY{l+s}{\PYZdq{}}\PY{l+s+si}{\PYZdl{}g1}\PY{l+s}{ }\PY{l+s}{i}\PY{l+s}{s}\PY{l+s}{ }\PY{l+s}{l}\PY{l+s}{a}\PY{l+s}{r}\PY{l+s}{g}\PY{l+s}{e}\PY{l+s}{s}\PY{l+s}{t}\PY{l+s}{\PYZdq{}}\PY{p}{)}
              \PY{k}{elseif} \PY{p}{(}\PY{n}{g1} \PY{o}{\PYZlt{}} \PY{n}{g2}\PY{p}{)}
                  \PY{k}{return} \PY{p}{(}\PY{l+s}{\PYZdq{}}\PY{l+s+si}{\PYZdl{}g2}\PY{l+s}{ }\PY{l+s}{i}\PY{l+s}{s}\PY{l+s}{ }\PY{l+s}{l}\PY{l+s}{a}\PY{l+s}{r}\PY{l+s}{g}\PY{l+s}{e}\PY{l+s}{s}\PY{l+s}{t}\PY{l+s}{\PYZdq{}}\PY{p}{)}
              \PY{k}{else}
                  \PY{k}{return}\PY{p}{(}\PY{l+s}{\PYZdq{}}\PY{l+s+si}{\PYZdl{}g1}\PY{l+s}{ }\PY{l+s}{a}\PY{l+s}{n}\PY{l+s}{d}\PY{l+s}{ }\PY{l+s+si}{\PYZdl{}g2}\PY{l+s}{ }\PY{l+s}{a}\PY{l+s}{r}\PY{l+s}{e}\PY{l+s}{ }\PY{l+s}{e}\PY{l+s}{q}\PY{l+s}{u}\PY{l+s}{a}\PY{l+s}{l}\PY{l+s}{\PYZdq{}}\PY{p}{)}
              \PY{k}{end}
          \PY{k}{end}
          
          \PY{n}{gt}\PY{p}{(}\PY{l+m+mi}{2}\PY{p}{,}\PY{l+m+mi}{5}\PY{p}{)}
\end{Verbatim}

\begin{Verbatim}[commandchars=\\\{\}]
{\color{outcolor}Out[{\color{outcolor}130}]:} "5 is largest"
\end{Verbatim}
            
    The majority of Julia operators are actually functions and can be called
with parenthesized argument lists, just like other functinos.

    \begin{Verbatim}[commandchars=\\\{\}]
{\color{incolor}In [{\color{incolor}131}]:} \PY{c}{\PYZsh{}\PYZsh{} these are equivalent}
          \PY{l+m+mi}{2}\PY{o}{*}\PY{l+m+mi}{3}
\end{Verbatim}

\begin{Verbatim}[commandchars=\\\{\}]
{\color{outcolor}Out[{\color{outcolor}131}]:} 6
\end{Verbatim}
            
    \begin{Verbatim}[commandchars=\\\{\}]
{\color{incolor}In [{\color{incolor}132}]:} \PY{o}{*}\PY{p}{(}\PY{l+m+mi}{2}\PY{p}{,}\PY{l+m+mi}{3}\PY{p}{)}
\end{Verbatim}

\begin{Verbatim}[commandchars=\\\{\}]
{\color{outcolor}Out[{\color{outcolor}132}]:} 6
\end{Verbatim}
            
    Functions can also be created without a name, and such functions are
called anonymous functions. Anonymous functions can be used as arguments
for functinos that take other functions as arguments.

    \begin{Verbatim}[commandchars=\\\{\}]
{\color{incolor}In [{\color{incolor}133}]:} \PY{c}{\PYZsh{}\PYZsh{} map() applies a function to each element of an array and returns a new }
          \PY{c}{\PYZsh{}\PYZsh{} array containing the resulting values}
          \PY{n}{a} \PY{o}{=} \PY{p}{[}\PY{l+m+mi}{1}\PY{p}{,}\PY{l+m+mi}{2}\PY{p}{,}\PY{l+m+mi}{3}\PY{p}{,}\PY{l+m+mi}{1}\PY{p}{,}\PY{l+m+mi}{2}\PY{p}{,}\PY{l+m+mi}{1}\PY{p}{]}
          \PY{n}{μ} \PY{o}{=} \PY{n}{mean}\PY{p}{(}\PY{n}{a}\PY{p}{)}
          \PY{n}{sd} \PY{o}{=} \PY{n}{std}\PY{p}{(}\PY{n}{a}\PY{p}{)}
\end{Verbatim}

\begin{Verbatim}[commandchars=\\\{\}]
{\color{outcolor}Out[{\color{outcolor}133}]:} 0.816496580927726
\end{Verbatim}
            
    \begin{Verbatim}[commandchars=\\\{\}]
{\color{incolor}In [{\color{incolor}134}]:} \PY{c}{\PYZsh{}\PYZsh{} centers and scales a}
          \PY{n}{b} \PY{o}{=} \PY{n}{map}\PY{p}{(}\PY{n}{x} \PY{o}{\PYZhy{}}\PY{o}{\PYZgt{}} \PY{p}{(}\PY{n}{x}\PY{o}{\PYZhy{}}\PY{n}{μ}\PY{p}{)}\PY{o}{/}\PY{n}{sd}\PY{p}{,} \PY{n}{a}\PY{p}{)}
\end{Verbatim}

\begin{Verbatim}[commandchars=\\\{\}]
{\color{outcolor}Out[{\color{outcolor}134}]:} 6-element Array\{Float64,1\}:
           -0.8164965809277261
            0.4082482904638629
            1.632993161855452 
           -0.8164965809277261
            0.4082482904638629
           -0.8164965809277261
\end{Verbatim}
            
    Julia accommodates optional arguments by allowing funtion arguments to
have default values, similar to \emph{R} and many other languages. The
value of an optional arguments dose not need to be specified in a
function call.

    \begin{Verbatim}[commandchars=\\\{\}]
{\color{incolor}In [{\color{incolor}135}]:} \PY{c}{\PYZsh{}\PYZsh{} A function with an optional argument. This is a recursive function,}
          \PY{c}{\PYZsh{}\PYZsh{} i.e., a function that calls itself, for computing the sum of the first n}
          \PY{c}{\PYZsh{}\PYZsh{} elements of the Fibonacci sequence}
          \PY{k}{function} \PY{n}{fibonacci}\PY{p}{(}\PY{n}{n}\PY{o}{=}\PY{l+m+mi}{20}\PY{p}{)}
              \PY{k}{if} \PY{p}{(}\PY{n}{n}\PY{o}{\PYZlt{}=}\PY{l+m+mi}{1}\PY{p}{)}
                  \PY{k}{return} \PY{l+m+mi}{1}
              \PY{k}{else} 
                  \PY{k}{return} \PY{n}{fibonacci}\PY{p}{(}\PY{n}{n}\PY{o}{\PYZhy{}}\PY{l+m+mi}{1}\PY{p}{)}\PY{o}{+}\PY{n}{fibonacci}\PY{p}{(}\PY{n}{n}\PY{o}{\PYZhy{}}\PY{l+m+mi}{2}\PY{p}{)}
              \PY{k}{end}
          \PY{k}{end}
          
          \PY{c}{\PYZsh{}\PYZsh{} sum the first 12 elements of the Fibonacci sequence}
          \PY{n}{fibonacci}\PY{p}{(}\PY{l+m+mi}{12}\PY{p}{)}
\end{Verbatim}

\begin{Verbatim}[commandchars=\\\{\}]
{\color{outcolor}Out[{\color{outcolor}135}]:} 233
\end{Verbatim}
            
    \begin{Verbatim}[commandchars=\\\{\}]
{\color{incolor}In [{\color{incolor}136}]:} \PY{n}{fibonacci}\PY{p}{(}\PY{p}{)}
\end{Verbatim}

\begin{Verbatim}[commandchars=\\\{\}]
{\color{outcolor}Out[{\color{outcolor}136}]:} 10946
\end{Verbatim}
            
    \begin{Verbatim}[commandchars=\\\{\}]
{\color{incolor}In [{\color{incolor}137}]:} \PY{n}{fibonacci}\PY{p}{(}\PY{l+m+mi}{20}\PY{p}{)}
\end{Verbatim}

\begin{Verbatim}[commandchars=\\\{\}]
{\color{outcolor}Out[{\color{outcolor}137}]:} 10946
\end{Verbatim}
            
    Function arguments determine its behaviour. In general, the more
arguments a funtino has, the more varied its behaviour will be. Keyword
arguments are useful because they help manage function behaviour;
specifically, they allow arguments to be specified by name and not just
position in the function call. In the below code block, an MM algorithm
is demonstrated. Note that we have already used MM algorithm, but now we
construct an MM algorithm as a functino. MM algorithms are blueprints
for algorithms that eiger iteratively minimize a majorizing function or
iteratively maximize a minorizing functino -- see
\href{https://amstat.tandfonline.com/doi/abs/10.1198/0003130042836}{Hunter
and Lange (2000,2004)} for further details.


    % Add a bibliography block to the postdoc
    
    
    
    \end{document}
